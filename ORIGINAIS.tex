\subsection{PREFÁCIO}

\textbf{UM LIVRO QUE SE ABRE AO INESPERADO E AO ACIDENTAL}

1 A um pequeno texto que vem antes do texto principal pede-se clareza e
pistas de leitura, mas na verdade, percor- ridos estes nove contos,
sinto-me tão eufórica e desnorteada como as personagens: impressões
dispersas, caixas dentro de caixas dentro de caixas, chaves sem portas e
portas sem chave; tudo isso criando imagens e sensações que seguem
caminhos

opostos.

2 Podia começar, por exemplo, pela família literária. Mas quem é a
família do Alexandre Andrade? Georges Perec?

Sim. Donald Barthelme? Sim. Eric Rohmer? Sim. Heinrich von Kleist? Sim.
Haverá outros, com certeza, mas destes en- contro vestígios:

\begin{itemize}
\tightlist
\item
  coisas triviais em confronto com coisas extraordinári- as (mas quais
  são quais?);
\item
  tendência para enumerações e listas;
\item
  uso intensivo de travessões e pontos-e-vírgulas;
\item
  sentido de jogo permanente (apanhar os dois pássaros que voaram, como
  Mónica; deixar a tese derivar para con-
\end{itemize}

\textbf{12 }CRISTINA FERNANDES

jecturas e versões alternativas, ao sabor das incógnitas que vão
surgindo, como Leda);

--- os nomes das personagens: quem, senão Alexandre Andrade, ousaria
juntar Vasco, Leda, Berenice, Klaus, Ezequias, Inácio, Tatyana, Ester,
Minerva, Luz, Samanta, Júlio e Penélope durante uns dias de Verão
tórrido em Coimbra? No entanto, não consigo demonstrar a relação
evidente.

Talvez porque nestas histórias isso pouco importa?

3 Passemos então ao interior dos \emph{quartos. }O que aí tem lu- gar
são narrações curtas (por dentro e por fora) e um bocado misteriosas. A
trama é, literalmente, uma filigrana em redor de um episódio sem
importância; o mistério é subtil, convive com o quotidiano, retira dele
força para crescer e tam- bém, sem darmos por isso, desaparecer. Em
``Quem anda a comer do meu prato?'' a rapariga troca a investigação do
título por uma ida arbitrária à National Gallery para ver
\emph{``Natureza-}

\emph{-morta com maçãs e romã'' }de Courbet apenas pela grandiosi- dade
das cores e porque pode escolher. Não sabemos a razão que move estas
personagens; no sentido contrário, porém, a cada página descobrimos a
importância do acaso, do aciden- tal, nas suas vidas dando azo a
comportamentos infinitos e imprevisíveis. É-nos oferecido um mundo
aberto, cheio de possibilidades e entusiasmo onde as causas e
consequências nem sempre encaixam umas nas outras.

4 Apesar de muito jovens, ou talvez por isso mesmo, perante um enigma ou
ameaça as personagens agem com ousadia e agilidade. Tomam as coisas mais
estranhas e ines-

UM LIVRO QUE SE ABRE AO INESPERADO E AO ACIDENTAL \textbf{13}

peradas que lhes acontecem como quem acaricia um cavalo que lhes surge
pela frente, vindo sabe-se lá de onde, depois decidem montá-lo e avançam
em ``L''. Ora, isto cria um ambi- ente literário raro de encontrar, que
vem de um tempo menos assertivo e privilegia a suspensão, um deixar-se
ir, deixar-se to- car. O mistério é luminoso, solar, quase alegre; por
certo todas estas personagens são capazes de ver o raio verde.

5 Depois destas considerações tremendamente refutáveis, termino com a
objectividade própria dos números sem, contudo, obedecer a nenhum
critério em particular. Apenas me diverti a contar quantas vezes algumas
palavras aparecem em ``Quartos Alugados'', eis o resultado: a palavra
``livro'' aparece 52 vezes; olhos 49, janela 40, conversa 40, Paris 33,

acaso 25, café 23, cama 23, mesa 21, música 21, cadeira 19,

cozinha 19, detalhe 19, copo 18, chuva 18, cabelo 17, leis 15,

jardim 13, silêncio 13, chá 13, escola 12, sorriso 11, gato 10,

biblioteca 10, razão 10, teoria 9.

\emph{The world is not with us enough.}

(Denise Levertov, \emph{O Taste and See })

Este livro é para a Alexandra.


\subsection{QUARTOS ALUGADOS}

\textbf{IN ABSENTIA}

m momentos de dúvida ou embaraço íntimo, ele repetia para si ou (mais
raramente) para outros:

\begin{itemize}
\tightlist
\item
  Nada disto foge ao normal. Nas alturas críticas, mo- mentos de
  ruptura, novos começos, o detalhe mais ínfimo pode erguer-se a símbolo
  e durar o resto da vida. Pode ser uma melodia, um recado, uma mancha
  de luz. No meu caso, foi uma gaveta aberta e uma luva dentro dela.
\end{itemize}

A gaveta pertencia a uma secretária que estava numa das divisões do
pequeno apartamento que ele visitara e que se decidira a arrendar.
Andava de um lado para o outro, deva- gar, como quem assimila as
características e cambiantes de um lugar que poderá vir a ser o seu. Era
um T1 virado a sul, acanhado mas com uma vista agradável para uma praça
sosse- gada. O senhorio manifestava o seu tacto pela maneira como se
apagava, como se fazia pequenino junto à porta de entrada, aguardando o
final da inspecção.

Abrira e fechara portas e armários, ao acaso. Aquela fora a primeira
gaveta que abrira. Lá dentro estava uma luva de pelica, da mão esquerda.

\begin{itemize}
\tightlist
\item
  Que luva tão pequena ---, pensara ele. --- Só deve ser-
\end{itemize}

\textbf{20 }ALEXANDRE ANDRADE

vir em mãos muito pequenas, em mãos minúsculas.

Quando fechou negócio com o senhorio (um cavalheiro de olhos e linhas do
rosto fatigados, que arrastava as frases mas fazia questão em levá-las
sempre até ao seu fim natural), a mão minúscula, erguida e aberta,
pairava no seu campo de visão.

Péricles fora colocado em Tondela, numa região que ele não conhecia.
Cabia-lhe ensinar Matemática a adolescentes do ensino secundário.
Gostava da Escola Secundária José Rel- vas, aonde chegava todos os dias
após um curto trajecto a pé que interrompia a meio para comer um bolo e
beber um café numa confeitaria. A integração na nova escola tinha
corrido bem. Péricles fazia amizades com facilidade. Os alunos, ou pelo
menos alguns deles, estavam naquele limbo entre a malí- cia ostentada e
a tentação da franqueza, um limbo mais largo e mais povoado do que
muitos julgam.

Péricles tinha 35 anos. Há quem se finja indiferente à aproximação do
ponto médio de uma vida, porém roído por dentro pelo medo. No caso do
professor de Matemática, a in- diferença que mostrava ao mundo era o
reflexo fiel dos seus estados de alma, ocultos mas confessáveis.

Depois da luva, veio o iogurte. Péricles encontrou um iogurte grego no
frigorífico na noite do dia em que se mudou para o apartamento com vista
para a praça. O iogurte estava no último dia de validade. Péricles
comeu-o, de pé, enquanto olhava pela janela. Saboreou cada colherada
lenta.

O apartamento vinha totalmente equipado com mobília e electrodomésticos.
Depois da luva e do iogurte, Péricles en- controu uma saqueta de chá de
menta entalada entre o colchão e o espaldar da cama. Esperou alguns dias
antes de fazer uma infusão com a saqueta.

--- Este sabor... Este aroma...

IN ABSENTIA \textbf{21}

Era surpreendentemente fácil imaginar uma mão peque- nina fechada sobre
a asa da caneca fumegante, naquele mes- mo apartamento, naquela mesma
cadeira em que Péricles se sentava.

Péricles aproveitou um pretexto para telefonar ao senho- rio. Inseriu na
conversa perguntas, que tentava fazer parecer casuais, sobre a anterior
ocupante da casa. O senhorio não era daqueles que se armam de
preconceitos contra a indiscrição: partilhou com Péricles tudo o que
sabia, que aliás não era muito.

\begin{itemize}
\tightlist
\item
  Era moça recatada, nunca causou problemas. Pagava a horas e tratou bem
  dos móveis. Não a vi mais do que um pu- nhado de vezes. Numa dessas
  vezes, quando fui consertar as persianas da cozinha, contou-me que a
  família era originária das redondezas mas que ela nasceu em Lisboa.
  Partiu de forma abrupta, sem o pré-aviso contemplado no contrato.
  Ainda as- sim, reembolsei-a do mês de caução. Ela merecia. Moça es-
  plêndida.
\end{itemize}

Resumia-se a isto o conhecimento do senhorio sobre a an- terior
inquilina. E o nome dela?

\begin{itemize}
\tightlist
\item
  Começava por um M... Marta? Mónica? Sim, era isso mesmo: Mónica.
\end{itemize}

Mónica!

Os lábios de Péricles moviam-se para pronunciar o nome enquanto cumpria
o seu trajecto de todos os dias, indiferente à chuva miúda que salpicava
de gotículas o seu rosto e o seu cabelo. Na escola, o dia correu-lhe
muito bem. Um colega, professor de Física e Química, cuja atitude
Péricles estranhara desde a sua chegada, misto de indiferença estudada e
paterna- lismo, surpreendeu-o com um gesto generoso. Num segundo da aula
da manhã, sentiu-se subitamente a transbordar de feli-

\textbf{22 }ALEXANDRE ANDRADE

cidade e envergonhado pela sua condição de receptáculo, em- bora
modesto, de tanta felicidade.

Péricles manteve-se atento a oportunidades para travar conhecimento com
os vizinhos de cima e de baixo. No andar de baixo vivia um casal de
meia-idade, ambos reformados. Eram muito simpáticos. Convidaram Péricles
para tomar um café. Falavam com gosto do seu passado nas danças de
salão, lamentavam uma fractura e uma condição cardíaca que os tinham
forçado a abandonar a dança. De Mónica, só tinham bem a dizer. Evocaram
alguns dos seus traços físicos: baixa mas elegante, cabelo liso e
escuro, maquilhagem elaborada mas não excessiva.

Tinham-se habituado a ouvir o ruído dos passos de Móni- ca pela noite
dentro. Não era um incómodo, mas pergunta- vam-se o que a levava a
calcorrear o apartamento horas a fio, sem uma pausa sequer para
descansar.

Só muito de longe em longe Mónica recebia visitas.

No andar de cima viviam dois jovens. Um era farmacêu- tico, o outro
parecia desocupado. Pareciam pouco dispostos a falar sobre Mónica, mais
por falta de conhecimento de causa do que por escrúpulo. O farmacêutico
disse que tinha perdido a conta às vezes em que recebera encomendas de
livros dirigi- das a Mónica.

--- Dezenas e dezenas de livros. O carteiro não a encon- trava em casa e
tocava para aqui.

Mónica agradecia sempre o favor com uma polidez que pertencia a outros
tempos.

Bate certo, dizia Péricles de si para si. Está de acordo com... Mas como
fazer para se inteirar da ocupação de Mónica, daquilo que a trouxera a
Tondela e daquilo que a levara a par- tir? A resposta a uma destas
questões conteria a resposta às

IN ABSENTIA \textbf{23}

restantes? Péricles detestava encontrar-se numa situação em que dependia
de um capricho do acaso, porém desta vez o acaso mostrou-se benigno. Na
sala de professores, escutou a queixa de uma professora de Inglês
relativa a uma ajudante de cabeleireira, ``rapariga franzina'' dotada de
``umas mãozinhas de anjo'' que desaparecera de maneira inesperada. As
probabi- lidades e o bom senso jogavam contra ele, mas Péricles não quis
deixar de explorar aquele débil filão, até porque o ca- beleireiro
ficava em caminho. Foi na hora de almoço desse mesmo dia que decidiu
entrar.

\begin{itemize}
\tightlist
\item
  Tenho uma coisa para dar à Mónica. Tem de ser en- tregue em mão. É um
  assunto importante. Ela trabalha aqui, não é?
\end{itemize}

Três mentiras, seguidas de uma pergunta veloz e tensa que serviu para
expulsar o remorso do impostor e arcar com o peso das suas esperanças. A
dona do cabeleireiro olhou para ele sem expressão.

\begin{itemize}
\tightlist
\item
  A Mónica foi-se embora e não deixou morada.
\end{itemize}

Um arrepio de entusiasmo inteiriçou o corpo de Péricles. Olhou para
objectos e paredes com um respeito novo. Uma cliente retribuiu-lhe um
olhar que parecia vir das profundezas de um qualquer abismo. Estava
sentada numa cadeira, envolta numa bata. Madeixas púrpura desenhavam-se
sobre a prata cobreada do seu cabelo. Péricles retirou-se, apressado.

Madalena. Madalena trabalhava como enfermeira no Hospital dos Covões, em
Coimbra. Vivia com os pais em Condeixa-a-Nova. Vinha visitar Péricles
aos fins-de-semana, sempre que o escalonamento dos turnos lho permitia.
Condu- zia o seu Fiat Punto com prudência pelas estradas nacionais,
seguia o curso do rio Mondego e atravessava-o. A cama da casa de
Péricles em Tondela era estreita, mas era ali que dormiam e

\textbf{24 }ALEXANDRE ANDRADE

se amavam. Entregavam-se ao amor físico com o abandono da fadiga.
Aproveitavam os domingos para explorar as redon- dezas no carro de
Madalena. Ficaram a conhecer Águeda, Seia, Santa Comba Dão. Gostavam de
parar em sítios improváveis e ficar dentro do carro durante uma hora ou
duas, a conversar, ouvir CDs e beber café de um termos.

A primeira carta anónima chegou numa sexta-feira de tempestade. Péricles
segurou a carta na ponta dos dedos, mais preocupado em mantê-la longe da
água que escorria dos seus cabelos e do seu impermeável do que em
espreitar remetente e destinatário. O nome de Mónica e o endereço
estavam escritos numa caligrafia neutra. A carta estava impressa em
maiúsculas numa folha de papel dobrada em quatro. Dizia assim:

SABEMOS QUEM TU ÉS E SABEMOS QUE SABES QUEM NÓS SOMOS. É INÚTIL

TENTARES ESCONDER-TE DE NÓS, E ISSO VALE TANTO PARA UMA CIDADE PEQUENA
COMO

PARA O IMENSO MUNDO. OS ACTOS, SABES, TÊM CONSEQUÊNCIAS. TU QUERES UM
PÁSSARO NA MÃO E OS DOIS QUE VOARAM COMO SE ISSO FOSSE POSSÍVEL.
QUEREMOS VER-TE NO SÍTIO DO COSTUME ÀS 23 HORAS DO DIA 15. TRAZ AQUILO
QUE SABES. VEM SOZINHA. PARA QUÊ

ARMAR ESCÂNDALO?

O dia 15 era dali a três dias, mas qual seria o ``sítio do cos- tume''?
E qual seria o artigo que Mónica era instada a trazer com ela?
Pensativo, com o cabelo em desalinho por causa do

IN ABSENTIA \textbf{25}

vento e da chuva e sem que lhe ocorresse ligar a luz eléctrica, Péricles
percorreu o apartamento de um lado para o outro, lentamente. Através da
janela, avistou uma outra janela dis- tante e iluminada e por detrás
dela um rosto transtornado pela ansiedade. Quem seria aquele espectador,
obviamente tão atento à casa onde Péricles estava, a casa que Mónica
outrora habitara? Lembrou-se do testemunho dos vizinhos de baixo, das
deambulações nocturnas de Mónica pelo apartamento; lembrou-se dos livros
que chegavam para Mónica (``dezenas e dezenas'') e de que os vizinhos de
cima, gentilmente, tomavam conta quando ela não estava em casa para os
receber.

Uma leitora irrequieta, portanto. Passadas tensas acom- panhando o virar
das páginas. Talvez gestos, talvez frases em voz alta. E um espectador
febril do outro lado da rua, alguém a quem aquele ir e vir cadenciado de
Mónica preenchera uma parcela da vida, transmitira sentido, deixara
traço.

Péricles aproveitou o dia seguinte, sábado, para explorar o apartamento
com uma minúcia que se recriminou por não ter ainda aplicado. Cada palmo
quadrado de soalho envernizado, cada aresta de móvel foi alvo da mesma
atenção. Péricles notou que as prateleiras do quarto estavam abauladas
por uma de- pressão central: resultado da pressão de livros pesados
(catá- logos de museu? compêndios científicos?) ou de livros de porte
mais modesto mas durante um tempo considerável? Junto à escrivaninha
viam-se sulcos deixados pelas rodas da cadeira giratória. Péricles
imaginou Mónica sentada, entregando-se a devaneios, raspando o verniz do
chão numa cadência que era só sua, mas que ela ignorava e a que só um
observador externo poderia fazer justiça.

Na noite do dia 15, Péricles saiu à rua e caminhou ao acaso, como se
movido pela esperança de reconhecer aque-

\textbf{26 }ALEXANDRE ANDRADE

les que tinham marcado encontro com Mónica. Demorou-se em cafés
semivazios. Misturou-se com uma pequena multidão que assistira a um
recital lírico.

Foi no dia a seguir a esse, ao final da tarde, que Péricles recebeu uma
visita. Reconheceu-a pelas madeixas de cabelo pintadas de púrpura, um
púrpura mais carregado e menos subtil do que a tonalidade que guardara
na memória desde que a vira no cabeleireiro. Reparou agora que a mulher
era de meia-idade, de tez morena, corpulenta, proporcionada. Fora
certamente muito bela. Péricles deixou-a entrar em sua casa sem lhe
perguntar ao que vinha.

\begin{itemize}
\tightlist
\item
  Vim por causa da Mónica.
\item
  Eu sei.
\end{itemize}

As mãos pequenas de Mónica afagando aquele cabelo, tratando-o com
esmero, com um profissionalismo feito de gestos curtos e seguros a que
não faltava o carinho. O hálito mentolado de Mónica difundindo-se pelo
espaço do cabe- leireiro com a cadência de uma frase banal, uma conversa
de circunstância com a cliente sobre o estado do tempo ou sobre as
vicissitudes da vida. A preocupação secreta que a faz atrasar o gesto,
não mais do que uma fracção de segundo. Mónica suspende o movimento da
mão que segura a escova, retoma-o quando está prestes a aperceber-se da
pausa.

\begin{itemize}
\tightlist
\item
  Eu estava no cabeleireiro quando você apareceu a per- guntar pela
  Mónica. Quando soube que era o novo professor de Matemática, perguntei
  até ficar a saber onde morava. Esta é uma terra pequena. As coisas
  sabem-se.
\end{itemize}

Péricles sorriu, com receio de que ela se fosse desculpar. O sorriso era
um encorajamento para prosseguir.

\begin{itemize}
\tightlist
\item
  A Mónica era uma moça amorosa. Todos os que a conheceram sentem
  saudades dela e das suas mãos de
\end{itemize}

IN ABSENTIA \textbf{27}

fada. Mas não encontrará muita gente com vontade de pro- nunciar o nome
dela, ou de falar sobre ela.

\begin{itemize}
\tightlist
\item
  Mas porquê?
\item
  Ela meteu-se com quem não devia. Más companhias. Circularam várias
  versões. Houve quem falasse num grupo vindo da Europa de Leste, houve
  quem garantisse que se tra- tava de um bando ligado ao roubo de cobre.
  As pessoas falam de mais, mas desta vez havia um fundo de razão.
\end{itemize}

Péricles perguntou, de si para si, se o vizinho do outro lado da rua
estaria a seguir aquela entrevista e a especular so- bre o teor da
conversa, agarrando-se à hipótese, remota mas verdadeira, de estarem a
falar sobre a anterior ocupante do apartamento.

A visitante, agora de pé e levando a mão ao peito:

\begin{itemize}
\tightlist
\item
  Acha que ela está em maus lençóis? Acha?
\end{itemize}

A carta anónima (``TU QUERES UM PÁSSARO NA MÃO E OS DOIS QUE VOARAM COMO
SE ISSO

FOSSE POSSÍVEL'') não estava à vista. Péricles finalmente convenceu-a de
que sabia tanto ou tão pouco como ela sobre o destino de Mónica, e que
prolongar uma conversa entre duas pessoas igualmente carentes de
informação era, agradável ou não, um exercício fútil.

\begin{itemize}
\tightlist
\item
  Eu daria tudo por ela. Tudo. --- Foram as últimas pa- lavras da
  senhora antes de se despedir. Acariciando a luva de pelica, reclinado
  na cama estreita, Péricles não encontrou mo- tivo para suspeitar de
  exagero poético. Tudo. Tudo por ela.
\end{itemize}

Péricles ia agora para a escola como num transe. Chamamentos,
cumprimentos de pessoas conhecidas, aromas (o café na confeitaria do
costume, um maciço de rosas-chá num jardim particular a poucas dezenas
de metros da escola) sucediam-se na ordem esperada. Aulas e afazeres
burocráticos

\textbf{28 }ALEXANDRE ANDRADE

preenchiam eficazmente o dia. Surgiam problemas. Péricles ocupava-se dos
problemas. Nas reuniões, exprimia a sua opi- nião, por vezes com calor.
Ria-se das anedotas. Procurava con- sensos e meios-termos quando as
posições se extremavam. Ao final da tarde, Tondela começava a
recolher-se, em peso, como um pequeno animal que prepara a dormida.
Péricles lia versos pela noite dentro enquanto, mais metódico do que
empolga- do, calcorreava o apartamento. Lia versos malditos, rimas de pé
quebrado e sonetos simbolistas.

Péricles foi abordado na sala de professores pela colega de Português. À
parte e a meia-voz, como numa conspiração.

\begin{itemize}
\tightlist
\item
  Soube que andaste a fazer perguntas sobre a Mónica.
\item
  Moro na casa onde ela estava antes de partir.
\item
  Eu sei. Conheço alguém que se dava de forma muito íntima com ela: o
  Dr. Castro, um homem de letras da lo- calidade que se retirou e vive
  afastado do mundo, numa casa isolada à beira da estrada para o
  Caramulo. Não sei que tipo de ascendente tinha ele sobre ela, mas era
  comum vê-los aos dois numa esplanada ou caminhando lado a lado. O Dr.
  Cas- tro tem uma legião de discípulos que o rodeiam como insectos
  quando ele desce a Tondela, mas só a Mónica tinha artes de romper
  aquela couraça de desencanto e ironia ácida.
\item
  Estás a dizer-me que...
\item
  Estou a dizer-te que, neste momento, a única via para te aproximares
  da Mónica passa por este homem.
\item
  O carro está com a Madalena. Uma casa isolada no Caramulo ou a selva
  birmanesa são-me igualmente inaces- síveis neste momento.
\item
  Empresto-te a minha bicicleta, se quiseres.
\item
  Isso era simpático.
\item
  Está tudo bem com a Madalena?
\end{itemize}

IN ABSENTIA \textbf{29}

\begin{itemize}
\tightlist
\item
  Está muito cansada porque trabalha por dois. Aceita demasiados turnos
  de noite. Mas anda satisfeita.
\item
  Ainda bem. Se quiseres, passa por minha casa hoje ao fim da tarde para
  eu te dar a bicicleta.
\end{itemize}

A bicicleta tinha as mudanças avariadas. Péricles levou mais de meia
hora para pedalar dez quilómetros. A tal casa isolada era afinal uma
espécie de casebre quase em ruínas, onde era difícil imaginar que alguém
pudesse viver. Péricles encostou a bicicleta a uma árvore e aproximou-se
da casa. Não havia um único sinal de vida. Péricles espreitou para o
interior através de uma janela, mas era impossível distinguir fosse o
que fosse.

Não se ouvia qualquer ruído a não ser o dos carros que passavam na
estrada, a poucos metros de distância.

Teria Mónica alguma vez vindo ali? Por que lado? Pé ante pé? Olhando em
seu redor? Com que trejeitos de apreensão, palpitações de expectativa?

Péricles escutou um rumor de passos. Olhando para trás, avistou um homem
que, como ele próprio fizera, se aproxi- mava da casa. Viu também um
Honda prateado estacionado a algumas dezenas de metros. O rosto do homem
era-lhe fami- liar. O sorriso transmitia cumplicidade e resignação.

\begin{itemize}
\tightlist
\item
  A gaiola está vazia, não é? Não estou surpreendido. Era de esperar.
  Não madrugámos o suficiente. Mas houve quem tivesse tentado. Está a
  ver estas pegadas? Estes rastos de pneus? Alguém esteve aqui. Pode ser
  que tenham conseguido falar com ele, pode ser que não tenham. Eu até
  seria capaz de descobrir a marca do carro, se quisesse perder tempo
  com essas coisas, mas de que serviria isso?
\item
  Você também vinha à procura do...
\item
  \ldots{}do excelente Dr. Castro, bem entendido. Não me
\end{itemize}

\textbf{30 }ALEXANDRE ANDRADE

está a reconhecer, pois não? Sou o pai do Martim. Conversei consigo na
última reunião de encarregados de educação.

\begin{itemize}
\tightlist
\item
  O Martim, claro. Ele tem feito progressos.
\item
  O rapaz até é esperto, mas não estuda, o que é que se há-de fazer?
\item
  Tem de se aplicar mais.
\item
  Ele não tem grande cabeça para os números, mas pode fazer melhor.
\item
  O que será feito do Dr. Castro?
\item
  Quem sabe? Mas digo-lhe uma coisa: ou me engano muito, ou não o vamos
  ver por Tondela nos próximos tempos. Uma coisa é ele estar rodeado por
  discípulos e admiradores, outra coisa é ser assediado por pessoas que
  não vêem nele mais do que uma etapa do caminho que conduz até à
  Mónica. Se ele tiver um pingo de juízo, irá afastar-se durante uns
  tempos. E sabe que mais? Estamos a perder o nosso tempo aqui. Venha
  daí, dou-lhe boleia.
\item
  Parece-me que a bicicleta não cabe na mala do seu carro.
\end{itemize}

Não cabia, com efeito.

\begin{itemize}
\tightlist
\item
  Então a gente vê-se por aí, senhor doutor.
\item
  Os meus cumprimentos ao Martim.
\end{itemize}

O percurso de volta era a descer. Péricles deixou-se ir, ima- ginando
que regressava a uma cidade repleta de enigmas mas nunca cruel o
suficiente para sonegar as soluções desses enig- mas àquele que se
dispusesse a encontrá-las, munido apenas das suas mãos nuas, do seu
engenho humano, do seu corpo vertical estremecido muito ao de leve pela
pulsação.

Ao chegar a casa, tinha à sua espera a segunda carta anónima.

IN ABSENTIA \textbf{31}

NÃO COMPARECESTE AO ENCONTRO.

A CADEIRA VAZIA PARECIA A PONTO DE SE TRANSFORMAR NA TUA PRESENÇA VIVA
MAS ISSO NÃO ACONTECEU. ASSUMES A DECISÃO DE

TRANSFORMAR TONDELA NUMA CIDADE DESPROVIDA DE TI E DA TUA REALIDADE? MAS
A TUA DECISÃO NÃO É O ALFA E O ÓMEGA

E TEMOS ASSUNTOS EM ABERTO. SÃO ASSUNTOS VULGARES E PEQUENINOS MAS ESTÃO
EM ABERTO. ESTAREI NO CAFÉ DE NANDUFE, AQUELE QUE TEM O TOLDO AMARELO,
NO DIA VINTE ÀS SETE DA TARDE. ESPERAREI LÁ POR TI. TRAZ AQUILO QUE
SABES E EU TRAREI AQUILO QUE SABES. NÃO FALTES, MÓNICA.

Pareciam tão longas a Péricles, as horas que ainda o sepa- ravam do dia
20. Péricles fazia oscilar a cadeira de escritório com uma doçura
infinita, no sentido horário e no sentido anti-

-horário. Nem uma luz na casa do vizinho do outro lado da rua. No andar
de baixo, ruído de pés calçados com sapatos de dança, movendo-se a
compasso. Sobrava um iogurte no fri- gorífico, que Péricles comeu
enquanto escrevia uma nota para não se esquecer de comprar laranjas para
Madalena, que viria no dia seguinte.

Madalena achou Péricles mais magro.

\begin{itemize}
\tightlist
\item
  É extraordinário aquilo que consegues emagrecer nu- ma semana.
\end{itemize}

Choveu durante todo o fim-de-semana. Madalena e Péricles passaram quase
todo o tempo a ler na cama estreita

\textbf{32 }ALEXANDRE ANDRADE

e a ouvir música.

\begin{itemize}
\tightlist
\item
  Há aqui um artigo interessante sobre o restauro do tecto da Capela
  Sistina.
\end{itemize}

No dia seguinte, segunda-feira, dia 20, Péricles chamou o aluno Martim
quando este se preparava para abandonar a sala de aula.

\begin{itemize}
\tightlist
\item
  Sabes, vi o teu pai no outro dia. Ele contou-te?
\item
  Não...
\item
  Íamos à procura da mesma pessoa. Sabes se ele teve mais notícias
  desse... do Dr. Castro?
\end{itemize}

Mas era claro que a criança não estava a par de nada. Péricles deixou-o
ir, para ser envolvido pela curiosidade dos colegas.

Agradeceu à colega de Português o empréstimo da bici- cleta e
perguntou-lhe se se ia bem a pé até ao café de Nandufe. Às sete horas já
era noite escura. Péricles enveredou por duas vezes por um caminho
errado, mas chegou ao ponto de encontro em cima da hora. Sem medo de se
enganar, com um estremecimento de familiaridade, dirigiu-se para a única
cadeira desocupada que estava à vista. A cadeira estava ligeira- mente
afastada da mesa de café, ao ar livre apesar da noite fria. Só quando se
estava a sentar Péricles olhou para o homem que estava sentado na outra
cadeira, com uma imperial bebida até meio na mão. O homem não mostrou
surpresa nem hostili- dade. O seu rosto não tinha expressão e parecia
não ter idade.

Vestia uma camisa de linho fina, apesar do frio da noite.

\begin{itemize}
\tightlist
\item
  Não basta um homem sentar-se numa cadeira para ocupar um lugar ---
  disse ele.
\item
  A Mónica não pôde vir --- disse Péricles.
\item
  Com certeza que não pôde vir. Claro que não pôde vir. Era fatal, e
  esta noite estava mesmo a pedir uma fatalidade.
\end{itemize}

IN ABSENTIA \textbf{33}

Seria de uma grande leviandade pretender combater o destino com as suas
próprias armas. Quem é você?

\begin{itemize}
\tightlist
\item
  Um amigo da Mónica.
\item
  A mensagem falava da Mónica, da urgência em que ela comparecesse, de
  uma relação de confiança mútua que existe entre nós. É de um contrato
  que se trata, selado com o tempo, inviolável. Compreende isto, senhor
  amigo da Mónica?
\item
  A Mónica é uma mulher fiel aos compromissos, mas por vezes surgem...
  Como dizer? Circunstâncias que não do- minamos.
\item
  Circunstâncias.
\item
  Circunstâncias.
\item
  Pois seja. Somos os dois vítimas das circunstâncias. É às
  circunstâncias que devemos estar aqui nesta noite de cão, gelados até
  aos ossos, sem vontade de falar ou de olhar o outro nos olhos.
  Cerveja?
\item
  Pode ser, obrigado.
\end{itemize}

A cerveja vinha a transbordar espuma. Péricles bebeu-a com sofreguidão,
depois com demora, quase com ternura. Começou a cair uma chuva miúda.

\begin{itemize}
\tightlist
\item
  A Mónica tem uma coisa de que nós precisamos. Há gente muito poderosa
  e muito influente que começa a dar si- nais de impaciência. Eu só
  quero o bem da Mónica, mas ou- tros não terão qualquer escrúpulo em
  fazer o que for preciso para recuperar a...
\item
  A coisa.
\item
  A coisa de que nós precisamos.
\item
  Não trouxe nada comigo.
\item
  Nem era preciso dizer isso. Trata-se de uma coisa que a Mónica não
  confiaria nem ao companheiro de armas mais fiel. Péricles queria
  regressar a casa o mais depressa possível,
\end{itemize}

\textbf{34 }ALEXANDRE ANDRADE

antes que a chuva ganhasse intensidade. --- Queria dar-lhe isto. É para
a Mónica.

Péricles deixou que o outro lhe pousasse na palma da mão um animal de
barro. À luz frouxa do único candeeiro público, Péricles julgou
tratar-se de um réptil de espécie indefinida.

\begin{itemize}
\tightlist
\item
  Diga-lhe que isto não faz parte do contrato --- o rosto do homem
  deformava-se agora numa careta de despeito e an- gústia ---, diga-lhe
  que estou a dar algo em troca de nada, e que se isto não é um penhor
  de respeito e, não me importo de usar a palavra, de AMOR, não sei o
  que isso é. Diga-lhe isto.
\end{itemize}

E afastou-se, a corta-mato, na direcção de Santa Ovaia. Já no centro de
Tondela, num largo que ficava entre o

Hospital Cândido de Figueiredo e a Igreja Paroquial, Péricles ouviu um
alarido pouco habitual nas tranquilas noites ton- delenses. Intrigado,
dirigiu os passos para o foco da agitação. Junto à entrada de um prédio
de habitação, uma mulher to- cava compulsivamente à campainha de um dos
andares e batia freneticamente na porta com a palma da mão. Dos dois
lados da rua, várias pessoas assomavam às janelas ou debruçavam-se nos
peitoris das varandas para tentar perceber o que se passava e dirigir
admoestações à perturbadora da ordem pública. Péri- cles ouviu chamar
pelo seu próprio nome. Voltando-se, reco- nheceu um funcionário
administrativo da escola que assistia ao espectáculo da varanda, de
roupão e pantufas, com uma calma olímpica.

\begin{itemize}
\tightlist
\item
  É o senhor professor Péricles, não é? Boa noite para si.
\item
  Boa noite.
\item
  Consegue calar essa louca? Está nisto vai para uns bons vinte minutos.
\end{itemize}

Ao aproximar-se, Péricles apercebeu-se de que a conhecia. Apesar da
noite sem luar, as madeixas de cor púrpura distin-

IN ABSENTIA \textbf{35}

guiam-se com nitidez, húmidas de água da chuva. Era a cli- ente do
cabeleireiro.

Péricles aproximou-se. Ela também o reconhecera. Olhou-

-o com olhos invadidos pelo desespero, voltou-se e comprimiu o corpo
contra a porta de vidro e metal. Soluçava.

\begin{itemize}
\tightlist
\item
  O que se passa? Posso ajudá-la?
\item
  Não me esconda nada, por favor. Sabe se ela está em casa ou não?
\item
  Mas quem? Quem é que está em casa.
\item
  Ainda não lhe contaram? Como é possível? É a so- brinha do Dr. Castro.
  Ela vive aqui.
\item
  A sobrinha do Dr. Castro?
\item
  Ela é advogada. Tem escritório em Viseu, mas passa cá temporadas.
  Diz-se que está na posse de informações sobre o paradeiro do tio.
\end{itemize}

As feições descompostas, a voz estridente, fizeram com que Péricles,
quase sem querer, pousasse a mão num gesto que ele queria tranquilizador
no ombro da mulher.

\begin{itemize}
\tightlist
\item
  Vá para casa. Isso foi um rumor, nada mais.
\item
  Foi muito mais do que um rumor. E mesmo que fosse? Uma hipótese ínfima
  de que houvesse um fundo de verdade valeria todos os sacrifícios do
  mundo.
\item
  O Dr. Castro desapareceu. A Mónica desapareceu.
\item
  A Mónica... Será verdade que ela anda metida com uma rede de apostas
  clandestinas? Diz-se cada coisa. Essa gen- te não recua diante de
  nada. Se ela for um meio para um fim, eles não hesitarão em
  espezinhá-la.
\item
  Rumores, nada mais que rumores.
\item
  Acha que isto tudo pode ser verdade? Será que ela corre perigo?
\end{itemize}

A voz dela era agora pouco mais do que um arfar. Ouviam-

\textbf{36 }ALEXANDRE ANDRADE

-se persianas a fechar, cortinas eram corridas, luzes apaga- vam-se.
Péricles sentiu-a desfalecer, ajudou-a a sentar-se num degrau.

\begin{itemize}
\tightlist
\item
  Porque é que as coisas têm de ser assim?
\item
  Não sei. Mas a esperança é uma coisa dura, que teima em permanecer.
\item
  É bom estar ao pé de si, é bom estar com alguém que vive na casa onde
  ela morava, pisa o mesmo chão que ela, dia após dia. Faz-me sentir
  mais próxima da Mónica.
\end{itemize}

Péricles ofereceu-se para a levar a casa. A chuva tinha parado. Olharam
ambos para trás uma única vez.

\begin{itemize}
\tightlist
\item
  Portanto, é aqui que vive a sobrinha do Dr. Castro --- disse Péricles.
\item
  Não a veremos nunca mais. Foi uma inconsciência da minha parte vir
  aqui fazer esta figura triste. Quanto aos dis- cípulos do Dr. Castro,
  aqueles lambe-botas que disputavam à cotovelada e ao encontrão os
  lugares ao lado do grande homem nas esplanadas... Também a esses
  duvido que lhes voltemos a pôr a vista em cima. Eles sabem que, à sua
  maneira, são elos de uma cadeia que conduz até à Mónica. Mostrarem-se
  em público seria um acto de temeridade.
\end{itemize}

Ao entrar em casa, Péricles ainda teve tempo de avistar o quadrado de
luz, imediatamente extinto, da janela do outro lado da rua. Retirou o
réptil de barro do bolso do imper- meável, constatou com agrado que não
sofrera com a chuva, colocou-o em cima da secretária. Ficou surpreendido
quando verificou que tinha duas chamadas não atendidas de Madale- na no
telemóvel. Como já era tarde e não a queria acordar, limitou-se a enviar
um correio electrónico radiante de paixão, carinho e saudade.

Ficou a pé até às quatro da madrugada, entretido a pes-

IN ABSENTIA \textbf{37}

quisar na Internet factos sobre cerâmica e mitologia das civi- lizações
pré-colombianas. Convencera-se de que a estatueta do réptil tinha algo
de maia ou de azteca. Encontrou mais motivos para reforçar esta
convicção do que para a rejeitar. A massa de informação que chegava ao
ecrã de computador dançava e descrevia nuvens rodopiantes diante dos
seus olhos toldados. Dormiu um sono ligeiro, que uma rajada de vento ou
o fluir de água num cano teria bastado para interromper.

No dia seguinte, chegou à escola mais cedo do que era seu costume. Na
secretaria, consultou discretamente o horário de ocupação do laboratório
de Química. Avançou um pretexto vago para pedir a chave do laboratório à
auxiliar da função educativa, sorriu em silêncio à laia de resposta à
estranheza natural de quem nunca tinha visto o professor de Matemática
entrar naquela sala.

A estatueta do réptil apresentava, no dorso e na cauda, manchas
minúsculas de um tom entre o vermelho e o casta- nho. Péricles fez uma
raspagem cuidadosa, examinou demo- radamente as amostras ao microscópio,
tirou notas abundan- tes. Contemplou demoradamente o armário dos
reagentes químicos. Talvez houvesse ali o necessário para averiguar da
natureza daquelas manchas, mas os seus conhecimentos não chegavam para
tanto e aproximava-se a hora da sua aula.

À vista desarmada, observou um pequeno orifício que perfurava a
estatueta de lado a lado, à altura do pescoço. Obra de um berbequim de
precisão ou de uma ferramenta manual? Teria o objecto servido de enfeite
de colar?

Ao almoço, sentou-se ao lado da colega de História, da boca de quem uma
vez ouvira comentários sobre mitologia (ou seria animismo ou
xamanismo?), no meio de uma conversa desgar- rada e sem sequência.
Tentou dirigir a conversa para temas

\textbf{38 }ALEXANDRE ANDRADE

aparentados com esses, mas sem qualquer sombra de suces- so. Falou-se de
revistas, de romances, de sabores de gelado, dos pioneiros do voo
transatlântico, da descoberta da insulina, de Viseu, das Caldas da
Rainha, de Ferreira do Zêzere, de computadores portáteis, de agricultura
biológica, da consti- tuição da República Portuguesa, de Geraldo
Geraldes o Sem Pavor, de gengibre, de cardamomo, de noz-moscada, de
Sérgio Godinho, de taxas de juro bonificadas. Péricles defendeu, com
calor, alguns pontos de vista controversos sobre política fiscal e
macroeconomia.

No corredor, Péricles avistou Martim ao longe, distraído e vergado sob o
peso da mochila. Péricles fez-lhe um aceno discreto, provavelmente
invisível.

Péricles reconheceu o envelope assim que abriu a caixa de correio. Sem
selo, decerto entregue em mão. Abriu-o com o dedo enquanto subia as
escadas.

E ASSIM É. TUDO ESTRANHAMENTE NO SEU LUGAR. EIS POIS A TUA OBRA, MÓNICA.
NÃO TE BASTOU RASGAR O VÉU DO UNIVERSO, QUISESTE ESFREGÁ-LO NO ROSTO DOS
TEUS DESERDADOS. É TALVEZ UM DIREITO QUE TE

ASSISTE, ASSIM COMO O DE SUBTRAIR À CIDADE A TUA CARA DE MENINA
CARIDOSA. TINHAS TODAS AS CARTAS NA MÃO, PARA QUÊ FAZER \emph{BLUFF?
}UMA VÉNIA PARA TI, MÓNICA, UMA VÉNIA QUE TE PERSIGA E QUE TE VISITE
COMO A CHUVA DOURADA DO MITO. ESTAMOS

ESTRANHAMENTE QUITES. NÃO NOS DEVES SEJA O QUE FOR. DÍVIDAS E VÍNCULOS
DE

IN ABSENTIA \textbf{39}

HONRA TAMBÉM DEFINHAM, COMO ANIMAIS OU FLORES. PARA QUÊ GUARDAR RANCOR?
MEDO E CÓLERA SÃO DORAVANTE OS MEUS NOMES, MAS CÓLERA DIRIGIDA CONTRA

O MUNDO E CONTRA MIM. NUNCA MAIS

TE VEREI. NUNCA MAIS DEIXAREI ESTA CIDADE. GUARDA E ACARINHA OS TEUS
MOTIVOS. DESPREZA O REMORSO. ESTOU BEM

E RODEADO DE COISAS QUE PREZO, MÚSICA E MAÇÃS VERDES. LEMBRARES-TE DE
MIM NÃO É IMPORTANTE.

Péricles deitou-se na cama, todo vestido, com a luz apaga- da.
Finalmente percebeu, ao fim de meia vida, que a tensão da batalha e o
apaziguamento podiam conviver no mesmo corpo. No dia seguinte, ao
regressar a casa, fatigado por causa de uma reunião de conselho
pedagógico na qual muito se vocife- rara e pouco se adiantara na
resolução de um problema espi- nhoso, Péricles foi surpreendido pela
presença de uma mulher que dormitava sentada no chão, do lado de fora da
porta. Era

Mónica.

\begin{itemize}
\tightlist
\item
  Estaria capaz de apostar que ele tinha mudado a fecha- dura, mas
  afinal a minha chave ainda roda. Não te preocupes, não entrei. Não
  passou de descargo de consciência.
\end{itemize}

Mónica, em pessoa, instalada na cadeira de braços, con- templando o seu
antigo apartamento.

\begin{itemize}
\tightlist
\item
  Gosto do que fizeste com a decoração. Eu nunca tive paciência para
  trocar bibelôs e gravuras de lugar, empirica- mente, até chegar à
  solução óptima. O meu sentido da estética navega noutros mares, mais
  remotos e caprichosos.
\end{itemize}

\textbf{40 }ALEXANDRE ANDRADE

Mónica, em quase tudo parecida com o retrato que dela fora construindo
Péricles, das feições bem desenhadas à mão diminuta e delicada, de dedos
curtos e fortes.

Mónica, satisfeita por poder partilhar o seu itinerário ín- timo de
lugares notáveis tondelenses, incluindo a confeitaria onde as bolas de
Berlim eram mais fofas e ricas em açúcar do que nas outras confeitarias.

Mónica explicando os seus dissabores, com uma naturali- dade que parecia
impossível supor afectada:

\begin{itemize}
\tightlist
\item
  não se tratava de apostas clandestinas nem de roubo de cobre, mas sim
  de um pequeno negócio de venda de pasti- lhas de \emph{ectsasy }aos
  frequentadores de discotecas locais; Mónica apenas se envolvera por
  interposta pessoa --- no caso vertente o baterista de uma banda de
  versões \emph{rockabilly }por quem se tomara de amores fugazes;
\item
  o indivíduo que enviara as cartas anónimas pouco mais era do que um
  tarefeiro do grupo, uma criatura instável em quem lhes custava
  confiar;
\item
  as referências a ``aquilo que sabes'', aparentemente tão ominosas, não
  passavam de fantasias;
\item
  o pequeno animal de barro era uma relíquia de um episódio passado,
  que, por envolver uma terceira pessoa, ela preferia deixar na sombra.
\end{itemize}

Mónica dando largas a uma inesperada propensão para os mexericos:

\begin{itemize}
\tightlist
\item
  um dos vizinhos de cima, não o farmacêutico que lhe costumava levar os
  livros, mas sim o outro, tinha uma vez fu- gido com um crítico de
  vinhos muito famoso que até escrevia nos jornais; durante as três
  semanas em que estivera ausente, o colega de apartamento tocara
  incessantemente, num volume altíssimo, aquilo que parecia ser a obra
  integral de Jacques Brel;
\end{itemize}

IN ABSENTIA \textbf{41}

\begin{itemize}
\tightlist
\item
  o vizinho de baixo fora condenado por desvio de fun- dos, com pena
  suspensa; nunca mais conseguira encontrar emprego depois desse
  episódio lamentável; não era apenas o seu coração débil que o afastava
  das pistas de dança;
\item
  a cliente do cabeleireiro que a idolatrava a ponto de passar noites ao
  relento era uma somítica patológica que prefe- ria arrecadar dinheiro,
  ou quando muito gastá-lo em bules de colecção que comprava no
  \emph{ebay, }do que ajudar a única filha que estava seis meses
  atrasada na renda e à mercê de uma acção de despejo.
\end{itemize}

Mónica em fotografias de férias: no Parténon, no Mé- xico, no palácio
dos Gonzaga em Mântua, num mercado de Tânger...

Mónica que afinal detestava infusão de menta. A saqueta que deixara em
casa era a última de uma caixa destinada a visitas.

Mónica sentada num banco rústico de madeira e verga, exprimindo
desconforto perante uma certa tendência do jor- nalismo de opinião em
Portugal: «Entregam-se a exercícios de retórica que assumem as roupagens
digníssimas de estruturas lógicas mas, quando se vai a ver, não cabem na
definição de dedução, nem sequer de indução; nem dão sinais de ousar
algo de mais inovador do ponto de vista formal, como a abdução
peirceana.»

Mónica atenta aos desabafos de Péricles sobre o dia de trabalho, a
desmotivação dos alunos, a sobranceria ou simples incompetência de
alguns dos seus colegas, as intermináveis burocracias associadas ao
processo de avaliação, o desgaste físico e anímico acumulado ao longo do
ano lectivo que agora terminava...

Mónica a jogar ténis: forte na aproximação à rede, mas

\textbf{42 }ALEXANDRE ANDRADE

traída por um \emph{backhand }deficiente.

Toda uma cidade em mansa metamorfose para acolher Mónica, a evidência de
Mónica, os trajectos de Mónica.

O ar entre as fachadas e o arvoredo, demasiado rarefeito para sustentar
mistérios.

--- Decidi voltar para esta casa quando te fores embora. Não me vejo a
morar em mais nenhum lugar. Falei com o senhorio e ele está de acordo.

Péricles concordou com a cabeça, como se esse desenlace fosse o único
possível. A voz de Mónica mal reverberava nas paredes.

Péricles despediu-se de Mónica numa tarde quente e seca que anunciava o
Verão. O aperto de mão foi dado de olhos nos olhos. O rosto de Mónica
era todo ele luz do sol, feições desenhadas num instante de perfeita
nitidez.

No ano lectivo seguinte, Péricles foi colocado em Vila Nova de Ourém. O
seu grande receio, que era o de ficar dema- siado longe de Coimbra para
ver Madalena com regularidade, não se concretizou.

Nos fins-de-semana e feriados, Madalena conduzia o Fiat Punto para sul,
ao encontro de Péricles. Faziam excursões de carro ou ficavam dentro de
portas consoante o tempo e a dis- posição.

Péricles encontrou uma escola bem equipada e em que se vivia um espírito
dinâmico e um ambiente propício a projectos educativos de qualidade.

Madalena e Péricles ganharam o hábito de seguirem uma qualquer estrada
nacional ao deus-dará, pararem num sítio elevado com uma vista
interessante e ficarem dentro do carro durante horas, encostados um ao
outro. Por vezes dormita- vam, com música a tocar baixinho. Acordavam à
vez, pensa-

IN ABSENTIA \textbf{43}

vam nos acontecimentos do dia anterior ou no futuro, nas suas
tonalidades de treva e luz.

MAIO 2011


\subsection{QUARTO ESCURO}

campainha da porta de casa soa como um zumbido doce, ou como um ronronar
tranquilo de gato doméstico, um gato que ronrona amiúde, durante o dia e
pela noite dentro. Não há hora para sair nem para entrar no grande
apartamento que partilhamos, que abrimos a todos os nossos amigos e a
todos aqueles que nos trazem a luz da sua presença e a melodia da sua
conversa. A discussão, os risos e os argumentos trans- bordam da sala,
por vezes para os quartos de cada um, invaria- velmente para a varanda
larga e longa que todos admiram e da qual se desfruta uma vista
belíssima (``escabrosamente bela'', disse alguém certa vez) sobre os
telhados, ruas e janelas da

Graça, sobre o horizonte urbano de Lisboa.

\begin{itemize}
\tightlist
\item
  Para que lado fica o Tejo? --- quis saber a Inge, um cotovelo a roçar
  o murete, um copo de \emph{Chablis }na mão. In- dicámos-lhe a
  direcção: algures por detrás daquela fachada, sem dúvida carregada de
  história e guardiã ciosa de existências apaixonadas.
\item
  Este bairro merecia coisas a acontecer --- proclamou, na sua voz
  potente, o Cristóvão. Metade do \emph{vol-au-vent }de perdiz
  permanecia esquecida no prato que ele equilibrava no
\end{itemize}

\textbf{46 }ALEXANDRE ANDRADE

colo. --- Agitação, histórias, uma movida, que sei eu!

--- Quem te diz que as coisas não acontecem? --- disse uma rapariga
baixa e sardenta que eu nunca tinha visto. Faria parte da comunidade de
leitores que a Vanessa frequentava?

\begin{itemize}
\item
  Coisas surpreendentes, coisas capazes de virar o mundo do avesso e
  abalar consciências.

  \begin{itemize}
  \tightlist
  \item
    Por detrás destas persianas inocentes --- disse eu, com um floreado
    de mão dirigido aos prédios que a vista alcançava
  \end{itemize}
\end{itemize}

---, destes cortinados de renda? Tenho as minhas dúvidas.

Entra-se no nosso apartamento para um vestíbulo onde ninguém se demora
mais do que é necessário para pendurar casacos e \emph{écharpes,
}abandonar guarda-chuvas num cilindro de latão pseudo-oriental.
Percorre-se o corredor mal reparando nos quartos: o do Cristóvão, o da
Inge e o da Maria Rosa, à esquerda; o meu, o do Emílio e o da Vanessa, à
direita. O corredor é longo, estreito e despido; é para a sala que todos
convergem, a sala enorme que espanta aqueles que a visitam pela primeira
vez. O espanto tem a ver com a amplitude, mas também com a acústica, com
a mobília, os objectos, a vista. A cozinha é pequena, mas cabem nela
todos aqueles que forem precisos para ensaiar um petisco novo, abrir
garrafas de vinho, continuar uma conversa boa de mais para ser deixada
para uma ocasião futura que provavelmente nunca surgiria.

Uma conversa sobre livros, por exemplo. Todos se interes- sam pelos
livros que os outros andam a ler. Pode passar-se um serão inteiro em
redor de uma frase, de uma personagem, de uma releitura oportuna, de uma
afinidade partilhada.

\begin{itemize}
\tightlist
\item
  Revisito Proust com muito prazer, mas abro uma ex- cepção para
  \emph{Sodoma e Gomorra }e \emph{O Caminho de Guermantes}. Ao fim de
  duzentas ou trezentas páginas de vacuidade social, dá vontade de
  gritar que sim, que aquele meio é mesquinho e
\end{itemize}

QUARTO ESCURO \textbf{47}

superficial até ao tutano, para quê insistir mais?

\begin{itemize}
\tightlist
\item
  Concordo com a Inge. Noutros volumes, a profusão faz sentido e
  acrescenta alguma coisa. Aqui, não.
\item
  Se bem que a \emph{Albertine...}
\item
  Mas justamente! A exaustão verbal, a exploração obses- siva da perda
  sentimental e da saudade...
\item
  Não faria qualquer sentido uma \emph{Albertine }sucinta.
\item
  A \emph{Recherche }foi sendo construída num movimento de expansão,
  desde o \emph{Contre Sainte-Beuve }até à morte do autor na Rue Hamelin
  (e não no quarto forrado de cortiça do Boule- vard Haussmann, como
  muitos pensam). É ocioso discutir se esta ou aquela parte resulta
  melhor. Faz parte da própria gé- nese da narrativa.
\item
  E esse modo de acumulação de experiências é o que mais se adequa ao
  percurso do narrador, das sensações para os signos... Vocês sabem,
  sempre achei que o Deleuze tinha razão por um lado e não a tinha por
  outro. O que quero dizer é...
\end{itemize}

O Emílio calou o sussurro agitado da sua voz para deixar passar a Maria
Rosa. A Maria Rosa vestia o seu eterno roupão de cor indefinida, gasto e
encardido. Sem olhar para ninguém, sem uma palavra, dirigiu-se à cozinha
e encheu um copo com água da torneira. Sorveu, mais do que bebeu, a água
e pousou o copo depois de o lavar com cuidado. As olheiras confundi-
am-se com a maquilhagem esborratada (lágrimas)? Regressou ao quarto e
fechou a porta, com um débil ruído de madeira contra madeira.

De forma inexplicável, o silêncio pesado fez ainda mais sumida a voz do
Emílio: --- Também é verdade que nada se compara àquela intensidade
sinistra do barão de Charlus nas \emph{Jeunes Filles en Fleur, }na praia
e depois quando entra no quarto do narrador, com o pretexto de emprestar
um livro de

\textbf{48 }ALEXANDRE ANDRADE

Bergotte. Sempre que releio essa parte, dou por mim a suster a
respiração. O medo e o embaraço do narrador são também meus.

Passava das duas da madrugada. Ninguém parecia ter vontade de dispersar.
Eu e um amigo do Cristóvão (indivíduo de olhar duro e porte sólido,
poucas palavras -- talvez mili- tar?) improvisámos uma ceia: tostas
integrais, boqueirões em vinagre, um resto de quiche de frango, papaia
cortada aos cu- bos...

A Vanessa diz que a Maria Rosa nunca pisca os olhos. Nunca consegui ter
a certeza se isso é verdade. Ora é a es- cassa luz do corredor, ora a
madeixa de cabelo que lhe cobre os olhos, ou então o seu hábito de
curvar a cabeça em ângulos estranhos.


\subsection{* * *}

A Inge veio para Portugal meio para estudar arquitectura urbanística,
meio para estar com um rapaz português. Co- nheceram-se quando ele
estava a fazer Erasmus na Alemanha. O amor não durou, mas a Inge ficou.
Agora anda com um repórter fotográfico. A relação parece sólida, mas a
Inge tarda em ir viver com o seu repórter e deixa-se ficar no quarto da
Graça, virado a sul. Não serei eu a censurá-la. O Emílio divide o tempo
entre uns estudos de comunicação social e uns biscates de que não nos
revela mais do que detalhes avulsos, quase sempre a roçar o escabroso ou
o tragicómico. Todos gos- tamos muito do Emílio. O Cristóvão dá a
impressão de estar sempre a falar de si próprio, mas quando vamos a ver
é muito

QUARTO ESCURO \textbf{49}

pouco o que sabemos sobre o seu passado. Conta-se que o pai fez fortuna
na Bolsa e que lhe deixou uma quantia de onde ele tira um rendimento,
modesto mas suficiente para pagar o quarto, a comida, os livros e as
camisolas Pedro del Hierro. Conta-se que estudou para barítono. Aquilo
que podemos todos constatar quotidianamente é que ele possui uma voz
sonora e cheia, assim como a disposição para a usar muito amiúde. As
suas palavras e as suas gargalhadas enchem a sala, reverberam, ganham
calor quando ele cede às provocações maliciosas da Vanessa. A Vanessa é
um encanto de pessoa e se há alguma coisa de misterioso nela não é
certamente o seu passado: ninguém ignora o seu denso historial de
desgostos e aventuras. Parece contente com um presente isento de sobres-
saltos em que as visitas, a comunidade de leitores, o convívio e as
discussões se encadeiam como numa partitura.

Sobre a Maria Rosa ninguém sabe o que quer que seja. Fe- cha-se no seu
quarto assim que chega a casa. Quando sai, vem apressada e sem vontade
de falar. Há qualquer coisa no olhar e nos modos que nos desencoraja de
lhe dirigirmos a palavra. O quarto da Maria Rosa é o mais pequeno de
todos. A porta não abre mais do que uma vintena de centímetros porque
bate na esquina da cama. Mais do que entrar ou sair do quarto, a Maria
Rosa esgueira-se para dentro e para fora dele. Por um capricho
arquitectónico, o quarto não tem janela. Sim, porque não me convenço a
chamar ``janela'' àquele quadrilátero de vidro fosco que dá para um
pátio interior a que não chega a luz do dia.

Só muito de vez em quando chegam visitas para a Maria Rosa. Não estou a
contar com o Rafa, que se tornou presença assídua nos nossos serões,
muito a meu pesar. Tal como sucede com os outros aspectos da sua vida, a
relação da Maria Rosa

\textbf{50 }ALEXANDRE ANDRADE

com esta criatura tosca e impertinente é para nós um enigma.

Por exemplo, ainda na noite passada o Rafa estava sentado no canapé.
Mergulhava a mão esquerda na tigela das sementes de girassol e a mão
direita na tigela dos mirtilos revestidos de chocolate preto do Equador
com 78\% de cacau. Enchia a boca às mãos-cheias enquanto devotava toda a
sua atenção à conversa. O Emílio expunha a um amigo do Cristóvão e à
respectiva cara-metade a sua teoria sobre o que atrai e cativa o
observador numa pintura.

\begin{itemize}
\tightlist
\item
  Tem a ver com a oposição figurativo/abstracto e não tem a ver.
  Pensemos em Pollock, Rothko. Há uma qualidade difícil de definir, uma
  ressonância com a época e os gostos mas que é também intemporal.
\item
  Magritte e De Chirico envelhecem mais facilmente do que os abstractos.
\item
  Voltamos sempre ao gosto inato pelas narrativas. Os expressionistas
  americanos, os suprematistas russos, sabiam deixar portas abertas para
  os respectivos universos estéticos, sob a forma de histórias, mitos,
  pouco importava.
\item
  Pollock, o artista possuído pelo seu daimon, espezi- nhando e
  pintalgando a tela com álcool, tinta e suor pela madrugada dentro.
\end{itemize}

Haveria alguma dose de ironia no olhar do Rafa? Im- possível dizer. À
medida que a noite avançava, distraía-se e começava a percorrer a sala.
O seu andar bamboleante era o de quem procura, sem verdadeiramente
acreditar nisso, algo que rompa a couraça do tédio. O seu corpo vasto,
invariavelmente vestido de cabedal preto, circulava entre os pequenos
grupos de pessoas como um satélite maciço, abandonado à mercê das leis
da Física. Saía para a varanda; o luar iluminava a sua longa cabeleira
em desalinho.

QUARTO ESCURO \textbf{51}

Certas noites, ele sai abruptamente depois de acenar num gesto largo que
não é dirigido a alguém em particular, em jeito de despedida.

Por vezes, abre a porta do quarto da Maria Rosa, repete o milagre de
fazer passar o seu gabarito impressionante pela nesga de ar entre a
porta e o lambril, fecha a porta atrás de si.

Segue-se o silêncio, ou um diálogo abafado pela acústica da casa; por
vezes um queixume, só raramente um grito, nem sempre tão angustiado e
dilacerante como daquela vez, naque- la noite que mesmo sem o grito
teria permanecido nas nossas memórias.

Será oportuno confrontar o Rafa, perguntar à Maria Rosa se podemos fazer
alguma coisa por ela, insistir na nossa dis- ponibilidade para ajudar?

O Emílio acha preferível evitar o conflito. Na sua opi- nião (que
partilha connosco enquanto marca com o dedo a página da antologia de
contos do John Cheever que anda a reler), a Maria Rosa seria
perfeitamente capaz de pedir ajuda, se precisasse. Intrometer-se na vida
alheia de forma leviana é uma das coisas mais detestáveis. Quem observa
as coisas do lado de fora não está em condições de julgar. Mais vale
deixar os acontecimentos seguirem o seu curso natural. Todos estes
preceitos, articulados pela voz débil mas rica- mente modulada do
Emílio, irradiam sensatez. Só muito mais tarde, a meio da madrugada (o
amigo de um amigo de alguém trouxe uma garrafa de conhaque sublime, à
qual pres- tamos tributo unânime enquanto Antony \& the Johnsons gemem
na aparelhagem), o Emílio admite que deve dinheiro ao Rafa. Não uma
fortuna, mas --- adivinho --- o suficiente para lhe pesar na
consciência. Não passará o Rafa de um vulgar agiota? Dir-se-ia antes um
faz-tudo, um indivíduo

\textbf{52 }ALEXANDRE ANDRADE

que vive de expedientes, sejam eles emprestar a juros, traficar drogas
leves, consertar persianas ou fazer candonga de bilhetes de concertos.
As histórias sobre o Rafa sucedem-se à medida que a primeira luz do dia
se espalha pelo horizonte urbano, penetra devagar pelas janelas da sala.
Nada disso, claro está, serve de justificação para ele molestar ou
importunar a Maria Rosa. Mas será que o Rafa, de facto, a importuna?
Algum de nós alguma vez se atreverá a meter conversa com ela? A Maria
Rosa levanta-se cedo, dirige-se à cozinha evitando pisar os copos e
livros deixados no chão, toma em silêncio o seu pequeno-almoço que é
rápido e frugal.


\subsection{* * *}

Numa tarde de muito sol, encontrei a Vanessa sozinha na sala, sentada no
chão, a comer um resto de \emph{gratin dauphinois }em garfadas pequenas
e lentas. A sua posição era a de entrega ao prazer gustativo ou aos
sobressaltos da memória, cada um por sua vez ou misturando-se em
cascata. Não queria interrompê-

-la, mas qualquer coisa na luz daquela tarde ou na cacofonia nervosa de
ruídos urbanos predispunha-me para a conversa, além de que não era todos
os dias que surgia uma ocasião para contrariar a relutância da Vanessa
em revelar mais do que uma versão pasteurizada da sua extraordinária
pessoa. Fingi procurar um CD, fingi hesitar entre a \emph{Méditation
pour le Carême }de Charpentier, pelo grupo Les Arts Florissants, com
direcção de William Christie, e as \emph{9 Lamentationes Hieremiae }de
Lassus, pelo grupo La Chapelle Royale, com direcção de Philippe
Herreweghe. Quando a Vanessa falou, fê-lo como se

QUARTO ESCURO \textbf{53}

as minhas delongas não passassem de uma expectativa, cheia de tacto,
relativamente ao que ela teria para dizer.

\begin{itemize}
\tightlist
\item
  Passei quase todo o dia a pensar em exemplos de teo- rias que sejam ao
  mesmo tempo conceptualmente simples, fér- teis e profundas. Há menos
  exemplos do que se julga.
\item
  Não me ocorre nenhum --- respondi, voltando-me para ela, um CD em cada
  mão. Tentei cruzar o meu olhar com o seu, mas em vão.
\item
  Há por exemplo o heliocentrismo, a teoria ondulatória da luz, o
  utilitarismo... Tudo coisas que estão ao alcance do entendimento de
  uma criança perspicaz.
\item
  As teorias da população de Malthus, talvez?
\item
  Sem dúvida. Bem lembrado. Estava aqui a decidir se a noção de
  arbitrariedade do significante poderia entrar nesta categoria.
\item
  Supor que Saussure poderia ser explicado com sucesso à tua ``criança
  perspicaz'' parece-me testar os limites do ra- zoável.
\item
  É menos abstracto do que julgas. Chega-se lá por meio de indução,
  exemplos...
\end{itemize}

Mas eu, de súbito, deixara de lhe prestar atenção. A luz do sol,
entrando na sala num ângulo que só alguns fins de tarde do ano tornavam
possível, iluminava agora a jorros o corredor, as portas abertas dos
nossos quartos e a porta, fechada como sempre, do quarto da Rosa Maria.
(O nome dela, viemos a saber, é Rosa Maria, e não Maria Rosa.)
Distinguia-se, sobressaindo da madeira da porta, algo que me
sobressaltara. A Vanessa não precisou de se voltar para trás para
perceber.

\begin{itemize}
\tightlist
\item
  Está ali desde ontem --- disse ela. --- Ainda não tinhas visto?
\item
  Aquele corredor costuma receber tão pouca luz que já
\end{itemize}

\textbf{54 }ALEXANDRE ANDRADE

me dou por feliz por não esbarrar contra as paredes.

\begin{itemize}
\tightlist
\item
  Vem, vou mostrar-te.
\end{itemize}

Deixou o prato no chão e conduziu-me pela mão, apesar de o percurso até
ao quarto da Rosa Maria ser rectilíneo e mensurável em palmos. Na porta,
à altura dos olhos de um homem adulto, via-se um círculo pintado com
tinta cor-de-

-laranja. A tonalidade era forte, quase a resvalar para o ver- melho.

\begin{itemize}
\tightlist
\item
  Nenhum de nós sabe o que isto quer dizer --- disse a Vanessa. ---
  Tínhamos esperança de que tu soubesses.
\item
  Lamento defraudar-vos. Não faço a mais pequena ideia. O que pode isto
  querer dizer? Um círculo é um círculo.
\item
  Eu acho sinistro. Faz-me lembrar narrativas bíblicas, o anjo
  exterminador, as portas das casas do povo escolhido aspergidas com
  sangue de animal. Um sinal.
\item
  Um sinal não passa de ruído para quem não conhece o código.
\item
  O que irá dizer disto a nossa senhoria? --- perguntou a Vanessa,
  subitamente maliciosa, como perante uma traves- sura à qual se
  reconhece, mau grado a torpeza, um requintado travo cómico.
\end{itemize}

O ronrom da campainha interrompeu as nossas especu- lações. Eu sabia que
os meus convidados só começariam a che- gar daí a uma hora, na melhor
das hipóteses, por isso adivinhei tratar-se dos convidados do Cristóvão.
Todos eles conheciam a casa ou vinham com alguém que conhecia a casa;
deixámos a porta aberta e o caminho livre para entrarem e fluírem para a
sala. Quando acabei de arrumar as bebidas no frigorífico e de dar
destino a uma pilha de livros emprestados que al- guém se lembrara de
devolver, já as vozes e a música tinham tomado conta da sala. A
aparelhagem debitava John Coltrane.

QUARTO ESCURO \textbf{55}

Discutia-se um prémio internacional de arquitectura que tinha sido
atribuído àquele que, entre os potenciais candidatos, menos o merecia.


\subsection{* * *}

A nossa senhoria é um encanto de pessoa. Visita regularmente o
apartamento, não só para se inteirar das infiltrações e avarias nos
electrodomésticos, mas também e sobretudo para conver- sar, ficar a par
do que nos inquieta, das nossas saúdes, dos rumores que correm pela
vizinhança. Costuma demorar-se; na despensa há uma garrafa de Licor
Beirão à qual só falta uma etiqueta com o seu nome. Preocupa-se connosco
como se fosse nossa mãe.

Quando a nossa senhoria fala da Rosa Maria, a sua voz perde firmeza e
desce um tom. Cheguei a suspeitar de que ela sabia mais acerca da Rosa
Maria do que nós, que estaria a par de sei lá que segredo surpreendente
e comprometedor. Estou agora convencido de que aquilo que a inquieta
nesta inquilina tão especial é algo de difuso, um misto de impressões e
de ignorância que pouco deve a certezas ou sequer a conjecturas. Nunca
as vi na mesma sala: é sempre numa terceira pessoa solene e sombria que
a nossa senhoria fala da Rosa Maria.

Quanto ao Rafa, já todos aprendemos a não proferir o seu nome à frente
da nossa senhoria. Causa-lhe terror e asco. Não me surpreenderia se ela
acabasse por lhe vedar a entrada na casa.

Não se pense que o Rafa é um facínora, um monstro reles que pavoneia um
mau carácter dentro destas quatro paredes;

\textbf{56 }ALEXANDRE ANDRADE

não se pense que a Rosa Maria tem o hábito de causar escân- dalo por um
sim e por um não. Perante um terceiro, é difícil descrever a Rosa Maria
sem lhe conferir os contornos de criatura exemplar. A Rosa Maria é um
inquilino-modelo. Não há memória de ela alguma vez ter deixado um pires
sujo no lava-louça.

E porém aquela maneira de andar que se diria um vaguear fantasmático,
com a cabeça ligeiramente inclinada para um dos lados, tão alheada de
tudo como se fosse sonâmbula...

...os gritos nocturnos, os sons de queda de objectos, as conversas
intermináveis cujo rumor abafado nos chega através da porta fechada ou
das paredes...

...as cartas com selos de países distantes, dirigidas à Rosa Maria,
chegadas ao seu destino apesar dos erros no endereço escrito em letras
garrafais, esborratadas pela água ou sabe-se lá que fluidos, ou numa
caligrafia minúscula e intrincada...

Entretanto a noite avançara. Os convidados --- coisa pou- co vista ---
tinham-se juntado em grupos pequenos e bichana- vam argumentos pró e
contra com uma falta de convicção que nem a hora tardia ajudava a
explicar.

Eu estava na sala, sentado no chão, com o meu copo (uma

\emph{lambic }belga de framboesa) pousado junto a mim, a meia-

-distância de duas destas discussões amorfas e tépidas (erros
recorrentes na avaliação do significado histórico do Futurismo italiano,
excelência de uma encenação recente da ópera \emph{Die Frau Ohne
Schatten, }de Strauss). De onde me encontrava, con- seguia ver o
enfiamento do corredor principal do apartamen- to. A certa altura
pareceu-me ver a porta do quarto da Rosa Maria a abrir-se. A dúvida
sobre se se tratava de uma banal ilusão, potenciada pela fadiga e pela
luz escassa, durou poucos momentos. Um, dois, três vultos de pequena
estatura (crian-

QUARTO ESCURO \textbf{57}

ças?) saíram do quarto, devagar e em silêncio. Seguiram-se mais dois,
estes com o tamanho de homens adultos. Estavam todos vestidos de escuro,
um cambiante soturno de cinzento que emergia com dificuldade das trevas
do corredor. A Rosa Maria mal apareceu: não mais do que um perfil e uma
das mãos, através do espaço estreito entre o lambril e a porta, aberta
até onde o permitiam a pequenez do quarto e a mobília. Houve abraços de
despedida que tinham a solenidade de uma família enlutada. A Rosa Maria
ficou a vê-los sair do aparta- mento, um por um.

No dia seguinte, ouvia-se um soluçar contínuo vindo do quarto da Rosa
Maria que durou toda a manhã. Haveria mo- tivos para nos preocuparmos
com o seu bem-estar, a sua saúde? Se os havia, soçobravam em silêncio
perante a segurança e a indiferença do Rafa, que entretanto chegara e se
instalara no seu canto de sofá favorito, indiferente ao debate de uns
amigos de amigos do Cristóvão (Bartók e o regionalismo na música, onde
reside a fronteira entre guardiães de uma tradição e abu- tres
culturais?), estudando criticamente os canapés e os cu- bos de autêntico
queijo manchego espetados em palitos à sua frente. A sua postura, a sua
maneira de se calar sabendo que voltaria a ser o centro das atenções
assim que se resolvesse a tomar a palavra, eram a de quem se sente em
casa.

A presença do Rafa tornava-se cada vez mais frequente. Pouco a pouco,
passara de visita mais ou menos incómoda a indispensável. Quando não
estava a servir de pronto-socorro informático ao Emílio (sistema
operativo, partição do disco rígido, antivírus, bases de dados,
instalação de periféricos, configuração de rede...) exercia as suas
actividades de pequeno corretor (quem diria que a Vanessa apostava ---
nunca sem antes se entregar a prolongadas deliberações com o Rafa --- no

\textbf{58 }ALEXANDRE ANDRADE

campeonato norte-americano de hóquei no gelo?), isto para não falar dos
seus negócios menos lícitos ou menos confes- sáveis, resolvidos em
locais mais recatados, isentos de ouvidos indiscretos. Entre uma e outra
diligência, o Rafa encontrava tempo para socializar na sala. Era aí que
se entregava a um dia- gnóstico benevolente sobre a ocupante do quarto
mais escuro da casa.

\begin{itemize}
\tightlist
\item
  A Rosa Maria está bem e não precisa de aconselha- mento. Sabe
  perfeitamente decidir pela própria cabeça o que é melhor para ela.
  Precisa apenas de estar sozinha durante uns tempos para reflectir
  sobre a sua vida e sobre o que quer fazer com ela.
\item
  Sempre foi dada a humores, mas não se deixa abater.
\end{itemize}

Tem fibra. Verga mas não parte.

\begin{itemize}
\tightlist
\item
  É dona de um senso comum sem falhas. Nunca se dei- xa apanhar sem os
  dois pés bem assentes na terra. Quem me dera poder dizer o mesmo de
  mim próprio! Brutal, este queijo.
\end{itemize}

Quando passo ao Rafa um copo de \emph{Riesling, }aquilo em que reparo é
nas minúsculas partículas de tinta vermelho-ala- ranjado que permanecem
visíveis nas unhas do seu polegar e indicador direitos.


\subsection{* * *}

Alguns dias depois, a Rosa Maria partiu. De um dia para o outro,
deixámos de a ver ou de a ouvir. Os rumores nocturnos que se escapavam
das paredes do seu quarto cessaram por com- pleto. A certeza final só
surgiu quando a senhoria começou a trazer potenciais inquilinos para
visitarem o quarto, homens

QUARTO ESCURO \textbf{59}

e mulheres, jovens ou maduros, que chegavam, entravam no quarto com o
sorriso expectante de quem descobre um espaço que pode vir a ser o seu
durante meses ou anos, saíam com uma expressão inquieta e acossada, o
olhar fugidio, minúscu- las crispações nos dedos e na face.

As cartas com selos de nações exóticas deixaram de aparecer na caixa de
correio.

A Rosa Maria não deixou nada atrás de si: nem um bi- lhete, nem haveres
pessoais. Nada a não ser um iogurte com pedaços, cujo prazo em breve
expirou.

A porta do quarto da Rosa Maria estava aberta. Bastava empurrá-la para
entrar. Nunca algum de nós entrara naquele quarto, continuávamos
relutantes em fazê-lo, como que fiéis a um acordo sem palavras, a um
consenso sobre o que era adequado e o que era escabroso.

Esse acordo sem palavras foi rompido em silêncio, talvez por ignorância,
pela Inge, que numa tarde de muita chuva empurrou a porta do quarto da
Rosa Maria até onde pôde e penetrou no interior.

Fui o único que a viu sair do quarto da Rosa Maria. Te- nho a certeza
absoluta disso. Ela não permanecera no quarto mais do que alguns
minutos. Pensei que se iria dirigir para a sala para contar a
experiência, mas afinal entrou no próprio quarto, sem olhar para trás. O
tempo corria, eu sentia-me incapaz de ler ou ouvir música ou de me
concentrar no que quer que fosse. Combati a tentação de bater à porta do
quarto da Inge. Quando a tentação ameaçava levar a melhor, a Inge saiu
do quarto vestida para sair. Levantei-me do sofá assim que ouvi a porta
do apartamento bater. Saí também, galguei as escadas, avistei a Inge no
cimo da rua. Segui-a com precaução, mantendo entre nós a distância
adequada a um perseguidor

\textbf{60 }ALEXANDRE ANDRADE

que conhece o perseguido. Estranhei a luz do dia, os aguacei- ros
esparsos no meu rosto nu. A Inge entrou num autocarro, corri para o
apanhar. Agora a Inge está sentada de costas para mim. Não me resta mais
do que contorná-la por trás, sentar-

-me no lugar livre à frente dela, olhá-la bem nos olhos. Parti- lhar o
que neles houver para partilhar. Não é um abismo, não passam dos olhos
verdes de uma pessoa.

ABRIL 2011


\subsection{QUEM ANDA A COMER DO MEU
PRATO?}

uem anda a comer do meu prato?

O prato é côncavo, faz lembrar uma malga. Deixo-o lavado e arrumado no
armário durante a noite. Foram já três as vezes em que o encontrei, de
manhã cedo, em cima da mesa de madeira rústica da cozinha com restos de
comida dentro (o meu lanche ou jantar da véspera, subtraídos ao
frigorífico) e uma colher pousada ao lado, como se abandonada por alguém
sem pressa nem reais motivos para permanecer depois da re- feição
clandestina.

O apartamento é pequeno, não mais do que um quarto que serve também de
sala de estar, a cozinha e uma casa de banho minúscula. Recordo-me, com
nitidez insólita, do dia em que o visitei pela primeira vez. Na sua voz
arrastada, a senho- ria parecia exprimir assombro e reticência, como se
arrendar aquele T1 sombrio e silencioso fosse um acto de suprema in-
sanidade. A minha primeira sensação foi a de que bastava abrir os braços
para o abarcar de uma extremidade à outra. A janela do quarto dá para um
largo, um dos mais concorridos da parte antiga de Rueil-Malmaison. De
noite, há três esplanadas que ficam abertas até tarde. As vozes
fundem-se num único sus-

\textbf{62 }ALEXANDRE ANDRADE

surro, que chega ao meu segundo andar mudado em maru- lhar.

Quem será que entra em minha casa durante o meu re- pouso? Quando dei
pela falta do resto de uma talhada de melão, cheguei a acreditar numa
distracção minha. Seguiu-

-se, dias mais tarde, um resto de \emph{quiche lorraine }que de uma
largura de quatro dedos passou para dois dedos; agora foi a tigela com
maçã \emph{granny smith }misturada com leite condensado (não resisto ao
casamento dos sabores doce e ácido) que se esvaziou de um dia para o
outro. Desta vez, o visitante levou a sua desenvoltura ao ponto de
folhear um número antigo do \emph{Nouvel Observateur }esquecido em cima
do frigorífico e até de começar a resolver o problema de palavras
cruzadas. O três vertical (``estado de seca'' = ``siccité'') era
particularmente obscuro.

A relação que chegou ao fim no dia 16 de Maio marcou o encerramento de
uma fase da minha vida. Foi deliberado; eu quis que fosse assim. A
situação (não só a lamentável cena de ruptura em pleno Jardim do
Luxemburgo, mas sobretudo o contexto em que ocorreu) pedia uma clivagem,
uma solução de continuidade nítida. A página foi virada, arrancada,
guar- dada num álbum para ser recordada mais tarde, quando o tempo tiver
cumprido o papel de artesão meticuloso que as más metáforas lhe
atribuem. Sair de Paris e vir viver para os arredores foi apenas uma das
componentes dessa necessidade de mudança, nem sequer a mais urgente. A
minha amiga Manon, que eu recordo luminosa num café do Boulevard des
Batignolles, deixou-me três recomendações para ultrapas- sar este hiato
emocional na minha vida: nunca desdenhar da companhia das pessoas; não
dar crédito a adágios sobre a feli- cidade e o amor; preservar a minha
intimidade sem deixar de

QUEM ANDA A COMER DO MEU PRATO? \textbf{63}

me abrir ao inesperado e ao acidental. (O cabelo dela, pentea- do à
pagem e favorecido pelo sol poente, ganhava o direito a ser comparado a
uma auréola enquanto ela falava.) É decerto o terceiro conselho, o da
intimidade, por mim interpretado à letra, que me leva a deixar a porta
fechada apenas no trinco e escancaradas todas as janelas. Não receio
intrusos nem saltea- dores. Os objectos com valor venal que guardo
comigo fariam fraca figura no balcão de um hipotético prestamista.
Quando alguém bate à minha porta, cumprimento e convido a entrar. As
gentes de Rueil-Malmaison têm fibra moral e gostam de sorrir. Custa-me
acreditar que alguém seja capaz de devassar a minha cozinha e a minha
mesa de madeira rústica.

Para quê esconder dos outros o que se está a passar? Fui contando tudo a
todos aqueles com quem me cruzava, com detalhes e cronologia. As
reacções oscilaram entre a increduli- dade, a preocupação e a
indiferença. Houve quem acreditasse num embuste da minha parte e houve
quem me pedisse a re- ceita da \emph{quiche lorraine}. Não sei se as
minhas atribulações são objecto de debate no mercado local, mas é certo
que poucos serão aqueles, nas redondezas, que ignoram as incursões
nocturnas nos meus trinta e cinco metros quadrados de área útil.

A Manon, que me fez uma visita no sábado passado, sugeriu o estratagema
seguinte: espalhar uma camada fina de farinha de trigo no chão da
cozinha antes de me ir dei- tar. Em caso de nova visita, os traços do
intruso serão visíveis de madrugada. Eu argumentei que o intruso, quer
eu queira quer não, deixa sempre sinais em evidência (comida em falta,
prato e talher sujo, revista folheada) e que não faço questão de ficar a
saber se calça 43 ou 44. A Manon (que trazia uma blusa cinzenta que não
lhe assentava bem nos ombros quase

\textbf{64 }ALEXANDRE ANDRADE

perfeitos) soltou uma gargalhada franca, impossível de simu- lar, e
pousou a mão no meu braço. Achou-me mais magra. Devolveu-me um livro de
que eu já me esquecera.

Uma das minhas colegas do curso de fotografia, a Gwenaëlle, pediu para
subir a minha casa. Triunfante, e com modos de anfitriã com algo para
demonstrar, apontou na di- recção de uma ourivesaria que se avista da
janela do quarto.

\begin{itemize}
\tightlist
\item
  Com um bocadinho de sorte, aquela câmara de vigi- lância apanha a
  janela do teu quarto no seu ângulo de visão. Com mais uma migalhinha
  de sorte, o teu clandestino terá sido filmado em flagrante delito de
  invasão de propriedade privada e sem dúvida numa pose digna da idade
  de ouro do burlesco, tendo em conta o grau de dificuldade da ascensão.
\item
  Em primeiro lugar, tudo me leva a crer que ele entra pela cozinha, que
  dá para uma rua estreita e sem comércio. Em segundo lugar, por que
  vias sinuosas viriam ter às minhas mãos os registos videográficos das
  suas escaladas, admitindo que existem?
\item
  Talvez eu conheça alguém que talvez te possa ajudar. Eram demasiados
  ``talvezes'' para o meu paladar, mas se-
\end{itemize}

ria embaraçoso não esgotar todas as hipóteses para obter uma resposta
para a pergunta que eu trazia nos lábios, tão insis- tente que eu quase
a imaginava suspensa sobre a minha cabeça dentro de um gracioso balão.
Nessa mesma tarde, recebi um SMS com um local, uma data, uma hora e um
nome. O nome era ``Jéremy''.

Se eu ainda hesitava em encontrar-me com ele, o que aconteceu nessa
noite fez-me decidir. Ao regressar de um encontro de amigos em Paris
(Rue des Lombards, clube de jazz, excelente ambiente mas música
sofrível), já com a madrugada a meio, deparei com sinais óbvios de uma
nova

QUEM ANDA A COMER DO MEU PRATO? \textbf{65}

intrusão. Desta vez, não faltava comida. Em vez disso: colcha
desalinhada, uma concavidade no meu colchão com a forma de uma pessoa
adulta deitada em posição supina e ainda (por- menor desconcertante!)
uma mão cheia de penas brancas de ave espalhadas pela cama, pelo chão do
quarto e pela mesa de cabeceira. Pareceram-me penas de cisne. Não tinham
o aspecto definhado de coisas guardadas numa gaveta anos a fio. Algumas
das hastes estavam quebradas, como se as penas tivessem sido arrancadas
com violência.

Concluí desta quarta intrusão que o meu visitante não nutria particular
apego pela lógica ternária que costuma ditar leis nas histórias e
fábulas. Tanto melhor. Os padrões tecidos pela tradição têm algo de
iníquo, de pegajoso. O mundo é feito de contingência. Para distrair a
mente, fiz uma tarte \emph{tatin }com toda a concentração de que fui
capaz. Segui a receita à risca. Não provei imediatamente porque é
preciso deixar arrefecer.

O Jérémy era um homem largo de corpo, rosto plácido e risonho, patilhas
crescidas, cabelo ondulado pelos ombros. Disse-me que acabara de se
sentar à mesa do café e que já pedira uma água com gás. Pedi a mesma
coisa. A minha co- lega pusera-o ao corrente da situação, que parecia
diverti-lo moderadamente. Quis saber o que eu fazia na vida, sem poder
adivinhar que a resposta seria necessariamente longa e dividida em
categorias: a profissão que eu deixara de exercer (assistente numa
galeria de arte em Paris), os biscates como consultora científica numa
enciclopédia sobre insectos, o curso de foto- grafia, o passatempo
(oboé) a que eu me entregava por conta própria, a independência
financeira que só muito de longe em longe requeria a assistência do meu
pai, a gozar a reforma em Tours e com demasiadas dores de cabeça por
culpa do inútil

\textbf{66 }ALEXANDRE ANDRADE

do meu irmão. O Jérémy contou-me que era técnico de som no Canal +. As
suas mãos intrigaram-me: dedos robustos de quem exerce um ofício manual,
mas com uma delicadeza de movimentos (segurar o copo, escrever na
agenda, dobrar um guardanapo de papel) própria de um instrumentista. Sem
dar por isso, pus-me a inventar um passado para aquelas mãos.

O interesse que o Jérémy demonstrou pelo meu caso exce- dia os serviços
mínimos da civilidade. Levou o zelo ao ponto de avançar com hipóteses.
Poderia o intruso ser o meu ex-

-namorado? A ideia era pertinente, mas pouco plausível por duas razões:
1) não fazia o género do Yvan deslocar-se aos subúrbios para me
assediar, e 2) ele detestava a minha \emph{quiche lorraine}. E no
entanto, a imagem do Yvan empoleirado no parapeito da minha janela, o
seu corpo sedentário em equilí- brio enquanto contemplava o meu corpo em
repouso enfiado numa camisa de noite, o seu espírito assaltado por
sabe-se lá que guinadas de arrependimento e nostalgia, surgiram-me com a
nitidez da realidade.

Já em minha casa (o Jérémy aceitou o convite sem pes- tanejar, subiu as
escadas estreitas com o passo demorado de quem abandona o corpo às
contingências da arquitectura sem, no íntimo, se dar por vencido),
servi-lhe uma porção de tarte \emph{tatin}. As suas palavras elogiosas
foram contrariadas por um contorcer de feições que denunciou acidez
excessiva. Eu sabia que deveria ter deixado as maçãs amadurecer mais
quarenta e oito horas, no mínimo. Está tudo nos detalhes: escaldar o
bule, golpear o dente de alho com a parte plana da faca, deitar umas
gotas de vinagre na água de escalfar o ovo. Está tudo nos detalhes.
Ninguém o ignora e contudo quase ninguém age em função dessa certeza.

Mostrei-lhe o algeroz que podia ter servido para facili-

QUEM ANDA A COMER DO MEU PRATO? \textbf{67}

tar a ascensão. Apontei os sítios onde a tinta parece lascada, quem sabe
se arrancada por unhas de dedos em movimento de preensão. O Jérémy fez
que sim com a cabeça.

Mal se falou em câmaras de vigilância. O Jérémy disse que não podia
prometer nada e que me iria ligar em breve.

Numa aula do curso de fotografia, passámos uma hora inteira a estudar
uma única fotografia. A fotografia represen- tava uma multidão a fazer
um piquenique ao sol, naquilo que parecia ser um relvado em declive,
provavelmente um parque de merendas. Penteados e fatos-de-banho (seria a
margem de um rio?) remetiam para os anos 20 ou 30. O ano pode- ria ser o
de 1936: Frente Popular, primeiras férias pagas para milhões de
franceses, etc. No entanto, não se discutiu o en- quadramento histórico
ou sociológico. A instrutora chamou a nossa atenção para a maneira como
a disposição dos pares, das famílias, dos grupos, dos veraneantes
solitários, assim como os olhares dirigidos para os vizinhos mais ou
menos distantes ou mesmo para a câmara e os gestos, casuais e práticos
na aparên- cia mas aqui e ali assumindo uma rigidez próxima da pose,
compunham um quadro de naturalidade estudada, encenada, uma elaborada
negação da própria essência da reportagem fo- tográfica. O trabalho do
fotógrafo inseria-se nas dinâmicas de questionamento do real que
caracterizaram as primeiras décadas do século XX, embora se tratasse de
um exemplo que levou o refinamento formal a extremos raros. Virando-se
para mim, a Gwenaëlle mostrou uma careta de enfado, risonha e exagerada.
Uma hora mais tarde, sentada à mesa da minha cozinha com uma chávena de
chá verde aromatizado com menta, dava largas ao riso e à frustração,
jurava a pés juntos que não voltaria ao curso, enumerava as maneiras de
passar o tempo mais frutuosas do que aquela. Perguntou-me como

\textbf{68 }ALEXANDRE ANDRADE

correra o encontro com o Jérémy. Quando a conversa quebrou, vi-a
percorrer as paredes com o olhar, como se procurasse os sintomas do
espaço devassado. Eu já notara que, à medida que as notícias se
espalhavam, as visitas entravam em minha casa como que receando
tornarem-se cúmplices de intrusão.

Deitada, de noite, à espera do sono, eu passava o tempo a explorar o
rectângulo de mundo delimitado pela janela. O céu nocturno, as estrelas
tão mais brilhantes do que em Paris, a Lua na fase prevista pelo
almanaque, fragmentos de fachadas e telhados.

O Jérémy telefonou-me para me convidar a acompanhá-lo numa visita que
ele iria fazer a um amigo de infância no fim-

-de-semana seguinte. Eu já tinha planos com a Gwenaëlle, mas o Jérémy
disse-me para a trazer. Concordei, sem prestar atenção aos detalhes.
Pontual, o Jérémy veio buscar-nos no sábado de manhã no seu Renault
Clio. Não evitei uma ex- clamação de surpresa quando soube que a casa do
amigo do Jérémy ficava em Fécamp. Fécamp! Adoro a Normandia e nunca vou
lá sem dar por deliciosamente bem empregue o meu tempo, mas não estava
preparada para uma excursão tão longa. Conformei-me, satisfeita com a
companhia e ciente de não ter qualquer compromisso ou obstáculo que
servisse de desculpa para desistir, ainda que o desejasse. Falou-se pou-
co durante o trajecto. Mais ou menos a meio, e após alguns rodeios
exasperantes, o Jérémy pediu indulgência antecipada para o amigo
(chamava-se Stéphane), que estaria a atravessar um período difícil e era
dado a flutuações de humor impre- visíveis. Bizarramente, depois deste
apelo sombrio a conversa ganhou fluidez. A Gwenaëlle e eu falámos sobre
fotografia, o Jérémy contou histórias refrescantes sobre acidentes em
rodagens de séries e telefilmes onde trabalhara. Por culpa

QUEM ANDA A COMER DO MEU PRATO? \textbf{69}

do trânsito, fizemos uma média deplorável: só chegámos a Fécamp ao final
da tarde.

A casa do Stéphane mereceria ser designada por ``man- são'', ou por um
seu sinónimo que sugerisse grandiosidade, e estava inserida num terreno
vasto a que não faltavam estufas, barracões vários, campos de ténis e
jardim ao estilo do Rei-

-Sol. Viam-se já numerosas viaturas estacionadas à toa, con- forme
calhava. Assim que chegámos, apercebemo-nos de um alvoroço que fazia as
pessoas circular numa roda-viva nervosa e anárquica. Ninguém prestou a
mínima atenção ao trio de re- cém-chegados. Dispersámo-nos em busca de
informação, fei- tos ave urbana à cata de migalhas. A luz declinava,
como uma confirmação de que o Verão caminhava para o seu fim. Sem dúvida
apiedado do meu aspecto desamparado, partícula livre num oceano de
propósito, um rapaz aproximou-se de mim. Era magro e mais alto do que
baixo; tinha o cabelo castanho separado a meio, de uma maneira que já
não se usa.

--- O dono desta casa tem sete gatos persas, premiados em exposições. Um
deles está perdido. Ninguém lhe põe a vista em cima desde manhã.
Chamo-me Éric.

Que fazer senão juntar-me à busca? Percorri a proprie- dade de lés a
lés, sozinha ou integrada em pequenos grupos que se formavam e desfaziam
ao sabor do acaso e de novos palpites sobre o destino natural de um
bichano àquelas horas e naquelas circunstâncias. Quando a noite caiu de
vez, surgiram lanternas e até uma tocha improvisada. Corpos roçavam-se
na escuridão, ouviam-se chamamentos ríspidos e frases entre- cortadas.
Alguém alertou para as víboras. Um casal sentara-se debaixo de um
pessegueiro, os beijos confundiam-se com as dentadas na polpa dos
pêssegos.

A certa altura, gritos de triunfo sobrepuseram-se àquela

\textbf{70 }ALEXANDRE ANDRADE

paisagem sonora sobre fundo de treva. Encontrei às apalpa- delas o
caminho na direcção de um círculo que se formara, no centro do qual
estava o Éric. No colo dele, iluminado por meia dúzia de telemóveis,
estava uma bola de pêlo castanho de onde sobressaíam dois olhos
assustados e um focinho achatado. O Éric acariciava-o docemente. À luz
medíocre dos pequenos visores de cristais líquidos, vi que as unhas do
Éric estavam lascadas de fresco; nem roídas nem gastas, mas sim partidas
e falhadas como por uma superfície abrasiva. Lembrei-me, com um pequeno
sobressalto, do meu algeroz, lembrei-me do meu parapeito de madeira
áspera e por pintar. O Éric murmurava qualquer coisa sobre o barracão de
ferramentas onde encon- trara o gato, um lugar ``tão sossegado que
nenhum inimigo do alarido acharia imperdoável a escapadela'', e sobre a
parábola do filho pródigo.

Mais tarde, na mansão, a festa (porque era disso que se tra- tava,
embora eu ignorasse qual era a ocasião que se celebrava) decorria
animada. Comia-se e bebia-se, escutava-se música, conversava-se com o
calor adequado a pessoas de convicções firmes mas largas de espírito.
Com o copo numa mão, um pra- to de plástico com acepipes na outra (fiz
uma nota mental para pedir a receita daquela \emph{tapenade}), eu
procurava com o olhar o Stéphane, anfitrião e felinófilo emérito, que
até aí apenas me fora apontado à distância. Acabei por avistá-lo de uma
varan- da. Ele estava no exterior, encostado a um ângulo de fachada mal
iluminado, na companhia do Jérémy. Julgavam-se sem dúvida fora do
alcance de ouvidos de terceiros. O Jérémy am- parava o Stéphane pelos
ombros. A noite era profunda e opaca.

\begin{itemize}
\tightlist
\item
  Compreendes o que isto significa?
\item
  Isto significa o fim, Stéphane. Mas o que esperavas, afinal? Não era
  por isto que todos esperávamos?
\end{itemize}

QUEM ANDA A COMER DO MEU PRATO? \textbf{71}

\begin{itemize}
\tightlist
\item
  Não é apenas o fim. É o fim de tudo. O fim do fim. Um pouco mais
  tarde, o Éric arrancou-me a um tran-
\end{itemize}

siente turbilhão de brindes e dialéctica e quase me arrastou pelo braço
para outra sala. «Aqui discute-se arte e reclamam a galerista», disse
ele. (Com efeito, em cada divisão falava-se de um tema diferente.) Como
saberia ele que eu tinha traba- lhado numa galeria? Ou teria sido eu a
revelar-lho, sem querer? Sentei-me no centro de um círculo (era mais
fácil sentar-me no centro do que alargar o círculo) dominado por um
homem muito alto e ruivo, que falava com sotaque alemão e voz de
barítono. Visto de longe, ou com a atenção dividida, pareceria um
adolescente, mas os olhos e o discurso traziam atrás o seu lastro de
anos vividos. Expus as minhas objecções sacrílegas sobre Poussin,
exaltei heróis pessoais como Rouault, Daumier, Ensor (este atravessa um
período de desconsideração crítica, ao que consta) e, claro, Gustave
Courbet.

Ainda mais tarde, noutro local e com menos luz:

\begin{itemize}
\tightlist
\item
  O que nos impede de irmos a Londres, amanhã mes- mo, só para ver essa
  natureza-morta de que falaste? Nada nos impede. Podemos não ir, mas
  que seja por uma boa razão e não por termos o livre-arbítrio tolhido.
\end{itemize}

Chamava-se Conrad, era austríaco, pintava, tinha um es- túdio em
Bobigny.

Apanhámos uma boleia e um táxi até ao aeroporto de Roissy, um voo
matinal até Londres. Na National Gallery, a \emph{Natureza-Morta com
Maçãs e Romã, }de Courbet, recompen- sou-nos com as suas tonalidades
vivas, insolentes, honestas. Tão ricas! Tão poderosas!

PORTO, AGOSTO 2011.


\subsection{SUL}

bom senso diz que o exílio deve ser à medida da profun- didade da queda.
Não foi o bom senso, mas sim a urgên-

cia e o acaso que me transportaram para longe de Portugal e do ``cenário
funesto'' do solar do Peso da Régua, através do Atlântico, num trajecto
que foi segmento de recta cuja extre- midade era agora uma taberna de
Buenos Aires onde eu estava sentado, tonto de fadiga, com um copo vazio
de cerveja diante de mim.

Tinha lugar reservado no paquete que iria largar no dia seguinte, rumo a
sul. Não tinha destino final. Perder-me nos confins remotos da Terra do
Fogo era algo que convinha ao meu estado de espírito torturado pelo
remorso e pelo espectro da oportunidade perdida. Tencionava ocupar as
horas que me separavam do anoitecer com bebida e conversas ocasionais,
na taberna que eu escolhera com a mesma confiança na bondade do acaso
com que iria escolher uma pensão para pernoitar. A segunda cerveja
tardava em chegar. O proprietário, um homem de idade de rosto largo e
rosado, ornado por uma barba branca e rala, dedicava a atenção a tudo
menos a satisfazer o meu pedido. Tive tempo para passear os olhos pela
sala acanhada:

\textbf{74 }ALEXANDRE ANDRADE

calendários de futebol, bibelôs, uns quantos troféus tão vetus- tos que
um observador crédulo poderia pensar tratar-se do próprio Santo Graal. O
único cliente que partilhava o espaço comigo era um sujeito de grande
porte que esquecera a sua bebida para se concentrar na leitura de uma
carta. A julgar pela expressão, mais do que as notícias transmitidas
pela carta, devastava-o o poder de que umas quantas linhas garatujadas
dispõem para mudar vidas e espalhar infelicidade.

A luz entrava pelas traseiras da sala através de uma porta escancarada.
Olhando nessa direcção, julguei distinguir vege- tação, o cinzento da
pedra, o vermelho do tijolo e uma nesga de céu. Rodei o tronco,
aproximei a cadeira; percebi que se tratava de um pequeno jardim
interior, do qual, na posição onde me encontrava, não avistava mais do
que uma parcela. Assim que fui capaz de perceber a disposição dos seus
elemen- tos, aquele jardim começou a despertar em mim uma sensação de
familiaridade potente e incómoda. É provável que a minha boca se tenha
entreaberto de estupefacção; não ousei avançar para a porta, preencher o
meu campo de visão com o resto do jardim.

Olhei em redor. O patrão não estava à vista. Dirigi-me ao único ocupante
da sala, o compungido leitor da carta que as aparências sugeriam conter
notícias tão lamentáveis.

--- É a primeira vez na minha vida que venho a este sítio

\begin{itemize}
\tightlist
\item
  comecei, numa voz que me soou bem mais grave do que queria --- , e
  contudo posso garantir-lhe que, mesmo às cegas, seria capaz de me
  orientar naquele jardinzito interior que se vê daqui, de identificar
  detalhes, acidentes do terreno e da arqui- tectura, espécies vegetais,
  tudo isto sem hesitar e sem precisar de usar o sentido da visão.
  Também o poderia descrever até ao mais ínfimo detalhe: o formato
  rectangular, mas com o lado
\end{itemize}

SUL \textbf{75}

sul um tudo-nada mais curto; o chão de terra; o tanque no meio, em forma
de rosácea e coberto de limos; os canteiros nos vértices do rectângulo,
plantados com rosas, gerânios, jacintos e gladíolos; a galeria que corre
ao longo da aresta norte, com bancos embutidos demasiado baixos e
estreitos para um adul- to; a única porta que dá para o edifício
principal, baixa e em ogiva, a meio da galeria. Nunca pisei aquele
jardim mas passei horas, tardes, dir-se-ia toda uma vida num jardim
idêntico, na terra onde cresci, noutra cidade, noutro hemisfério.

Juntar o acto à bravata pareceu-me a coisa mais natural, e aquela que se
impunha aos olhos do meu único ouvinte, por sinal mediocremente
interessado no meu discurso. Improvisei uma venda com uma esteira de
ráfia macia que estava pousada numa das mesas, cobri os olhos, dirigi-me
às apalpadelas para a porta das traseiras.

O sol na nuca e a terra batida debaixo dos meus pés indi- caram-me que
estava no jardim.

Pelo cheiro e pelo toque, fui avançando.

Senti a frescura húmida do mármore do tanque, cheirei as rosas,
sentei-me a custo no banco estreito e baixo.

Esgravatei a terra dos quatro canteiros, um por um, reco- nheci o aroma
das flores, cada espécie associada ao seu ponto colateral.

Recordava-me de que um dos azulejos do friso da gale- ria estava
quebrado --- percorri-os todos com os dedos, um a um, até o encontrar.
Lembrei-me de um dia de chuva, do piano vertical, da inépcia dos homens
que se encarregavam do seu transporte para dentro da casa, de um ângulo
demasiado fechado, do som da madeira contra a faiança.

De acordo com as minhas recordações, estava agora frente à porta.
Avancei, sem hesitar. Precisava de me baixar cerca

\textbf{76 }ALEXANDRE ANDRADE

de cinco centímetros e de ter cuidado com os degraus, que começavam sem
aviso, e com a subida sem corrimão.

Comecei a subir, muito devagar. Senti a queda abrupta na temperatura.
Esquecera-me já de como aquela casa parecia possuir o seu clima próprio:
sufocante no Verão, dada a cor- rentes de ar glaciais no Inverno.

Com as costas da mão, senti o relevo do friso de ladrilhos que
acompanhava a subida, e que eu sabia serem brancos e azuis.

Não contei os degraus, mas soube que chegara ao último sem precisar de
sondar o patamar com a planta do pé. En- contrava-me, pois, no corredor.
Desembaracei-me da minha venda improvisada; os meus olhos acusaram pouca
diferença entre a treva imposta voluntariamente e a penumbra em que
estava imerso aquele corredor, mal servido por janelas altas, pequenas e
pouco numerosas.

À minha direita havia portas de quartos. Fechadas. Não se ouvia um único
rumor na casa. A hora da sesta era cumpri- da com um rigor quase
marcial. Girei a maçaneta da primeira porta, entrei de mansinho. Esperei
até que os meus olhos se habituassem à escuridão. Secretária, cadeira,
roupeiro, peque- na estante, duas camas: uma feita, outra desfeita e
ocupada. Deitei-me na cama vazia. O sono abatia-se sobre mim com o peso
de uma fatalidade histórica, de uma verdade óbvia e co- nhecida de
todos. Adormeci assim que senti na face a textura suave da almofada, no
corpo a firmeza familiar e acolhedora do colchão.

Quanto tempo dormi? Não sei dizer. Não tinha um reló- gio comigo. O sol
não penetrava pelas persianas cerradas. Alguém se debruçava sobre mim,
um rosto cujas feições venciam o escuro fitava-me, julgando-me ainda a
dormir.

SUL \textbf{77}

O medo sucedeu à surpresa nesse rosto; ao medo sucedeu algo semelhante à
resignação perante o facto consumado, ao alívio da habituação. O rosto
era jovem e feminino, enquadrado por cabelo cortado curto atrás,
crescido em cima, caindo em cara- cóis largos. Os olhos eram grandes e
rodeados por uma som- bra perpétua, os lábios assimétricos e
expressivos, as orelhas pequeníssimas e levemente salientes, o queixo
curto mas bem recortado.

Lucrecia. A minha irmã.

``Lucrecia.'' O nome da minha irmã morta subiu-me aos lábios assombrados
num impulso de reconhecimento, de iden- tificação com aquele rosto que a
tensão abandonara. Agora, uso-o como um qualquer estratagema narrativo,
sem escrúpu- los de autenticidade; naquele momento, ``Lucrecia'' foi uma
evidência que se impôs com a força das coisas do mundo. A minha irmã
(portanto) sacudiu-me pelo ombro, com a brus- quidão necessária para
dissipar sonhos e visões capciosas. Ergui-me e abracei-a. Entre
lágrimas, ela também me tratou por irmão. Chamava-me ``Silvio'', que não
é o meu nome, mas abraçava-me como a um irmão perdido há muito ou em
vias de se perder. Evitámos falar, saboreámos este desleixo do tem- po e
da sorte nos braços um do outro. Os soluços de Lucrecia foram-se
espaçando, cessaram.

--- Silvio, contigo agora em casa tudo vai ser diferente.

Explicações sobre este ``tudo'' tiveram de ficar para mais tarde. Após
consultar o despertador que estava sobre a mesa-

-de-cabeceira, Lucrecia levantou-se de um salto e puxou-me pelo pulso
para fora do quarto. Entrámos num outro quarto do outro lado do
corredor. A um canto havia um roupeiro, cujas portas Lucrecia abriu de
par em par. Para minha sur- presa, encontrei dentro dele uma quantidade
extraordinária

\textbf{78 }ALEXANDRE ANDRADE

de roupa exactamente à minha medida: camisas, casacos, calças e meias,
tudo em preto, branco e gradações de cinzento.

--- O jantar começa dentro de 10 minutos --- disse Lucre- cia, saindo.
--- Não te atrases.

Mudei de roupa com a maior rapidez que me foi possível. Não tive
dificuldades em me orientar pelos corredores som- brios da casa.
Descia-se uma escadaria, atravessava-se uma saleta mal mobilada mas
repleta de quadros de antepassados, antes de se chegar à sala de jantar.
Fui o último dos convivas a chegar. Lucrecia tivera tempo de anunciar a
necessidade de mais um prato e um talher. Sorri polidamente, saudei
quase sem mover os lábios, sentei-me.

Ao meu lado direito estava Lucrecia, vestida agora com uma adorável
blusa branca debruada a azul-marinho. Piscou-

-me o olho para me encorajar. Estava bela, estava radiante, enchia a
sala.

Nos topos da mesa, o pai e a mãe de Lucrecia. A presença de um convidado
inesperado não suscitou neles qualquer sur- presa. Fitavam-me de forma
intermitente, com mais cortesia do que interesse.

À minha frente, um homem de idade, rotundo e falador, e uma mulher entre
a juventude e a maturidade, que pouco comeu e nada disse durante a
refeição.

Comeu-se esplendidamente. Havia \emph{consommé, }truta no forno e pudim
de laranja. Estranhei o facto de não se beber nada a não ser água. Por
uma janela aberta, escutava-se o canto ininterrupto de um grilo,
ouviam-se as vozes altas e arrogantes daqueles que escolhem a noite para
sair e mostrar ao mundo que estão vivos, são belos e ardem em vontade de
triunfar.

Recusei o café, despedi-me e subi com Lucrecia. Senti-me seguido por
vários olhares de estima e benevolência.

SUL \textbf{79}

Deitados no quarto dela, cada um na sua cama e agora com todo o tempo à
nossa frente (ou pelo menos o tempo daquela noite, sem luar mas povoada
de gritos e imprecações), entregámo-nos, à falta de explicações, às
recordações comuns, com a sinceridade e o ímpeto de quem receia ou sabe
ser ine- vitável uma separação próxima. As frases que saíam da boca de
Lucrecia eram cadenciadas pela aflição, mas também pelo alívio próprio
de quem encontrou um confidente. O tom neu- tro que imprimia às suas
revelações não disfarçava a saturação de angústia e dúvida.

Lucrecia falou-me dos seus pais e de como o amor profun- do e inegável
que sentiam por ela era o único sentimento que partilhavam. Descreveu-me
as etapas que tinham sido atraves- sadas até esse amor se metamorfosear
em violento sentimento de posse, pretexto e impulso para as manobras
mais torpes. O afecto de Lucrecia era disputado pelos progenitores como
um troféu de guerra.

Descreveu em seguida, com uma candura e uma ausência de paixão que me
deixaram um nó no estômago, o extremo de descaramento que o pai dela se
permitira, ao introduzir a amante no lar familiar. Pamela começara a
frequentar a casa como preceptora de Lucrecia, mas, a pouco e pouco,
esse tí- tulo acabara por se reduzir a um mero e débil pretexto para
confundir as bocas do mundo. Era ela a mulher de poucas palavras que eu
vira à hora de jantar. O seu quarto ficava no canto noroeste da casa.
Deslizava pelos corredores como uma sombra. A maneira como Pamela
idolatrava o pai de Lucrecia tinha algo de doentio.

O outro conviva do jantar era o Dr. McAllister, médico da família há
três gerações e velho amigo do avô paterno de Lucrecia. Eram tantas as
vezes que almoçava, jantava ou per-

\textbf{80 }ALEXANDRE ANDRADE

noitava que todos o viam mais como um hóspede do que como uma visita.
Corriam sobre ele rumores sinistros que Lucrecia nunca se preocupara em
tentar confirmar, ainda que soubesse como fazê-lo. Bastavam-lhe certos
episódios, que ela preferia não recordar, para acreditar no seu íntimo
que a natureza dele estava à altura da reputação.

Lucrecia devia à sua saúde frágil o facto de receber em casa a educação
e de raramente sair. Passara pelas mãos de uma longa cadeia de
preceptores cujo último elo era Pamela. Pamela não dedicava à educação
de Lucrecia mais do que al- gumas horas por mês, e sempre com evidente
enfado. Lucrecia ocupava-se com leituras de autodidacta, dormia muito e
de- senhava paisagens imaginárias, quase sempre imensas, quase sempre
com um único artefacto humano (casa, máquina agrí- cola, bicicleta)
pequeno e periférico.

A casa era arrendada. A família tinha chegado a Buenos Aires depois de
os revezes da sorte a terem compelido a vender a sua fazenda em Rosario.
A intenção tinha sido a de embarcar rapidamente e refazer a vida na
Europa, no Brasil, no Chile, em qualquer um dos lugares remotos onde
existiam familiares ou amigos radicados e promessas, mais ou menos
vagas, de um qualquer posto de trabalho ou oportunidade. O tempo pas-
sara, a casa que não deveria ter sido mais que uma escala rápi- da
tornara-se residência permanente. Os móveis estavam dis- tribuídos pelas
divisões da casa com a falta de cuidado própria de quem se julga apenas
de passagem. Lucrecia garantiu-me que existiam ainda malas por desfazer,
arcas pesadas cheias de roupa, baixela e livros, guardadas no sótão e
esquecidas. As promessas de uma vida nova nunca se tinham concretizado.

Madrugada adentro, Lucrecia começou, primeiro a medo, depois com a
intensidade de quem preza um cúmplice acima

SUL \textbf{81}

de todas as outras coisas da vida, a partilhar comigo as suas suspeitas,
demasiado terríveis para uma pessoa só.

Sara, a mãe de Lucrecia, estava doente havia muito. Ne- nhum médico fora
capaz de diagnosticar com precisão o mal de que padecia e que a ia
debilitando cada vez mais, mês após mês. O Dr. McAllister conseguira,
pelo menos, acertar com o \emph{cocktail }de drogas que, à falta de uma
cura, lhe mitigava as dores e a mantinha lúcida. Era ele quem, todas as
noites, preparava a medicamentação e lha administrava no quarto. Sara
conservava, por enquanto, a força de vontade e a energia suficientes
para se entregar a um simulacro de vida normal: tomava as refeições à
mesa, lia ou cosia na saleta depois do jantar durante alguns minutos,
por vezes ia ao ponto de tocar trechos de ópera na espineta, que era
antiga e precisava de afi- nação. Nunca saía de casa, contudo.

Ninguém em casa ou na vizinhança duvidava de que Pamela era a amante do
pai de Don Paredro, o pai de Lu- crecia. Os esforços para esconder essa
evidência tinham-se transformado numa rotina frouxa, seguida com escasso
zelo. À medida que Pamela fora desleixando as suas obrigações de
preceptora, começara a evitar cruzar o olhar com o de Lu- crecia.
Lucrecia antipatizara com ela desde o início, mesmo quando ainda não
possuía motivos concretos para o fazer.

(Fora nessa primeira noite ou na seguinte que Lucrecia me falara de
Pamela? Escrevo de um lugar e de um tempo onde estas memórias apenas se
deixam aceder à custa de esfor- ços desmedidos. As primeiras noites que
passei naquela casa arrendada grande e fria fundem-se numa só; os dias,
inodoros e sempre iguais, colapsam, vazios de significado.)

A maneira como Lucrecia falava do Dr. McAllister, o tom de voz, a
duração das hesitações, traduziam um medo enraiza-

\textbf{82 }ALEXANDRE ANDRADE

do. As suspeitas que alimentava eram demasiado monstruosas para serem
declinadas em palavras, mesmo perante um irmão. As descrições que fazia
da maneira como o Dr. McAllister preparava os medicamentos, solene e
imperturbável, como um oficiante numa cerimónia, provocavam-me calafrios
que se de- moravam no meu corpo.

\begin{itemize}
\tightlist
\item
  Receio pela mamã quando ela está sozinha com essa criatura.
\end{itemize}

Existiam histórias antigas entre Don Paredro e o Dr. Mc- Allister. O Dr.
McAllister fora resgatado de uma situação ver- gonhosa por Don Paredro,
que arriscara dinheiro e reputação sem qualquer perspectiva de ganho.
Tudo isto se passara mui- tos anos antes do nascimento de Lucrecia.

\begin{itemize}
\tightlist
\item
  Não te deixes enganar pelas aparências: o Dr. McAl- lister parece
  tratar o papá com alguma sobranceria, mas seria capaz de fazer o
  impensável por ele. É o tipo de homem para quem uma dívida de gratidão
  se sobrepõe a tudo. Um cava- lheiro, chamemos-lhe assim.
\end{itemize}

As noites eram consumidas em diálogo e reminiscências. Dormíamos de dia,
de manhã e à hora da sesta. Não tínhamos nada para nos ocupar. A nossa
presença apenas era solicitada à hora das refeições.

Havia sempre um lugar para mim à mesa. A casa e a família acolheram-me
nas suas rotinas. Os meus passos ecoa- vam nos corredores, vazios à
excepção de peças de mobília desirmanadas.

Ao corrente de tudo, do trivial como do inaudito, pergun- tava-me agora
qual deveria ser o meu papel e se deveria agir, e como, e quando.

Muita coisa dependia (pressenti-o, e Lucrecia concordou comigo) de
estabelecer relações com o Dr. McAllister, clara-

SUL \textbf{83}

mente a personalidade mais influente da casa. Ganhei o hábito de me
demorar depois dos jantares em que a gárrula criatura era presença muito
assídua. Descobrimos gostos comuns, far- rapos de mundividência que nos
entretínhamos a partilhar. O doutor mostrou-se encantado por encontrar
em mim um parceiro para o gamão, jogo pelo qual apenas Pamela demons-
trava um mínimo de entusiasmo e nenhuma aptidão. As nos- sas partidas,
que se prolongavam pelo serão adentro, eram invariavelmente
interrompidas às dez e meia. Era essa a hora a que o Dr. McAllister
recolhia ao quarto de Sara para se en- tregar à demorada tarefa de
preparar os medicamentos para a enferma crónica. Nessas alturas, eu
deixava-me ficar enterrado na poltrona, passeando os olhos pelo
vespertino ou escutando a mistura de música e estática proveniente de
uma telefonia que parecia ser peça de museu. Pamela, ociosa, fitava-me
com uma expressão singular entre a curiosidade e o desdém. Don Paredro,
por essa hora, tinha já recolhido aos aposentos, nunca sem um brevíssimo
afago paternal na minha face.

Sara era uma torre de força. O seu sofrimento era real, mas só muito de
vez em quando ela o deixava transparecer pela maneira como o seu corpo
se inteiriçava durante um dos cruéis espasmos que a acometiam.
Tornámo-nos próximos; passávamos tempo juntos, quase sempre em silêncio.
Desfru- távamos da frescura do jardim. Partilhávamos o gosto, tão inex-
plicável como uma paixão, pelo ruído da água a correr.

Definitivamente livre de obrigações e horários, agora que Pamela
dispensava a fachada de preceptora para eternizar a sua presença na
casa, Lucrecia dispunha do seu tempo como bem entendia. Aprendia fagote
sozinha, seguindo um método inventado nos confins do sertão brasileiro
por um fazendei- ro excêntrico; devorava opúsculos de filosofia e
consolação

\textbf{84 }ALEXANDRE ANDRADE

moral que encontrava na biblioteca herdada do avô materno; demorava-se
na cozinha, entregava-se a experiências culinárias esdrúxulas que
redundavam quase invariavelmente em fa- lhanços épicos.

Chegou a estação das chuvas. Nos corredores da casa, nas divisões
vazias, abundavam os recantos apropriados a conver- sas a dois,
conspirações, contagem de armas.

Seria apenas impressão minha, ou Don Paredro mostrava-

-se mais taciturno e irritável a cada dia que passava?

Num domingo de chuva batida e copiosa, surpreendi uma conversa entre Don
Paredro e Pamela. Não me senti culpado de indiscrição. Eu estava sentado
num banco do jardim, delei- tado com o cheiro a terra molhada
proveniente do canteiro dos gladíolos. Ao meu lado, jazia pousado um
exemplar de \emph{Journal of the Plague Year, }de Daniel Defoe. O meu
estado de espírito era desajustado à leitura. As vozes chegaram até mim
através de uma janela mal fechada: irascível a de Pamela, con- tida
(sarcástica?) a de Don Paredro.

\begin{itemize}
\tightlist
\item
  ...entre estas malditas quatro paredes a tua família.
\end{itemize}

Sufoco!!!

\begin{itemize}
\tightlist
\item
  Não fui eu quem...
\end{itemize}

--- ...

\begin{itemize}
\tightlist
\item
  O tempo joga a nosso favor. O Dr. McAllister...
\item
  ...capaz de esbanjar os melhores anos da minha vida, estás tão
  enganado!
\item
  Só te peço que...
\item
  ...respeito. Apenas isso! Respeito.
\item
  Sabes bem que...
\end{itemize}

(As frases mutiladas ganhavam um sentido que pouco de- via à sintaxe ou
à semântica; desfaziam-se em música; harmo- nizavam-se com a chuva.)

SUL \textbf{85}

O temperamento volátil de Pamela parecia capaz de pre- cipitar a
situação. Pela calada da noite, Lucrecia e eu arqui- tectámos um plano,
em sussurros ainda mais suaves do que o costume. Lucrecia executaria uma
manobra de diversão enquanto eu revistaria os aposentos de Pamela.
Aquilo que existia entre Lucrecia e a sua ex-preceptora era uma mistura
de despeito e rancor, com um vago resíduo do respeito mú- tuo que tinham
chegado a nutrir uma pela outra. Foi a esse vestígio remoto que Lucrecia
apelou quando se sentou ao lado de Pamela e lhe pediu ajuda para
classificar alguns dos seus apontamentos escolares antigos, actividade
que deveria con- sumir pelo menos um par de horas das respectivas
atenções. Eu dispunha assim de tempo de sobra para penetrar às escon-
didas no quarto de Pamela e entregar-me a uma revista minu- ciosa. Não
era a altura de ser dominado pelos escrúpulos. Os fins justificam os
meios. Quem sabe se me cairia nos braços um elemento decisivo, o
fragmento de informação necessário para compreender a dinâmica dos
processos e intrigas dentro daquela casa? Tudo correu conforme o
previsto. Tive toda a liberdade para esquadrinhar o quarto de Pamela com
o vagar de quem não se pode dar ao luxo de ignorar a reentrância mais
inacessível, o fundo de gaveta mais escondido por qui- los de
bricabraque. Encontrei jóias e fotografias, bilhetes de teatro e
panfletos políticos. Encontrei até um diário, redigido numa letra
arredondada de garota de escola, página após pá- gina preenchida com
aforismos, desabafos, notas sobre horti- cultura, apontamentos de
leitura. Nem uma referência a Don Paredro, a Sara, a Lucrecia, ao Dr.
McAllister ou a mim.

Porém a densidade de certos silêncios, alguns olhares evi-

tados, toda uma pletora de sinais, diziam-nos (a mim e a Lu- crecia) que
uma catástrofe estava para acontecer.

\textbf{86 }ALEXANDRE ANDRADE

Eu convencera Sara a desconfiar do Dr. McAllister e a evi- tar ingerir
fosse o que fosse da sua farmacopeia. Desde então ela vicejava, vencia a
maleita, começava a reconquistar o seu lugar na casa. Seria impossível
pedir uma confirmação mais decisiva do que esta para as nossas
conjecturas sobre as mano- bras perversas do Dr. McAllister.

\begin{itemize}
\tightlist
\item
  Sabes quem era aquele cavalheiro com uma cicatriz na testa com quem
  nos cruzámos no corredor, ao lado do papá? Era um meirinho. Devemos
  uma soma colossal, e há também os juros. O papá conseguiu renegociar
  algumas parcelas, mas o banco é implacável. Vão penhorar a nossa
  mobília, as nossas pratas e os nossos quadros.
\end{itemize}

Como se houvesse muita mobília para penhorar...

Ao jantar, no final de mais um de tantos dias sem história, Lucrecia
voltou-se para mim e segredou-me: «É hoje! Não per- cas o Doutor de
vista. Desafia-o para o gamão, conta-lhe a tua vida, o que quiseres, mas
não o largues.»

\begin{itemize}
\tightlist
\item
  Mau... Que segredinhos são esses? --- perguntou Sara, que nos fitava
  com um sorriso mal disfarçado nos lábios. Lu- crecia retribuiu o
  sorriso e fingiu um ar compungido. Os seus olhos brilhavam de alegria
  pela recuperação da mãe, mas eu sabia-a apreensiva. Quando Sara saiu
  da sala para recolher aos seus aposentos, Lucrecia seguiu-a com um
  olhar que exprimia o seu receio de que alguém atentasse contra a vida
  dela nessa mesma noite.
\end{itemize}

No salão, o nervosismo do Dr. McAllister transparecia. Ganhar-lhe
partida após partida era uma brincadeira de crian- ça. Ao contrário do
que se tornara habitual, Pamela retirara-se e Don Paredro deixara-se
ficar. Enterrado numa das poltronas que a fúria do fisco ainda não
levara, fitava sem cessar o Dr. McAllister, como se lhe quisesse
transmitir uma mensagem

SUL \textbf{87}

muda. Os seus lábios, já de si delgados, desapareciam agora de tão
comprimidos. Quase senti pena do Dr. McAllister. Ele jogava
mecanicamente, sem pensar; a transpiração descia-lhe pelas faces
flácidas em gotas grossas. Respondia às minhas tentativas de estabelecer
conversação com balbuceios, inter- jeições, sentenças desgarradas.

Numa altura em que o serão já ia longo e se tornara claro que eu não
seria o primeiro dos três a recolher, o Dr. McAl- lister serviu-se de um
pretexto pueril para tentar ausentar-se. Claro está que não o deixei
sozinho. Com uma desculpa ainda mais fantasista do que a dele, segui-lhe
na peugada. Não me dei ao trabalho de olhar para Don Paredro. O Dr.
McAllister avançava pelos corredores com o passo pesado de um conde-
nado à morte a caminho do cadafalso. Mantive uma distância de alguns
metros. Quando ele abrandava, fosse por hesitação ou por estratégia, a
mancha escura das suas costas, encimada pela cabeça levemente tombada
para a frente, crescia no meu campo de visão como tinta derramada a
alastrar numa folha de papel. Deteve-se à altura da porta fechada do
quarto de Sara, chegou até a esboçar o quarto de volta necessário para
se colocar em frente à porta, como quem pretende bater cerimo-
niosamente, ou rodar a maçaneta e entrar sem se fazer anun- ciar. Porém,
não fez esse gesto; limitava-se a aguardar. Apro- ximei-me delicadamente
e puxei com brandura pela manga do seu fato de flanela gasta. Pareceu-me
ver um aceno de cabeça que exprimia aquiescência e alívio. Ele
acompanhou-me sem resistência até ao seu quarto, entrou, sentou-se no
bordo da cama sem acender a luz. Não precisou de palavras nem de ges-
tos para me dar a entender que ia ficar bem, que não precisava mais de
mim. O seu olhar cansado continha talvez uma parte de gratidão.

\textbf{88 }ALEXANDRE ANDRADE

No dia seguinte, Pamela fez as malas e partiu para sem- pre. As
instruções ríspidas que gritava aos ouvidos dos car- regadores ecoavam
pelos corredores e penetravam nas divisões mais remotas. Mal o ruído do
motor do táxi foi engolido pelo mar de sons da cidade de Buenos Aires,
instalou-se na casa uma paz duradoura. Nos dias que se seguiram, todos
nós adoptámos, como que por acordo tácito, hábitos recatados e sisudos.
As conversas à mesa e no salão eram curtas, inócuas e em voz muito
baixa. Os gestos e a locomoção faziam-se com uma discrição de mosteiro.
Até Lucrecia se abstinha de tocar o fagote.

Semana após semana, eu observava traços minúsculos de uma intimidade
conjugal reencontrada entre Sara e Don Pare- dro. Quando o tempo o
permitia, desciam ao jardim e davam algumas voltas de braço dado,
detendo-se para cheirar as flores ou para evocar uma recordação
distante.

O Dr. McAllister deixou-se ficar, mas tudo no seu corpo, no seu rosto
envelhecido, nas suas raras palavras, falava de confusão e de vergonha.

O furor dos credores parecia ter-se atenuado. Vivia-se uma existência
doméstica equilibrada, sem luxos nem carências.

Cada dia assemelhava-se ao anterior.

«Tens de partir!» --- tanto me custou dizer isto a Lucrecia, no final de
mais uma noite de conversas à deriva, reminiscên- cias arrancadas a um
passado partilhado ou construídas de raiz. A simples ideia de me privar
da presença dela, de perder uma irmã pela segunda vez, doía-me mais do
que arrancar um membro, mas odiar-me-ia para sempre se não a fizesse
olhar a verdade de frente. A mãe dela estava agora saudável e a salvo de
maquinações e conjuras. Lucrecia precisava de pensar no seu futuro.
Naquela casa vazia, acabaria por definhar, sem

SUL \textbf{89}

mestre nem propósito. Tudo era preferível a ficar. Convenci-a a falar
com os pais, mostrei-lhe brochuras de uma escola de música muito
reputada que aceitava alunos internos. Lucrecia simulou indiferença,
mesmo contrariedade, mas eu sabia estar a tocar nas suas aspirações mais
profundas.

No dia seguinte, ao regressar da conversa com os pais (longa e isenta de
exaltações, no salão, à porta fechada), Lu- crecia deixou que o brilho e
as lágrimas que trazia nos olhos substituíssem as palavras. Os seus dias
naquela casa vazia e en- torpecedora estavam contados. Iria partir daí a
semanas para uma nova vida na escola de música.

Confesso que, mal soube da notícia, a minha preocupação principal passou
a ser a de preparar a minha saída de cena. Reuni discretamente os meus
pertences, os escassos objectos que acumulara desde a minha chegada.
Cabiam num saco de cabedal de trazer a tiracolo.

Escolhi a altura mais calma do dia, aquela em que as ro- tinas dos
diferentes ocupantes coincidiam num ponto morto. Queria, acima de tudo,
evitar despedir-me de Lucrecia. Fugir pela calada das ocasiões de
sofrimento sempre foi a minha forma muito própria de cobardia.
Escapei-me sem rumor daquele lar pacificado. Desci a escadaria que
conduzia ao jar- dim, estranhei a intensidade da luz exterior, rocei as
flores dos canteiros com os dedos pela última vez, mergulhei a mão na
água do tanque pela última vez, julguei escutar notas de músi- ca
através da janela do quarto de Lucrecia, preocupei-me com a maneira como
estas impressões derradeiras perdurariam na minha memória.

Transpus a porta, no lado oposto do jardim, que comuni- cava com a
taberna.

Não me importo de admitir que cheguei a esperar que o

\textbf{90 }ALEXANDRE ANDRADE

curso do tempo se tivesse suspendido --- esperei-o não como se o
merecesse, mas antes como um capricho da natureza, tão arbitrário do
ponto de vista humano como uma vaga de calor ou uma epidemia. Um relance
rápido sobre a taberna desen- ganou-me. Não havia sinais do meu copo de
cerveja em cima da mesa, o cliente solitário (arruivado, franzino) era
outro, os sinais da passagem do tempo eram múltiplos.

A luz era diferente da que eu recordava. A luz do sol in- cidia segundo
uma direcção sem dúvida consentânea com o calendário.

O patrão entrou e olhou-me dos pés à cabeça, intrigado com a minha
aparência, com o meu saco de cabedal, com o facto de eu ter entrado na
sala por aquela porta, com a água que pingava da minha mão direita,
talvez ainda suja dos limos do tanque.

Perguntou-me o que queria. Pedi uma cerveja bem fresca. Saboreei a
cerveja devagar. Havia tempo de sobra para procurar uma pensão nem
demasiado dispendiosa nem dema-

siado miserável onde pernoitar.

O cliente franzino olhava para mim de vez em quando, sem interesse, sem
vontade de criticar ou tecer conjecturas.

Do lugar de onde estou a (inutilmente, compulsivamente) verter em
palavras o que me sucedeu, um lugar de latitude meridional extrema que
prefiro não nomear, isolado e à me- dida da calamidade indizível que me
trouxe até aqui, recordo a última noite que passei em Buenos Aires, o
quarto exíguo e imaculado, a janela de onde se avistava uma rua mal
ilumi- nada, uma praça de táxis, algumas prostitutas.

Eu respirava o tempo como um gás intoxicante, estava sentado numa
cadeira de ferro e inspirava o tempo, expirava-o com medo do seu poder.

SUL \textbf{91}

O tempo. Agora limito-me a estar atento aos seus abcessos contaminados
pela memória e pelo remorso, de pé sobre esta terra que piso sem deixar
marca.

SETEMBRO 2011

\subsection{VAGA DE FUNDO}

ARRENDA-SE QUARTO

\emph{A menina estudante}

\emph{Ideal para a Universidade do Minho A um minuto a pé do campus de
Gualtar}

\emph{Casa de banho privativa Serventia sala e cozinha}

\emph{TV Cabo e Internet de banda larga 200 euros por mês incluindo gás,
luz e água}

\emph{Telefone 9* *** ** ** Disponível de imediato}

Ágata releu o anúncio colado com fita-cola à vitrine da confeitaria onde
entrara para tomar um café pingado, logo após se ter apeado da camioneta
que a deixara em plena cidade de Braga. Havia outros anúncios, mas esses
não cativavam a atenção, deixavam-se ignorar sem dar uma amostra de
luta. Os outros anúncios pareciam gastos pela luz do sol de Verão que
incidia na fachada envidraçada da confeitaria. Aquele anúncio tinha um
ar novo e enxuto como uma promessa.

\textbf{94 }ALEXANDRE ANDRADE

Ágata ligou para o número do anúncio no seu telemóvel.

O quarto era limpo e espaçoso, a mobília era funcional e confortável. O
outro quarto da casa estava ocupado por uma rapariga chamada Eva. Eva
garantiu-lhe que o prédio era tran- quilo e que o caminho para a
faculdade se fazia a pé sem pro- blemas, de Verão como de Inverno.

\begin{itemize}
\tightlist
\item
  Tenho a senhoria ao telefone --- disse Eva. --- Digo-lhe que ficas com
  o quarto?
\item
  Diz-lhe que fico com o quarto --- disse Ágata.
\item
  Ela fica com o quarto.
\end{itemize}

Ágata pendurou a roupa nos cabides de madeira, sólidos e macios ao
toque, que a solicitude alheia deixara no roupeiro do seu quarto.

\begin{itemize}
\tightlist
\item
  O que vieste estudar a Braga, Ágata?
\item
  Vim estudar Bioquímica.
\item
  Isso é interessante. Eu sou finalista de Psicologia. Olha, hoje à
  noite vou a uma festa com uns amigos numa discote- ca. Vai servir para
  inaugurar o ano lectivo. Vai ser divertido. Queres vir connosco?
\item
  Não sei... Acabei de chegar de X... --- Ágata pronun- ciou o nome da
  pequena vila onde crescera e aprendera algu- mas das árduas verdades
  da existência.
\item
  Teres acabado de chegar de X... não é desculpa. Vamos sair às 10
  horas. Se quiseres tomar banho é só abrires a torneira da água quente
  e a água sai logo quente.
\end{itemize}

A festa já estava animada quando Eva e Ágata chegaram. Conhecidos de
anos passados reencontravam-se e saudavam-se com entusiasmo. Os cinco
euros de ingresso davam direito a uma bebida branca. Eva quase não
largou Ágata e fez questão de a apresentar a todos os seus amigos e
conhecidos. Uma ra- pariga nova nunca deixa de despertar as atenções,
esta é uma

VAGA DE FUNDO \textbf{95}

lei universal cujas excepções se distribuem muito esparsamente pelo
curso da história.

\begin{itemize}
\tightlist
\item
  Esta é a Ágata, acabada de chegar de X... Tenham cui- dado com ela
  porque ela ainda vai chegar longe.
\end{itemize}

Um beijo em cada face, um nome abafado em parte ou no todo pela música
estridente, quase sempre esquecido minutos depois.

De regresso a Gualtar e ao apartamento, exaustas mas felizes, Eva e
Ágata deixaram-se cair no sofá. Ágata queria acabar de desfazer a mala
antes de se ir deitar, mas não resistiu a fazer uma pergunta a Eva.

\begin{itemize}
\tightlist
\item
  Eva, reparei que algumas pessoas na festa passaram a noite toda a
  conversar num canto. Durante uma pausa na música, consegui perceber
  algumas frases e palavras isoladas: ``decreto'', ``despesas já
  cabimentadas'', ``prazos de execução orçamental'', ``nomeação'',
  ``riscos de desagregação social''. As expressões e o tom eram as de
  quem exerce o poder, por mais informais que fossem a postura e o
  cenário. Não me apre- sentaram a nenhum deles. Todos os outros se
  mantinham à distância deste grupo. Quem são eles?
\item
  Queres mesmo falar sobre isso agora? Já é tarde.
\item
  Não estou com sono.
\item
  Então está bem. Por onde começar? Desde a noite dos tempos que os
  homens e as mulheres sentiram a necessidade de se organizar em
  sociedades, suportadas por estruturas de poder que salvaguardassem o
  bem comum.
\item
  Ena, isso vem de longe.
\item
  Há muitos anos que os estudantes universitários em Braga, Guimarães e
  povoações limítrofes se habituaram a gerir eles próprios os seus
  assuntos, com um mínimo de interferên- cia exterior. Estamos a falar
  de centenas de pessoas residentes
\end{itemize}

\textbf{96 }ALEXANDRE ANDRADE

em quartos ou casas particulares, ou em alojamentos disponi- bilizados
pelos Serviços de Acção Social. Não há limites geo- gráficos nem
critérios de admissão. Basta alguém manifestar vontade de aderir e
automaticamente passa a ser um dos nos- sos.

--- ``Um dos nossos''? Trata-se de uma espécie de organi- zação?

Eva explicou tudo.

Havia um poder central que zelava por todos os aspectos da vida, da
saúde ao lazer, dos transportes à justiça. Estudantes de Medicina
ofereciam consultas a preços simbólicos, os de Direito prestavam
aconselhamento jurídico. Quem tinha car- ro próprio ou outro meio de
transporte punha-o ao serviço de uma rede de transportes informal,
organizada em rede. Refeições económicas eram preparadas e servidas,
concertos e sessões de cinema eram organizados. Havia, bem entendido,
ocasiões em que se tornava necessário recorrer à lei ordinária, aos
hospitais, às cantinas, mas o esforço de todos ia no sentido de reduzir
ao mínimo essa necessidade.

A vida em comum era regida por leis e regulamentos. O poder legislativo
era exercido por uma assembleia eleita, por sufrágio universal, de três
em três anos. Vigorava um sistema de duas câmaras. A câmara baixa reunia
semanalmente (ex- cepto em época de exames) num restaurante que era
famoso por dois pratos: a lasanha de bacalhau e o cabrito assado. O
reconhecimento da excelência desses dois pratos era tão unânime como a
convicção sobre a banalidade de todos os demais. Era comum reservarem
uma sala privada para as sessões da assembleia. A câmara alta, elegida
por meio de um complexo sistema indirecto, reunia num restaurante
diferente e só de mês a mês. A câmara alta aprovava, rejeitava ou reen-

VAGA DE FUNDO \textbf{97}

viava para reformulação as leis que emanavam da câmara baixa. Tinha fama
de conservadorismo e amor pela ortodoxia.

\begin{itemize}
\tightlist
\item
  Isso é muito interessante --- disse Ágata. --- E a quem cabe a
  execução dessas leis? Se existem leis, tem de existir al- guém que as
  ponha em prática.
\item
  É aqui que entra o pessoal que viste hoje na festa.
\item
  São eles?
\item
  São eles. Costumam encontrar-se em bares e discote- cas. São nomeados
  pela assembleia. São eles quem nos gover- na, todas as decisões passam
  por eles. Há quem tenha medo deles, não sei porquê. Até são
  acessíveis, e há lá um ou dois rapazes giros. Por vezes podem
  responder com maus modos e isso, mas é só porque o peso da
  responsabilidade é tão grande.
\item
  E quem julga e pune aqueles que violam as leis?
\item
  Para isso existe um sistema muito bem organizado de advogados,
  procuradores e juízes. São os estudantes de Direito que asseguram
  essas funções. O trabalho é muito, coitados, e o curso deles é duro,
  por isso não admira que haja tantos pro- cessos em atraso. Eles
  recebem uma remuneração, mas é uma remuneração irrisória. Mal chega
  para as deslocações. Conto-
\end{itemize}

-te o resto amanhã, está bem? Daqui a bocado adormeço assim mesmo,
sentada e toda vestida.

Ágata acabou de desfazer a mala. Os seus poucos objec- tos pessoais
cabiam na gaveta da mesa de cabeceira e ainda sobrava muito espaço.
Antes de se deitar, Ágata deu algumas voltas pelo quarto, meio a andar,
meio a improvisar passos de dança, e gozou, como quem sorve, o
sentimento novo de estar entregue a si mesma, sem ter de prestar contas
a ninguém sobre os seus movimentos ou sobre a maneira como ocupava o
tempo. As paredes não eram barreiras, mas antes um cenário para a
representação do princípio do mundo.

\textbf{98 }ALEXANDRE ANDRADE

No dia seguinte, bem cedo, Ágata dirigiu-se à secretaria da Universidade
e inscreveu-se nas cadeiras do primeiro semestre da licenciatura em
Bioquímica: Análise Matemática, Álgebra Linear e Geometria Analítica,
Fundamentos de Química, Bio- logia Celular e Laboratórios de Química I.

Os primeiros dias de aulas correram bem. Ágata ficou satis- feita com os
horários que lhe couberam em sorte. As primei- ras impressões dos
professores foram positivas. Ágata comprou uma bata em segunda mão a um
colega que tinha mudado de curso. A bata estava como nova.

De noite, ia a festas, estudava ou ficava em casa a con- versar com Eva
e a ouvir música. Falava com os pais e com o irmão por Skype várias
vezes por semana.

Os eventos mais grandiosos podem ser desencadeados por um detalhe fútil,
mas na maior parte dos casos um capricho não passa de um capricho.
Ágata, cujo apetite pela extrava- gância era, regra geral, modesto, não
resistiu a comprar um unicórnio de gesso, pintado de cores garridas, que
viu na mon- tra de uma loja de bricabraque numa rua tranquila do centro
da cidade. Era do tamanho de um cão da raça husky e poderia ter sido
atracção de feira para crianças numa existência ante- rior. O chifre era
longo e polícromo. Ágata trouxe-o para casa e colocou-o à entrada, por
trás da porta. Quem entrava pre- cisava de se desviar para evitar a
estocada dolorosa no abdó- men. Eva achou graça à ideia da amiga.

\begin{itemize}
\tightlist
\item
  Vai passar a ser a nossa mascote. Já estou a pensar em piadas sobre o
  animal, e também na maneira de responder às piadas que as visitas
  fizerem.
\end{itemize}

Passados alguns dias, um rapaz que Ágata conhecera fu- gazmente num bar,
e cujo nome lhe escapava, veio sentar-se à sua frente na cantina.
Hesitou antes de entrar no assunto.

VAGA DE FUNDO \textbf{99}

\begin{itemize}
\tightlist
\item
  Aquele bicharoco que vocês têm à entrada da vossa casa...
\item
  O unicórnio? Eu chamo-lhe Augusto. É giro, não é?
\item
  É grande de mais. As suas dimensões ultrapassam os limites previstos
  nos regulamentos de segurança. Têm de o tirar dali.
\item
  Tirá-lo dali? Nem pensar nisso.
\item
  É uma questão de segurança. Se houver um incên- dio, se for preciso
  evacuar a casa de repente, podem tropeçar, podem cair, alguém pode
  sair magoado, estás a ver? Podem transferi-lo para a sala ou para um
  quarto, mas na entrada não pode ficar.
\item
  Que regulamentos são esses? --- perguntou Ágata a Eva, indignada,
  assim que chegou a casa.
\item
  Acho que tenho uma cópia ao lado da minha cama, no meio daquela
  confusão. Se quiseres remexer estás à vontade.
\end{itemize}

Ágata encontrou o regulamento, entre uma revista de Be- las-Artes
(``Edgar Degas classique et moderne: la rétrospective au Grand Palais'')
e um catálogo de vendas ao domicílio.

\begin{itemize}
\tightlist
\item
  Amiga, este regulamento não tem pés nem cabeça.
\item
  Os desígnios da lei são incompreensíveis. É verdade nas parábolas e é
  verdade na vida real.
\item
  O que não é absurdo é impraticável, e o que não é impraticável está
  obsoleto.
\item
  Sim, diz-se por aí que muitos desses artigos e alíneas foram redigidos
  na recta final de noites bem comidas e melhor regadas. Sabes como é.
  Entre uma lasanha de bacalhau e um cabrito assado, o vinho tinto da
  casa vai escorregando com ligeireza. Depois vêm a baba de camelo, o
  café e o digestivo. Como é que a inteligibilidade das leis não haveria
  de se res- sentir?
\end{itemize}

\textbf{100 }ALEXANDRE ANDRADE

\begin{itemize}
\tightlist
\item
  A câmara alta não teria obrigação de exercer o seu pa- pel
  fiscalizador?
\item
  Esses são ainda piores.
\end{itemize}

Não restavam dúvidas de que o unicórnio tinha de mudar de lugar.
Trataram disso nessa mesma noite. Ágata arranjou um canto livre no seu
quarto. Consolou-se com a ideia de que se iria servir do chifre para
pendurar a bolsa ou peças de roupa sortidas.

Nos corredores da faculdade, entre duas aulas, nas saí- das nocturnas,
começou a perguntar aos colegas e conheci- dos o que achavam do sistema
vigente, se tinham queixas, se não achavam que a classe dirigente
revelava alguns tiques de prepotência, etc., etc. Entre indiferentes e
situacionistas, lá foi encontrando alguns descontentes, alguns desabafos
soprados entre dentes, algumas histórias inquietantes.

\begin{itemize}
\tightlist
\item
  O que se faz quando uma lei é injusta, Eva? Devemos seguir o que nos
  diz o sentido de justiça, ou o que manda a lei?
\item
  Ainda estás a pensar no unicórnio?
\item
  Já deixou de ter a ver com o unicórnio. Agora gosto menos dele. Cheira
  a bolor, e as cores são vulgares sem serem graciosamente
  \emph{kitsch}. Estou a falar de outras coisas.
\item
  Esta assembleia foi eleita há um ano. Faltam dois para cumprir o
  mandato.
\item
  Não se pode esperar tanto tempo. Não vês o que se passa à nossa volta?
  Os serviços de apoio médico pioram de dia para dia, o sistema de
  transporte está um caos, o banco de apontamentos \emph{online }está
  inacessível há semanas. E que dizer da transparência das contas? Os
  regulamentos dizem que as contas actualizadas têm de estar sempre
  disponíveis, mas eu ontem fui ao \emph{site }e as últimas que
  encontrei datam de 2008 e estão cheias de erros. O que acontece ao
  nosso dinheiro? De
\end{itemize}

VAGA DE FUNDO \textbf{101}

que serve pagarmos a nossa quotização mensal?

\begin{itemize}
\tightlist
\item
  As festas, por exemplo, custam pequenas fortunas. É preciso pagar ao
  DJ, pagar as bebidas, pagar o aluguer do es- paço. As entradas não
  cobrem o custo, longe disso. E as festas aqui são muitas e são boas,
  tens de concordar.
\item
  As festas são boas, mas não vivemos só de festas. Al- guém tem de
  fazer alguma coisa. Isto não pode continuar.
\item
  Manifesta-te, denuncia, diz de tua justiça. Liga para a rádio
  universitária e põe a boca no trombone.
\item
  A rádio só passa música, por sinal péssima.
\item
  Eles têm um fórum da meia-noite à uma. Qualquer pessoa pode
  participar. É só ligar e falar.
\item
  De certeza que eles filtram as chamadas.
\item
  Não filtram nada. Liga para lá, amiga.
\end{itemize}

Ágata telefonou para o programa de rádio. O rapaz que atendeu tentou
dissuadi-la quando soube o motivo.

\begin{itemize}
\tightlist
\item
  Isso é má onda. Está fora do âmbito daquilo que costu- mamos discutir
  aqui. Tentamos não misturar a vida académi- ca com a política.
\item
  A política é tudo.
\item
  Não queres mudar o assunto da tua intervenção?
\item
  Nem pensar nisso.
\item
  Bem, deixo-te em lista de espera.
\end{itemize}

À uma e meia da manhã, Ágata rendeu-se à evidência.

\begin{itemize}
\tightlist
\item
  Acreditas agora em mim? --- perguntou Ágata a Eva, no dia seguinte.
\end{itemize}

As acções e as palavras de Ágata tinham-lhe feito merecer uma reputação
incipiente de contestatária. Alguns colegas de vários cursos começaram a
passar tempo com ela, a exprimir de viva voz as suas inquietações e
críticas. Ágata, consciente da sua condição de caloira e do muito que
lhe faltava aprender,

\textbf{102 }ALEXANDRE ANDRADE

hesitava em assumir-se como a líder por que pareciam ansiar aqueles que
a seguiam, aqueles que escutavam as suas palavras com a compenetração de
um prosélito.

Alguns desses rapazes e raparigas, já com várias matrícu- las em cima,
forneciam-lhe a informação preciosa de que uma recém-chegada precisa.

--- Esta lei que nos rege... --- Ágata pensava em voz alta, num corredor
discreto da Faculdade que apenas conduzia a um arrumo e um laboratório
temporariamente inactivo. Estava rodeada por um pequeno grupo, atento e
disciplinado.

\begin{itemize}
\item
  Esta lei não pode ser o fim e o princípio, não pode ser a última
  palavra. É preciso que exista um estrato superior de legalidade. Todos
  os países, todas as organizações, possuem uma constituição ou
  estatutos.

  \begin{itemize}
  \tightlist
  \item
    Havia o Vasques... --- disse uma moça em tom cons- pirativo, que se
    assustou com o eco das próprias palavras.
  \item
    Quem é esse Vasques?
  \item
    Um veterano de Belas-Artes --- esclareceu um jovem de olhos cavados.
    --- Nunca o conheci pessoalmente, mas falei com quem lidou com ele.
    Dizia-se que sabia de cor todas as leis e regu- lamentos, mas que
    também conhecia na ponta da língua uma Lei Suprema que foi redigida
    logo na fundação desta comunidade.
  \item
    A lenda é mais complexa do que isso --- interveio outro dos
    satélites que orbitavam em torno de Ágata. --- Contava-se que tinha
    memorizado a Lei Suprema, de que não existem cópias escritas.
    Dizia-se que se considerava o único guardião desses princípios
    fundadores cujo espírito ele achava ter sido irremediavelmente
    corrompido. Quem o conheceu diz que ele se sentia muito bem nesse
    papel.
  \item
    E onde posso encontrar esse Vasques? --- perguntou Ágata.
  \end{itemize}
\end{itemize}

VAGA DE FUNDO \textbf{103}

\begin{itemize}
\item
  \begin{itemize}
  \tightlist
  \item
    Ele deixou a vida académica sem concluir o curso. Há quem diga que
    foi por culpa da bebida, há quem fale num desgosto amoroso. Agora
    vive numa casa rústica na Póvoa de Lanhoso, com a mãe.
  \end{itemize}
\end{itemize}

Nessa mesmo dia, Ágata fez vários telefonemas até conse- guir obter
informações sobre horários de camioneta para a Póvoa de Lanhoso. Estava
ao ar livre, a uma distância pru- dente da saída da Faculdade. Enquanto
falava, quase num sus- surro, lançava olhares apreensivos sobre o ombro,
imprimia rotações bruscas e atentas sobre si mesma.

A tal ``casa rústica'' afinal nada tinha de rústico. Era uma vivenda
sóbria, volumosa, imponente. Nem sequer ficava na Póvoa de Lanhoso, mas
sim nos arredores.

Ágata foi muito bem recebida. O tal Vasques era clara- mente um
excêntrico, seria ocioso negá-lo, mas a hospitalidade era a de um
cavalheiro à moda antiga. Falaram até ao cair da noite. Vasques até lhe
propôs que pernoitasse; a presença vigi- lante da mãe era penhor da
ausência de segundas intenções. Ágata declinou, ciente de que sobrava
tempo para apanhar a última camioneta de regresso a Braga.

No seu quarto, Ágata escreveu no diário (negligenciado, quase virgem):
«\textbf{Eu só sei que estava cega e que agora con- sigo ver}». A
citação brotou-lhe da memória com a fluidez de um pensamento original.

No dia seguinte, quase se esqueceu de desejar bom dia a Eva na ânsia de
lhe contar o seu encontro da véspera.

\begin{itemize}
\item
  \begin{itemize}
  \tightlist
  \item
    Eu pensava que esse Vasques não passava de uma lenda urbana.
  \item
    Pois bem, ele é bem real e contou-me coisas sur- preendentes.
    Vivemos na ilegalidade e na imoralidade. A Lei Suprema, que é justa
    e nobre, que reflecte a sabedoria e a sensi-
  \end{itemize}
\end{itemize}

\textbf{104 }ALEXANDRE ANDRADE

bilidade daqueles que a redigiram, é violada dia após dia. As leis que
nos regem estão fora da Lei! Achas isto normal, Eva?

\begin{itemize}
\tightlist
\item
  Ninguém nos garante que a memória do Vasques seja 100\% fiel.
\item
  A quem recorrer? Quem zela pela Lei Suprema?
\item
  Havia um grupinho que se costumava reunir num sa- lão de bilhar um
  bocado manhoso. Se bem me recordo (mas esqueci-me de quem me disse
  isto, e quando, e se foi a brincar ou a sério), cabia-lhes a função de
  fiscalizar o cumprimento da Lei Suprema. Mas não conheço ninguém que
  os leve a sério. São instáveis, têm hábitos estranhos, já houve quem
  lhes cha- masse ``dissipados''.
\item
  Isto não pode continuar assim. É que não pode mesmo.
\end{itemize}

Ágata começou a investir uma porção maior do seu tempo e da sua atenção
no estabelecimento de uma teia de relações duradoura, composta por gente
influente. A aprendizagem das minuciosas hierarquias e escalas de
valores que regiam a vida académica foi dura e lenta. Ágata começou a
dar-se com os finalistas de Gestão, com as estrelas da equipa feminina
de voleibol, com os \emph{webmasters, }com os bolseiros do laboratório
de lasers.

O tempo passava. Os exames do primeiro semestre começaram a reclamar
muita da sua atenção. Depois, chegou a altura de se inscrever nas
cadeiras do segundo semestre: Biomoléculas, Estatística, Física Geral,
Laboratórios de Química II e Química Orgânica. O frio insinuava-se nas
mentes e nas conversas, como uma nova personagem de romance.

Como uma espécie de enredo secundário, uma nova pes- soa passara
entretanto a fazer parte da vida de Ágata. Com o humor ligeiro próprio
dos apaixonados, Ágata contava que

VAGA DE FUNDO \textbf{105}

devia ao acaso e às ruas estreitas de Braga, em partes iguais, a
aproximação com Renato. Fora à saída de um concerto da banda Pontos
Negros, no centro da cidade. Entre aqueles que compunham o grupo, no
máximo uma dúzia, metade eram conhecidos de Ágata e a outra metade eram
conhecidos de conhecidos. Um carro mal estacionado obrigou-os a fluir em
dupla fila atabalhoada entre a chapa e a fachada do prédio vizinho, e
foi assim que Ágata e Renato, um dos conhecidos de conhecidos, se viram
lado a lado e começaram a conversar. A conversa, no início não mais que
uma troca de impressões superficial, estendeu-se e cresceu até excluir
tudo o resto, in- cluindo os companheiros, o fluir da madrugada e o
dédalo de ruas que percorriam, sem pensar, à deriva.

Renato era estudante de Filosofia do segundo ano. O que nele mais atraía
Ágata eram a voz, as mãos e a sensibilidade que o fazia ir directamente
até ao âmago das coisas sem perder de vista os detalhes, a beleza
fulgurante do supérfluo. Mar- caram encontro para o dia seguinte.
Passaram a ver-se quase todos os dias.

Ágata considerou a hipótese de dividir as páginas do seu diário em duas
colunas, uma para a vida pessoal e sentimental e a outra para o
activismo político. Mas essa ideia depressa se lhe tornou estranha e
artificiosa; não chegou a pô-la em prática.

Que alternativas se colocavam para derrubar aquele poder iníquo que
mantinha sob o seu jugo odioso toda uma comu- nidade?

\begin{itemize}
\tightlist
\item
  O Vasques contou-me que existem mecanismos de \emph{im- peachment,
  }devidamente consagrados e prontos a serem aplica- dos. Mas parece-me
  inútil contar com a rapaziada patusca do salão de bilhar para levar
  isso a cabo.
\end{itemize}

\textbf{106 }ALEXANDRE ANDRADE

\begin{itemize}
\tightlist
\item
  Sim, seria uma perda de tempo.
\end{itemize}

Eva estava a enxugar uns pratos. O dia era de chuva. Ága- ta seguia com
o olhar fixo o tamborilar das gotas de água no vidro da janela, sem
prestar atenção.

\begin{itemize}
\tightlist
\item
  A outra hipótese --- disse Eva ---, se de facto não estás disposta a
  esperar dois anos, é tentar persuadir uma maioria de dois terços na
  câmara baixa a votar uma moção de censura. Não existem precedentes,
  que eu saiba, mas há sempre uma primeira vez para tudo.
\item
  Dois terços é muito. E eu conheço tão poucos mem- bros da câmara
  baixa.
\item
  Mas agora estás tão bem relacionada que não há-de tardar muito até
  chegares ao coração do poder. É um trabalho de paciência.
\end{itemize}

A paciência era algo de que Ágata era capaz. Noite após noite, semana
após semana, deu-se com as pessoas certas, relacionou-se, apalpou
terreno, frequentou festas deploráveis apenas porque nelas se aguardava
a presença de alguém que queria aliciar para a sua causa, aceitou
convites para noitadas de jogos de tabuleiro totalmente enfadonhas e
dispensáveis, entregou-se a pequenos e médios tráficos de influência,
con- versou, bajulou, persuadiu, insistiu. Em caves e em esplanadas, em
bibliotecas e em nichos discretos de alguns dos bares da moda,
conspirou, argumentou, contra-argumentou, per- suadiu, doutrinou.
Conheceu pessoas extraordinárias, fez daquelas amizades que se vê logo
que irão durar uma vida in- teira, causou impressão pela solidez das
suas convicções e pelo rigor das suas análises. E porém, ao fim de
algumas semanas de esforço, será que Ágata se sentia mais próxima do
objectivo de seduzir para a sua causa um número suficiente de pessoas
para suportar uma moção de censura? Era forçoso reconhecer

VAGA DE FUNDO \textbf{107}

que não. O reconhecimento da sua capacidade de liderança, as afinidades
de opinião, os amigos influentes, tardavam em traduzir-se em apoios
concretos. A aproximação dos exames de frequência complicava ainda mais
a situação: era preciso estudar, o tempo livre faltava, os estudantes
faziam-se escassos nos lugares públicos da bela e nobre cidade de Braga.

\begin{itemize}
\tightlist
\item
  Sei o que me resta fazer, Eva.
\item
  Até tenho medo do que vais dizer.
\item
  Quando a lei não é justa, quebrar a lei não só não é um crime, como se
  converte em obrigação moral.
\item
  Afinal tinha razão em sentir medo.
\item
  Foi o Vasques quem me explicou isto.
\item
  Amiga, por favor, não te metas em sarilhos.
\item
  Esgotadas que foram todas as vias legais, resta-me ape- nas uma
  alternativa.
\item
  Não tens tempo para golpes de estado. Concentra-te na Química
  Orgânica. Faças o que fizeres, não desleixes a Química Orgânica, senão
  vais-te arrepender.
\item
  Uma coisa não impede a outra. Agora tenho sono, vou-
\end{itemize}

-me deitar.

Chegara a hora da clandestinidade. Ágata não se retirou da vida activa,
mas foi dando a entender, de forma discreta e controlada, que abandonara
as suas pretensões mais ou menos sediciosas e que se confinaria
doravante ao modesto mas ven- turoso escopo que cabe tradicionalmente
aos estudantes uni- versitários.

Ágata só comunicou as suas intenções reais a um grupo muito restrito de
fiéis. Entre essas duas ou três dezenas de rapa- rigas e rapazes
passaram a vigorar os mecanismos do segredo e da cautela. Encontravam-se
muito de longe em longe, limita- vam a conversação a temas banalíssimos
quando em público,

\textbf{108 }ALEXANDRE ANDRADE

desenvolveram um intrincado esquema de senhas, contra-

-senhas e códigos. Aguardavam, vigilantes, a sua hora. Sabiam que a
vitória final lhes pertenceria.

As suas faces contraíam-se de expectativa, nos seus olhos brilhava uma
centelha feroz, permanente.

Discutiam entre si o direito natural, as arquitraves do edi- fício
teórico que os sustentava e o modo de fazer. Ágata inquie- tava-se por
causa do modo de operação, interrogava-se sobre os passos a seguir,
sentia-se ignorante e indigna da confiança dos seus seguidores. Mas como
se aprende a insurreição? As lições da História podem ajudar, mas apenas
até certo ponto. Numa terça-feira de Maio (a Primavera instalara-se em
força, uma brisa doce acariciava os braços timidamente nus, Braga
parecia toda ela cheirar a fruta e a roupa lavada de fres- co), uma das
conjuradas pôs discretamente nas mãos de Ágata, que saía de uma aula de
laboratório, um panfleto toscamente impresso. Ágata examinou-o com
cuidado. A gramática era péssima, a qualidade gráfica deplorável, mas a
mensagem era clara. Apelava-se à revolta. Haveria então mais um grupo,
além deles, a tentar tomar o poder pela força? Quantos seriam?

Quem seria o seu líder? Como estariam organizados?

Foram feitas diligências e inquéritos. Perguntas discretas foram
colocadas com um ar casual. Gestos alheios foram acom- panhados de
perto, especulou-se sobre os passos seguintes dos rivais. Os rumores
nasciam e propagavam-se como projécteis durante a sua vida efémera.

Os exames estavam à porta; a mobilização ressentia-se disso. Mesmo nas
reuniões com o seu núcleo duro (realizadas sempre em local diferente,
com todas as elaboradas aparências da normalidade), Ágata assinalava
agora sempre uma ausência, quando não eram mais.

VAGA DE FUNDO \textbf{109}

\begin{itemize}
\tightlist
\item
  Receio bem que tenhamos deixado passar a altura estratégica para
  actuar --- queixou-se Ágata a Eva. Era uma manhã de sábado. Tinham
  ficado em casa para esperar a se- nhoria, que ia trazer um forno de
  microondas para substituir o antigo que se tinha avariado. Tinham
  deixado as janelas aber- tas para deixar entrar algum ar fresco. --- O
  momento certo era antes da época de exames. Agora, Braga esvazia-se a
  olhos vistos. Não há nada para ver. Mesmo que passássemos à acção e
  triunfássemos, a nossa vitória seria irrelevante.
\item
  Nem tudo foi trabalho perdido, amiga. Estabeleceste contactos,
  despertaste simpatias, alimentaste o debate. Isso vale ouro! Dá tempo
  ao tempo, deixa que o processo se desen- role ao seu ritmo, espera
  pela vaga de fundo.
\item
  A vaga de fundo?
\item
  Uma miríade de vontades que de repente se transforma numa única
  vontade. Deverás estar muito atenta. Os sinais serão subtis mas
  inconfundíveis. Saberás então que chegou a altura. Que calor que está
  aqui dentro! O ar não circula.
\end{itemize}

Na opinião de Renato, Ágata exigia demasiado de si própria, corria atrás
de demasiadas lebres ao mesmo tempo, era preciso parar para respirar
fundo, redefinir prioridades, evitar perder o norte, manter os pés
assentes na terra no meio do turbilhão de ideias e sensações novas, algo
tão natural num caloiro uni- versitário arrancado à sua remota terra
natal.

Nos braços de Renato, o tempo parecia suspender-se.

Concluídos os exames e as melhorias, Ágata regressou a X\ldots{} Adiou a
resposta aos convites de amigos antigos e de novos colegas que a
incitavam a partir com eles mais ou menos ao sabor do acaso e a
aproveitar o Verão. Sentia vontade de repousar durante uma ou duas
semanas.

Apeou-se da camioneta e mergulhou na inércia, no calor,

\textbf{110 }ALEXANDRE ANDRADE

na cadência indolente dos trabalhos sazonais.

O pai achou-a mais magra, a mãe achou-a mais gorda. Ágata encontrou o
seu quarto limpo e arrumado, a cama feita. Folheou alguns livros cuja
existência quase esquecera. O réptil de barro que a irmã fizera na
escola e lhe oferecera estava fora do lugar.

Ágata falava com Renato quase todas as noites, por Skype.

A ligação falhava com muita frequência.

Ágata recebeu um postal ilustrado de Eva que represen- tava o santuário
do Bom Jesus, fotografado segundo quatro ângulos diferentes. «Desculpa
lá mas não consegui encontrar nada mais piroso, e não foi por falta de
procurar. Escrevo-

-te desta amostra de metrópole, com uma caneta esferográfica azul que
ameaça deixar-me mal a cada letra\ldots{}»

Com a brevidade a que a obrigava o rectângulo de cartoli- na, Eva
revelava novidades inesperadas. As recentes alterações nas leis da
associação iam obrigar a novas eleições. Tudo dis- solvido, tudo por
reconstruir, novos estatutos por redigir. Em suma, a tábua rasa com que
sonhou Descartes.

Portanto, era assim.

Ah, ter a urgência da juventude, a impressão de cavalgar o tempo, a
pessoa amada e oito metros quadrados de superfície útil à sua espera,
mais as áreas comuns.

DEZEMBRO 2011/JANEIRO 2012

\subsection{VOI CHE SAPETE}

á quem afirme que as vagas de calor demasiado intensas têm o poder de
diluir a virtude e a moral num caldo moroso de acasos e acidentes. No
Verão de que aqui se fala, Coimbra era castigada com temperaturas
próximas dos 40 graus pela terceira semana consecutiva. Só se permanecia
na cidade por obrigação, infortúnio ou inconsciência. Os espaços
públicos, castigados pelo sol durante os dias demasiado lon- gos, eram
percorridos com parcimónia e cautela. Sussurrava-se em vez de falar,
dentro de portas e no exterior, como se a espe- rança de alguma vez a
canícula acabar fosse um animal arisco.

Uma residência de estudantes é um lugar cuja ocupação é, por natureza,
sazonal. Os efeitos conjugados da inclemência estival e do calendário
escolar tinham deixado deserta uma república, situada na parte velha de
Coimbra, que se distin- guia das demais por não ter nome. O edifício,
alto e estreito, confundia-se com as fachadas castanhas dos que o
ladeavam. Havia cinco andares; em cada andar, havia um só aparta- mento,
com três quartos e uma cozinha comunitária. Subia-se por uma escadaria
estreita e irregular. Entre cada andar, uma janela larga permitia
avistar um pátio interior, comum a ou-

\textbf{112 }ALEXANDRE ANDRADE

tros edifícios contíguos. No pátio, havia um tanque de lavar a roupa,
uma macaca desenhada no chão a giz branco, dois cones de sinalização
rodoviária a servir de baliza e um car- rinho de supermercado, como num
filme com escrúpulos de realismo social.

Vasco descia as escadas entre o quarto e o segundo andar. Descia as
escadas agora e repetidas vezes, dia após dia. A situação era a
seguinte. Vasco vivia num quarto do segundo andar, um quarto exíguo em
todas as dimensões menos na da altura, tornado ainda mais pequeno devido
à acumulação de objectos, roupas, jornais, latas vazias. Era o quarto
mais exposto ao sol durante o dia, aquele em que o ar se tornava mais
hostil de tão irrespirável, aquele em que uma permanên- cia mais
prolongada se confundia com um acto de loucura. E contudo Vasco
partilhara aquele mesmo quarto com outra pessoa durante um ano completo,
sem excluir qualquer das es- tações e sem guardar recordações de
desconforto. Quase todos os livros, cerca de dois terços dos CDs e
talvez uma ou outra peça de roupa de pequena dimensão, perdida num fundo
de gaveta, pertenciam ainda a Leda, que chamara também seus àquele tecto
e paredes durante esse tempo, até ter decidido ir-se embora. Leda vivia
agora no mesmo prédio, mas dois andares mais acima. O seu quarto era
mais fresco, mais amplo e mais escuro do que aquele que partilhara com
Vasco. Vasco gostava de pensar que era devido à nostalgia, ou a um
tortuoso mecanismo de negação, que Leda preferira deixar quase todos os
seus livros no segundo andar. Leda dedicava todas as horas de cada dia
deste Verão, quente a ponto de se poder falar em maldição divina sem
receio de chocar, à conclusão da sua tese sobre as campanhas militares
que se seguiram à restauração de 1640. Quando Leda precisava de
consultar um livro de

VOI CHE SAPETE \textbf{113}

referência, verificar uma data, tirar a limpo um detalhe, pedia o
obséquio a Vasco, que nunca estava a uma distância que um chamamento de
volume mediano não fosse capaz de transpor. Quase sempre Vasco
languescia na divisão contígua àquela onde Leda trabalhava, folheando
bandas desenhadas antigas e ansiando pela ocasião de ser útil. Quando os
seus serviços eram solicitados, descia as escadas quase com solenidade,
en- trava no seu quarto inabitável e permanecia apenas o tempo de
folhear o Veríssimo Serrão, o Alexandre Herculano ou o Oliveira Marques,
gravar na memória a informação requerida e regressar para a transmitir
ao mais atento e delicado ouvido de Coimbra.

De cada vez, Leda agradecia com um sorriso luminoso e nunca deixava de
insistir com Vasco para que ele saísse, apanhasse ar, aproveitasse a sua
condição de homem livre sem compromissos nem prazos para respeitar.
Vasco deixava ver o sorriso pudico daqueles que prezam as pequenas
dádivas da vida num grau que o senso comum consideraria excessivo, ga-
rantia a Leda que preferia ficar ali ao seu serviço, que a pilha de
bandas desenhadas ainda nem ia a meio, que já se afeiçoara ao gato que
costumava estar empoleirado no parapeito da janela entre o terceiro e o
segundo andares.

Vasco descia as escadas com a lentidão dos ociosos mas sem a indolência
no passo, que era tenso e ritmado. Trazia na mão um papel onde estava
escrita uma pergunta, na caligrafia de Leda, que era elaborada e
angular. Qual foi a carreira do duque de Osuna depois do seu malogro
como comandante das tropas espanholas na batalha de Castelo Rodrigo?
Leda escrevia as perguntas em rectângulos de papel que eram folhas de
rascunho cortadas em oito. Leda repetia a pergunta em voz alta depois de
estender o rectângulo de papel a Vasco.

\textbf{114 }ALEXANDRE ANDRADE

Com o tempo e a prática, Vasco aperfeiçoara a arte de aparen- tar
indiferença apesar do alvoroço que nele causava aquela voz plácida e
musical, alimento e fundamento da sua vida, inspiração e alegria
sublime.

Entre o terceiro e o segundo andares, a janela tinha como ocupante
habitual um gato de pêlo longo, que se acomodava o melhor que podia na
estreita nesga do parapeito e vigiava os acontecimentos no pátio, tão
raros e esparsos nesta altura do ano. O gato não se distinguia pelas
aptidões sociais, e Vasco, por norma, limitava-se a responder à
indiferença dele com a sua indiferença. Nesse dia, um súbito movimento
de cabeça de um animal tão plácido e pouco dado aos repentes chamou-

-lhe a atenção. O que vira ele pela janela? Vasco olhou por cima do
dorso tenso do gato, sem conseguir vislumbrar o que quer que fosse que
pudesse ter justificado o sobressalto. Atribuiu a fugaz e minúscula
impressão alaranjada no seu campo visual a um artefacto óptico.
Apressou-se a descer o que faltava para chegar ao seu quarto inabitável,
esclareceu nos compên- dios o destino honroso do duque de Osuna, voltou
a subir as escadas sem se permitir mais do que um relance mecânico ao
bichano novamente inerte, transmitiu as informações a Leda, que o
escutou de perfil como num retrato de Piero Della Francesca, atenta e
pródiga em gratidão na justa me- dida do favor.

Um dia sucedeu-se ao anterior, na cadência mole apro- priada ao mês e à
cidade paralisada pelo calor. Vasco acordou com uma banda desenhada no
colo, deitado no colchão que por vezes acolhia as suas sestas, ao lado
da sala de trabalho de Leda. Quanto tempo dormira? Não tinha relógio e
deixara o telemóvel no quarto. A luz do dia infiltrava-se por debaixo da
porta. Vasco ouviu o ruído dos dedos de Leda no teclado,

VOI CHE SAPETE \textbf{115}

bateu à porta, entrou. Leda saudou-o, risonha, fresca, sem de- nunciar a
noite sem sono (mais uma).

--- Está um dia esplêndido, como sempre. Queres ser um anjo mais uma vez
e ir procurar a referência daquela citação sobre o número de canhões e
morteiros do exército espanhol na batalha das Linhas de Elvas?

Desta vez, Vasco assomou à janela assim que viu a tensão no corpo do
gato. Abriu a janela, que era de batentes. Esprei- tou para fora e viu
no pátio a rapariga vestida de cor-de-

-laranja, quase encostada à fachada, a expressão e a atitude de quem
aguarda algo de belo e que virá de cima. Não era para Vasco que ela
olhava, mas sim para um dos andares superiores. Rodar o pescoço para
olhar para cima foi uma decisão por parte de Vasco, mas o movimento de
interceptar e guardar o objecto que tombava foi um reflexo puro.

Era uma flor, uma rosa vermelha, fresca, cor de vinho. Um dos espinhos
espetara-se-lhe no polegar. Vasco chupou a gota de sangue.

Lá fora, a rapariga continuava a olhar para cima. No seu olhar não havia
decepção nem dor, apenas a absurda certeza de que o que havia para
acontecer acabaria por acontecer graças ao fluxo normal das coisas e não
por qualquer oblíquo cambiante do destino. O último relance que Vasco se
conce- deu deixou-lhe na memória a imagem de um rosto redondo, branco,
maquilhado; de sobrancelhas longas e escuras; de um penteado à Louise
Brooks mas em louro-melaço; de um sor- riso muito doce que parecia
desenhado.

Vasco trazia a rosa na mão direita, na mão esquerda o quadrado de papel
rabiscado com as informações que Leda solicitara.

--- Que flor tão bonita --- disse Leda, roçando com o nariz

\textbf{116 }ALEXANDRE ANDRADE

as pétalas para sentir o seu aroma. Demorou-se, em silêncio. Pôs a flor
num copo cheio de água que estava à sua beira, como de propósito.

\begin{itemize}
\tightlist
\item
  Não te piques --- disse Vasco.
\item
  Não te preocupes. De onde vem esta rosa?
\item
  Encontrei-a por aí. Três morteiros e dezanove canhões, mas um deles
  perdeu uma roda e ficou imprestável.
\end{itemize}

Quem teria arremessado a rosa? As especulações cruza- vam a mente de
Vasco e interferiam com a sua leitura. Estaria alguém a viver no quinto
andar do prédio? Vasco acreditava que a república estava deserta com a
excepção de Leda e dele próprio, mas não podia ter a certeza. Saiu para
o patamar mo- vido pela intenção, pouco firme, de subir um andar e
dissipar as dúvidas, mas a descida pareceu-lhe de súbito infinitamente
mais desejável do que a ascensão. Lembrou-se de que ainda sobrava uma
lata de Coca-Cola fresca no pequeno frigorífico que lhe servia de
mesa-de-cabeceira. Desceu até ao seu quarto. Da janela, avistava-se um
final de tarde que prometia al- guma trégua no calor. Vasco decidiu ir
fazer compras a um minimercado próximo. Era pouco provável que Leda o
cha-

masse àquela hora.

No caminho de regresso, Vasco foi interceptado por uma jovem da sua
idade: vestido curto, cabelo volumoso e frisado, óculos escuros de
lentes grandes e redondas. Vasco recordava-

-se vagamente de a ter visto nos corredores da Faculdade, mas não tinha
a certeza disso. Pousou no chão os sacos de compras.

\begin{itemize}
\tightlist
\item
  És tu aquele que desvia flores do seu percurso natural com a agilidade
  de um atleta?
\item
  Não faço disso um hábito.
\item
  Não é crime nem é delito, mas reparaste na expressão de desgosto da
  destinatária?
\end{itemize}

VOI CHE SAPETE \textbf{117}

\begin{itemize}
\tightlist
\item
  A expressão dela não foi de desgosto. Eu estava só a dois andares e
  meio de altura e distingui claramente um sor- riso. Parecia desenhado
  na maquilhagem, mas era sincero.
\item
  A tua maquilhada sorridente tem nome, sabias? Cha- ma-se Lídia, é um
  amor de pessoa, estuda Linguística e sorrir de orelha a orelha é a
  maneira dela de dizer que a vida acaba de lhe desferir um golpe.
\end{itemize}

Quem seria aquela rapariga? O que a teria levado até ali? O seu tom era
mais de ironia do que admoestação. Havia congelados dentro dos sacos de
plástico. Vasco não podia de- morar-se no meio daquele fim de tarde de
Verão.

\begin{itemize}
\tightlist
\item
  Deixa-me ajudar-te com esses sacos. O meu nome é Berenice.
\end{itemize}

Estavam a poucas dezenas de passos da república. Respira- va-se a custo
dentro do quarto de Vasco. Vasco não convidou Berenice para ficar.
Continuaram a conversa no patamar som- brio do segundo andar.

\begin{itemize}
\tightlist
\item
  Essa Lídia é uma amiga tua?
\item
  Passamos tempo juntas e trocamos confidências em número suficiente
  para eu distinguir nela a indiferença fingida da indiferença
  autêntica, para eu saber que para ela o amor é isto: um objecto a
  fazer as vezes de testemunho de um senti- mento genuíno, uma flor
  arremessada a horas certas, guardada e aconchegada contra o peito.
\item
  A inversão dos papéis: o rapaz à janela como uma donzela emparedada
  pelo ciúme; a rapariga livre de palmilhar a cidade, passear o seu
  segredo pelos recantos de Coimbra. Quem é ele? Pensei que o prédio
  estava deserto.
\item
  Não é como se eles não se vissem no mundo: saem, vão aos bares,
  frequentam festas. Mas é para o ritual que vivem, para a antecipação
  do momento, para a repetição sem risco.
\end{itemize}

\textbf{118 }ALEXANDRE ANDRADE

\begin{itemize}
\tightlist
\item
  Arrependo-me de ter agarrado a rosa. Ainda por cima, feri-me num
  espinho.
\item
  Tira o penso do dedo. Parece-se demasiado com um troféu e por esta
  altura a ferida já cicatrizou há muito. Sufoca-
\end{itemize}

-se aqui dentro, o ar não circula. Vem daí, deixa o Rapunzello do quinto
andar debruçado da janela, entretido a subverter os estereótipos de
género dos contos de fadas. Quero mostrar-te uma coisa, aliás várias
coisas.

\begin{itemize}
\tightlist
\item
  Tenho alguém à minha espera.
\item
  Quem espera por ti?
\end{itemize}

Vasco falou-lhe de Leda, descreveu Leda e a teia de obri- gações que o
unia a ela, tão mais complicada quando posta em palavras do que vivida.

\begin{itemize}
\tightlist
\item
  Não é caso de vida ou de morte --- disse Berenice. --- A tua laboriosa
  Leda não sentirá a tua falta e, se sentir, nada a impede de descer
  dois andares pelo seu pé, folhear os com- pêndios e confrontar a sua
  impressão com a palavra do histo- riador escrita preto no branco.
  Anda, o fim de tarde está mara- vilhoso. Com um pouco de imaginação,
  até se sente uma brisa a refrescar a pele. Queres trazer alguma coisa,
  ou talvez subir para ir buscar uma banda desenhada?
\item
  Estou bem.
\end{itemize}

O que Berenice tinha para mostrar a Vasco ficava na parte mais velha de
Coimbra e era um edifício largo e baixo, cuja fachada, crivada de
janelas pequenas e quadradas distribuídas sem ordem aparente, remetia
para tempos antigos. Dava para um largo aonde se chegava por uma rua
estreita e inclinada em que mal cabiam duas pessoas de braço dado.

\begin{itemize}
\tightlist
\item
  É aqui que eu moro --- disse Berenice. --- Esta é uma república de
  estudantes, mas prepara-te para contar as dife- renças em relação à
  tua. Para começar, repara nas dificuldades
\end{itemize}

VOI CHE SAPETE \textbf{119}

técnicas que se colocariam a quem pretendesse apoderar-se de projécteis
arremessados do edifício para o nível térreo, fossem estes testemunhos
de amor romântico ou elementos do mais rasteiro quotidiano. Mas ninguém
se entrega a esses exercícios por aqui. Não confiamos na força da
gravidade para alcançar os nossos objectivos, conhecemos outros meios,
menos expe- ditos mas mais obedientes à nossa vontade de homens e mu-
lheres livres. Subimos?

A república era composta por um único piso. Um corre- dor longo e
rectilíneo era ladeado pelas portas dos quartos, dispostas regularmente.
Ao fundo havia um espaço comum que funcionava como cozinha e sala de
convívio.

\begin{itemize}
\tightlist
\item
  Conta-se que isto foi um convento --- disse Berenice
\end{itemize}

---, mas também há quem garanta a pés juntos que nunca pas- sou de um
albergue para viajantes. Há quartos individuais e quartos ocupados por
mais de uma pessoa. Ninguém controla quem dorme em cada quarto. A
república é gerida por um colectivo, somos os nossos próprios senhorios.

\begin{itemize}
\tightlist
\item
  Deve fazer muito frio, no Inverno.
\item
  Passamos horrores por causa do frio, agravados pelo hábito de andarmos
  descalços ou de meias. O atrito das solas dos sapatos com este chão de
  tacos de madeira produz um ruído abominável e pouco amigo da
  discrição.
\item
  Discrição? De quem se querem esconder?
\item
  Esta arquitectura é a ideal para criar laços entre as pes- soas,
  exactamente ao contrário do amontoado vertical de estratos onde gastas
  os teus dias. Aqui, quando alguém tem alguma coisa a tratar com
  alguém, sai do seu quarto e vai ter com essa pessoa, passo após passo,
  com toda a naturalidade. Às vezes isso é feito à vista de todos,
  outras vezes a situação requer sigilo. É como na vida.
\end{itemize}

\textbf{120 }ALEXANDRE ANDRADE

\begin{itemize}
\tightlist
\item
  Sigilo? Dá-me um exemplo...
\end{itemize}

Tinham chegado à sala comum, que era ampla e parecia ter sido mobilada
ao sabor dos caprichos de sucessivas vagas de ocupantes. Berenice
apontou para um sofá coberto por um xaile multicor, gasto pelo uso, e
deixou-se cair numa cadeira de baloiço.

\begin{itemize}
\tightlist
\item
  Dorme aqui esta noite, Vasco. Se é exemplos que que- res, vais ter um
  nunca-acabar de exemplos. Uma colecção, um rosário interminável de
  exemplos, um caleidoscópio de ex- periências que irá reduzir à
  condição de termo de comparação indigno o prosaico arremesso de uma
  rosa, na esperança de que o ente querido a apanhe em vez de um
  \emph{voyeur }oportunista.
\item
  Este sofá é confortável, mas não me vejo a dormir nele. Não tenho
  roupa para mudar nem escova de dentes. Acho melhor regressar.
\item
  Vasco, fica onde estás. Estamos no Verão, o que não falta são quartos
  livres. És de estatura média, havemos de en- contrar alguém que te
  empreste uma camisa e umas \emph{boxers }compatíveis com o teu sentido
  estético. Há uma loja de con- veniência aberta até tarde a duas
  esquinas de distância. Até podemos comprar pensos rápidos para a tua
  ferida no dedo, que afinal, dou a mão à palmatória, ameaça reabrir a
  qualquer momento. Fica. Se é por causa da tua Leda, dez contra um em
  como ela ainda não se apercebeu da tua escapadela.
\end{itemize}

Vasco dormiu o sono dos justos e dos simples numa cama por fazer de um
quarto austero decorado apenas com duas re- produções: uma da
\emph{Vénus }de Cranach e a outra da lata de sopa Campbell's de Andy
Warhol. A janela quadrada parecia ainda mais pequena vista de dentro do
que vista de fora. Quando acordou, Vasco deixou-se ficar na cama por uns
minutos, a pensar em Leda. Não tinha mensagens novas no telemóvel.

VOI CHE SAPETE \textbf{121}

Sem saber o que fazer, explorou a estante (muita ficção por- tuguesa da
primeira metade do século XX, Joaquim Paço d'Arcos, Raul Brandão, Manuel
Teixeira-Gomes). Quando se fartou, saiu do quarto e dirigiu-se à
cozinha, onde encontrou dois jovens da sua idade: um deles estava de pé,
era alto e ruivo e exprimia-se num português correcto mas cheio de
sotaque (escandinavo, holandês?); o outro, barbudo e avantajado, estava
sentado com uma malga de café com leite nas mãos. Saudaram Vasco com uma
afabilidade que não traía a mínima surpresa por darem de caras com um
desconhecido. Oferece- ram-lhe leite, cereais e metade de uma carcaça.
Despediram-se sem se apresentarem.

\begin{itemize}
\tightlist
\item
  Vejo que já travaste conhecimento com o Klaus e com o Ezequias. ---
  Berenice parecia diferente. Talvez fosse da au- sência de maquilhagem,
  talvez fosse da maneira como estava vestida: blusa preta, calções de
  caqui, sandálias. O cabelo estava apanhado por uma tira de pano
  cinzenta. --- São os madrugadores de serviço. Quando todos os outros
  ainda res- sonam, já eles fizeram 50 elevações, devoraram os seus Corn
  Flakes e descobriram pelo menos uma verdade fundamental da existência.
  Já comeste? Então vem daí.
\end{itemize}

O primeiro dia de Vasco na Residência foi dedicado a conhecer os
protagonistas. Berenice tomou-lhe o braço e fê-

-lo percorrer o corredor com a lentidão de um cicerone cons- ciencioso.
À frente de cada porta, fechada ou mostrando um interior desabitado ou
uma cena doméstica povoada, Berenice inteirava Vasco sobre as
circunstâncias do seu ocupante, numa voz sussurrada mas de dicção
claríssima.

Inácio era um solitário que idealizava o amor. Era raro encontrá-lo no
quarto. Passava grande parte do tempo a vaguear por Coimbra, em busca de
confirmações para os seus ideais.

\textbf{122 }ALEXANDRE ANDRADE

Era um conversador nato e aceitava de bom grado ser contra- riado, mas
as suas construções mentais, buriladas ao longo de anos, eram imunes a
evidências. No trato, era simpático e delicadíssimo.

Tatyana era daquelas que esmorecem e definham após uma ruptura. Era
vê-la agora deitada de bruços em cima da cama, abraçada a uma almofada,
num quarto onde abunda- vam os sintomas de negligência e indiferença. A
curva des- cendente do seu ânimo era tão previsível que se diria
tratar-se de um percurso conhecido e balizado, quase uma purga, em vez
de um vulgar abandono às marés aleatórias da depressão. Tatyana era
filha de pais russos e tinha um sinal de nascen- ça conspícuo na
pálpebra esquerda, negro sobre a pele muito branca, que só se via quando
ela fechava os olhos.

Ester e Minerva amavam-se mas tinham decidido ``dar algum tempo'' a si
mesmas, para verem o que a sucessão dos dias e semanas faria a um
sentimento que julgavam duradouro e sólido. Era uma experiência e um
exercício de conhecimento. Tinham ambas abandonado o quarto que
partilhavam. No quarto, agora deixado vazio, respirava-se a sua
presença.

Luz e Ezequias (o rapaz da malga de café com leite) ti- nham começado
por partilhar gostos e convicções. Com o andar do tempo, as afinidades
tinham-se mudado em carinho e em amor. Agora, passado o sobressalto e a
surpresa de se encontrarem juntos, redescobriam as afinidades (filmes,
livros, quadros) por aquilo que estas eram, e (muito em segredo) re-
ceavam que a matéria-prima de que era constituído o seu amor fosse
afinal demasiado mesquinha.

\begin{itemize}
\tightlist
\item
  Mas tu estás a tirar notas!? --- exclamou Berenice.
\item
  É melhor assim. O que passa pela minha cabeça não deixa rasto ---
  explicou Vasco.
\end{itemize}

VOI CHE SAPETE \textbf{123}

\begin{itemize}
\tightlist
\item
  Deita isso fora e vamos sair. O dia está demasiado lin- do para que a
  ideia de me ofereceres um gelado não te ocorra. Vasco rasgou a folha
  do bloco de notas, mas enfiou-a no bolso de trás das calças em vez de
  a deitar fora. Nunca lhe ocorrera oferecer um gelado a Berenice, mas
  percebeu de re- pente que cobiçava uma taça com três bolas e
  \emph{topping }de cara-
\end{itemize}

melo mais do que qualquer outra coisa no mundo.

Mais tarde, sentado à mesa de uma esplanada na baixa de Coimbra, no
momento em que começara a assimilar a certeza lúgubre de que a amêndoa e
o maracujá casavam melhor na imaginação do que no mundo real, Vasco foi
arrancado à con- templação do progresso penoso dos transeuntes pela voz
de Berenice, para quem aquele momento não passara de um in- terregno
agradável na sua missão de o instruir sobre os labirin- tos das paixões
humanas.

\begin{itemize}
\tightlist
\item
  Falta falar-te da Samanta e do Júlio.
\item
  O que têm a Samanta e o Júlio?
\item
  Esses dois... No passado recente deles haveria pasto para histórias
  das mil e uma noites, repletas de ramificações, de notas de rodapé e
  de camadas sobrepostas, mas vou ficar-
\end{itemize}

-me pelo básico. A Samanta e o Júlio vivem lá na residência e andam
juntos, essa é a parte mais fácil de contar. Estão um com o outro há
anos, adoram-se e foram feitos um para o outro. Mas acontece, vê lá tu,
que se lhes meteu na cabeça que estava na altura de porem ponto final na
relação. Não me perguntes nada sobre o emaranhado de razões, argumentos,
pressentimentos e agoiros que os conduziu a essa conclusão funesta; não
quero ir por aí, e tu, Vasco, tão pouco queres ir por aí, acredita em
mim. O que é certo, e do domínio público, é que eles querem acabar. Ora
bem, uma relação tão longa e tão sólida não se interrompe com um estalar
de dedos. Moral

\textbf{124 }ALEXANDRE ANDRADE

da história: por vontade de esperar pelo momento certo, por inércia, por
isto ou por aquilo, eles continuam juntos, mas cada vez mais impacientes
e implicativos um com o outro.

\begin{itemize}
\tightlist
\item
  Essa história é menos invulgar do que julgas. Conheci dezenas iguais,
  a minha experiência das coisas do coração é menos parca do que julgas.
  Pelo que descreves, esse Júlio e essa Samanta estão simplesmente à
  espera da ocasião certa, que há-
\end{itemize}

-de surgir vinda do nada, como um relâmpago ou uma ideia repentina. É
sempre assim que as coisas acontecem.

\begin{itemize}
\tightlist
\item
  Achas? Achas mesmo que sim? Pois é, já me esque- cia: tu acreditas na
  singularidade, no golpe de asa, no instante de inspiração, na flor
  desviada do trajecto que as leis naturais lhe impunham. Não
  compreendes que certos processos têm de existir no tempo, que as
  cadeias de acontecimentos e os arabescos que os acompanham têm uma
  duração incompres- sível, uma lógica e um ritmo próprios. A não ser,
  claro está, que alguma intervenção exterior mais ou menos providencial
  precipite os acontecimentos. O que mais abunda por aí são \emph{dei ex
  machina }amadores, mais ou menos bem intencionados, mais ou menos
  canhestros. E é aqui que entras tu, Vasco.
\item
  Que entro eu? Não, eu não entro em lado nenhum.
\end{itemize}

Não quero meter-me no que não me diz respeito.

\begin{itemize}
\tightlist
\item
  Deixa-me explicar-te a situação. Vais ver que tem um encanto muito
  especial. O Júlio, fecundo em recursos como sempre foi, teve a
  brilhantíssima ideia de começar a fazer a corte à Penélope. A Penélope
  é uma estudante de Enfermagem que não é má rapariga mas que parece ter
  medo da própria sombra, e cujas afinidades com o Júlio parecem
  resumir-se a uma ascendência transmontana de que ambos se orgulham. O
  estratagema salta aos olhos: o objectivo é o de quebrar o impasse, o
  de mostrar ao mundo e aos principais interessados
\end{itemize}

VOI CHE SAPETE \textbf{125}

que os tempos de idílio entre ele e a Samanta passaram de- finitivamente
à história. Pois bem, surpresa das surpresas, não resultou. A Samanta
não se deixa convencer por um ardil tão transparente. Longe de se
afastar do Júlio, dá sinais de se reaproximar. Aquilo que se impunha,
neste momento, era encontrar um pretendente para a Samanta, mas um que
fosse convincente, não uma marioneta.

\begin{itemize}
\tightlist
\item
  Não aspiro a ser a causa de uma ruptura. O meu ca- dastro passa muito
  bem sem esse pecadilho. Além disso, não tenho tempo.
\item
  Claro que tens tempo, tens tanto tempo entre mãos que não sabes como o
  ocupar, a não ser fazendo recados e subindo e descendo as escadarias
  de um prédio decrépito e de- serto. Pensas que a tua Leda vai sentir
  cruelmente a tua falta? Desengana-te, não deixaste qualquer vazio por
  preencher, um ou dois suspiros de resignação será tudo o que a tua
  ausência merecerá. Só lhe fará bem aprender a levantar-se e queimar
  al- gumas calorias, descer ela própria até ao teu andar e consultar
  ela própria os canhenhos no teu quarto onde o ar não circula quando
  quiser verificar a verdade dos factos que com tanto esforço e devoção
  verte para o seu documento. Não só te sobra tempo, como me parece que
  seria bom para ti consumir esse tempo fazendo parte de um processo
  dinâmico por natureza, sujeito a caprichos, imprevisível, declinável
  em causas e conse- quências, vulnerável ao livre-arbítrio dos seus
  intervenientes. No mínimo dos mínimos, deixa-me apresentar-te a
  Samanta e verás como tudo o resto acontecerá com naturalidade. Já
  acabaste o teu gelado?
\end{itemize}

De volta à residência, Berenice retirou-se discretamente depois de
conduzir Vasco até ao quarto que Samanta e Júlio partilhavam. O quarto
era minúsculo: no interior, mal havia

\textbf{126 }ALEXANDRE ANDRADE

espaço para o colchão de casal e para uma estante da Ikea re- pleta de
livros. Quis o acaso que Samanta estivesse sem mãos a medir com a tarefa
de arrumar os livros da estante. Aceitou de bom grado a oferta de ajuda
de Vasco.

\begin{itemize}
\tightlist
\item
  És amigo da Berenice?
\item
  Mais ou menos. Foi ela quem me trouxe aqui, eu não conhecia esta
  residência. Conhecemo-nos desde ontem.
\item
  Perguntei por perguntar.
\end{itemize}

A maior parte dos livros eram sobre filosofia, psicologia e sociologia.
Samanta mudou três vezes de ideias quanto à ma- neira de os ordenar: nem
a ordem alfabética de autor, nem a época nem o assunto lhe pareceram
critérios adequados. Ao fim de um par de horas, vencidos pela enormidade
do esforço, limitavam-se já a conversar, sentados em cima do colchão.
Samanta mostrava a Vasco algumas passagens escolhidas dos livros que ia
retirando, um pouco ao acaso, do montículo que tinha aos seus pés.

A fisionomia de Samanta não encaixava no estilo a que Vasco atribuía a
sua preferência. Samanta era alta e estreita de corpo. O seu físico de
bailarina condizia com a precisão mus- culada que imprimia aos seus
gestos mais triviais. O cabelo era castanho, corrido e solto, os traços
faciais pareciam con- vergir para o queixo pontiagudo, os olhos eram
inteligentes e vivazes. Vasco habituara-se a aceitar como facto
consumado a sua parcialidade para com figuras cheias e plácidas, mais
dadas à escolástica e ao repouso do que ao exercício, mas viu-

-se compelido a achar Samanta atraente e não pôde suprimir a vontade de
passar mais tempo com ela.

\begin{itemize}
\tightlist
\item
  Deixemos os livros como estão, a confusão engendra a descoberta ---
  disse Samanta. --- Ofereço-te um copo para agradecer os teus bons e
  leais serviços. Que dizes?
\end{itemize}

VOI CHE SAPETE \textbf{127}

\begin{itemize}
\tightlist
\item
  Eu ofereço o segundo.
\item
  Conheço um sítio fixe, mas é longe de mais para ir a pé. Tens carro?
\item
  Não tenho carro, mas sei a quem pedir emprestado.
\item
  Não te preocupes, eu tenho mota. A não ser que a ideia te assuste.
\end{itemize}

Passaram um serão muito agradável, num bar onde o am- biente era
fantástico, a música excelente.

Samanta ainda não falara sobre Júlio; Vasco preferia não lhe dar
entender que estava ao corrente de tudo.

Na manhã seguinte, Vasco acordou no mesmo quarto alheio da noite
anterior com a vaga certeza de que, para ter passado a noite com
Samanta, teria bastado quase nada, uma ou duas frases que manifestassem
essa vontade, um olhar mu- tuamente sustentado por mais um ou dois
segundos.

Vasco levantou-se e vagueou pela residência, incapaz de encontrar
Samanta. Berenice não estava no quarto. Mesmo àquela hora matinal, o
calor anunciava a sua presença, inva- dia os espaços, dificultava a
respiração. Do fundo do corredor, Klaus acenou amigavelmente, vestido
apenas com umas \emph{boxers }brancas, de toalha ao ombro.

Vasco passou a manhã a ler e a dormitar. As horas foram passando. «Mesmo
que ela demonstre interesse em mim», pensou Vasco, «será apenas porque
quer magoar o Júlio, pa- gar-lhe na mesma moeda. Estas coisas estão
constantemente a suceder.»

Vasco foi à cozinha para beber um copo de água. O calor era tanto que a
água saía morna da torneira. Ester apareceu, com a maquilhagem desfeita
e o aspecto de quem acaba de despertar de um sono pesado e pouco
repousante. Ester bebeu sumo de maçã directamente do pacote que retirara
do frigorí-

\textbf{128 }ALEXANDRE ANDRADE

fico. Sorriu para Vasco, um pouco como camaradas de armas com o moral
por terra costumam sorrir uns aos outros.

\begin{itemize}
\tightlist
\item
  Ainda restará alguma alma viva em Coimbra? Ou se- remos só nós os
  dois?
\end{itemize}

Ester contou a Vasco que fazia voluntariado na ala pediátrica dos
Hospitais Universitários. Levava livros e lia em voz alta para as
crianças internadas. Fazia-a sentir-se melhor na sua pele, e agora mais
do que nunca. Vasco julgou que Ester estava a referir-se à sua separação
recente.

Vasco aceitou acompanhar Ester ao hospital. Passaram um par de horas
muito agradável. Leram um conto de Eça de Queirós a um adolescente que
sofrera uma infecção urinária e a uma rapariga cuja doença não chegaram
a descobrir. Vasco e Ester faziam vozes diferentes para cada personagem.
Depois da leitura, deixaram-se ficar a vaguear pelas imediações do
hospital, aproveitando a relativa frescura que a tarde trouxera.

\begin{itemize}
\tightlist
\item
  Não se acredita naquilo que a passagem do tempo faz ao amor --- disse
  Ester com a voz carregada de amargura.
\item
  O tempo tem um poder desmedido --- disse Vasco, acenando vagamente com
  a mão.
\item
  Já soube que te aproximaste da Samanta. Acho ópti- mo. Não tenho nada
  que achar nem deixar de achar, mas fico feliz por ela. Ela e o Júlio
  pareciam apostados em deixar a relação cair de podre. Nestas coisas, é
  preciso agir, não deixar que o tempo nos imponha as suas leis
  terríveis.
\end{itemize}

Vasco não conseguia perceber se Ester estava a assumir, sem o dizer, que
Samanta não vira nele mais do que um degrau para escapar às tais leis
terríveis. Mas, e se fosse assim? Vasco nunca alimentara ilusões e
ninguém, a começar por Berenice, lhe fornecera qualquer pretexto para se
imaginar mais do que um figurante no percurso de Samanta, na direcção de
algo

VOI CHE SAPETE \textbf{129}

que o futuro lhe prometia, à medida da sua graciosidade, do seu apetite
pela existência.

E com que desprendimento falava Ester, como se não fosse parte
interessada nos delicados processos de interacção entre a cronologia e
os sobressaltos do coração! Claro que não poderia tardar a inflexão na
direcção dos seus traumatismos recentes.

\begin{itemize}
\tightlist
\item
  É assim, não me afastei da Minerva nem pusemos pon- to final em nada.
  Está tudo em aberto. Encontro-me com ou- tras raparigas, ela se quiser
  pode fazer o mesmo, é lá com ela.
\item
  Vivem a vossa vida, no fundo.
\item
  Exacto, é isso.
\end{itemize}

Regressaram a pé, dois peões solitários numa cidade a que o sol
declinante conferia um aspecto de fim de festa.

Vasco passou os dias seguintes entre a residência e as ruas limítrofes,
entre a vigília modorrenta e o sono frágil. Saía fre- quentemente com
Samanta, iam jantar, a bares e ao cinema. Davam-se bem e partilhavam
gostos e opiniões, e contudo Vasco sentia que a relação estagnava, não
evoluía. As conversas fluíam com desenvoltura, mas faziam
invariavelmente tangen- tes cheias de prudência aos assuntos mais
delicados e íntimos, como se um acordo não escrito, em vigor desde o
início, os impedisse de ir mais longe no conhecimento mútuo.

Vasco deu por si a tentar descobrir segundos sentidos em tudo aquilo que
Samanta dizia na presença de Júlio, por mais inócuas e neutras que
fossem as frases que ela dirigia ao ex-namorado. «Não se põe para trás
de costas, de um dia para o outro, uma história tão longa e intensa como
a que eles viveram», dizia Vasco para si nas horas de insónia, debruçado
à janela, aspirando com violência o ar estagnado da noite coimbrã.

Um dia, Ezequias veio bater à porta do quarto que Vasco

\textbf{130 }ALEXANDRE ANDRADE

ocupava. Sentou-se no chão, demorou-se, demasiado embara- çado para ir
directo ao assunto e sem ânimo para inventar um pretexto. Vasco
estranhou aquele comportamento, pois ele e Ezequias davam-se bem e não
havia cerimónias entre ambos.

\begin{itemize}
\tightlist
\item
  Queres dizer-me alguma coisa, Ezequias? Vá lá, sou todo ouvidos.
\item
  É assim, Vasco, queria pedir-te que me fizesses um fa-
\end{itemize}

VOI CHE SAPETE \textbf{131}

coisa que se deva fazer nesta fase da relação. Tentar ir ao fundo das
coisas é uma tentação terrível e quando se começa já não dá para voltar
atrás. Apercebi-me de que aquilo que nos unia e atraía era frágil, não
passava de um amontoado de coisas em comum que se esboroava assim que
lhe tocávamos, não tinha consistência. E sei que ela fez o mesmo
percurso. Olhávamos um para o outro e cada um via dúvida no olhar do
outro. As discussões começaram por causa disso, tenho a certeza, foi uma
espécie de fuga em frente, uma revolta contra a evidência. Vasco
deixou-o falar. Ezequias precisava de desabafar e Vasco sentia-se bem na
presença dele, não lhe desagradava es- cutar aquela voz arrastada e
desiludida que ressoava no corpo volumoso do amigo. Na semana seguinte
chegou a enco- menda. Era uma embalagem de cartão do tamanho de uma
caixa de sapatos. Vasco guardou-a no quarto até que, dias mais tarde,
Ezequias, com a expressão tensa de quem abomina a

duplicidade, a veio buscar.

Nos dias que se seguiram, Vasco interessou-se pelo caso de Luz e
Ezequias. Vasco simpatizara com Luz desde o início. Luz era a doçura em
pessoa, irradiava boa disposição e parecia incapaz de uma palavra de
rancor ou de um movimento de mau génio. Vê-la sofrer parecia, mais do
que uma circunstân- cia da vida, uma ofensa brutal ao mundo.

No entanto, Vasco hesitava em aproximar-se de Luz, por vários motivos.
Em primeiro lugar, as incertezas sobre o seu envolvimento com Samanta
faziam-no recear dar a impressão de intimidade acrescida entre ele e
Luz. Em segundo lugar, a última coisa que Vasco queria era parecer que
estava a tomar partido entre Luz e Ezequias e assim correr o risco de
perder a amizade de ambos. Por fim, a natureza da encomenda que Ezequias
lhe pedira para guardar continuava a intrigá-lo e a

\textbf{132 }ALEXANDRE ANDRADE

necessidade de manter aquele segredo iria constranger qualquer conversa
que tivesse com Luz.

Berenice, entretanto, parecia cada vez menos presente. Vasco apenas a
via de relance, ou em saídas comuns entre amigos. Berenice nunca deixava
de lhe dirigir um aceno amis- toso, ou uma piscadela de olho irónica.
Vasco atingira a fase em que poderia dispensar um cicerone e movia-se
com todo o à-vontade no meio dos ocupantes da residência, que o aceita-
vam como se fosse um dos seus. Os acenos de Berenice eram os de um
mestre que aprova os progressos do seu discípulo e que sabe no seu
íntimo que ele chegará longe.

Tatyana simpatizava com o dilema em que Vasco estava envolvido. «Não te
cabe a ti resolver a quadratura do círculo, deixa a Luz e os Ezequias
seguirem os seus percursos, o que tem que ser tem que ser.» O sotaque de
Tatyana era quase imperceptível, mas a pronúncia de algumas palavras em
por- tuguês sugeria as suas origens eslavas. Por exemplo, em vez de
``compromisso'' dizia ``campramisso''; a redução vocálica típica da
língua russa era tão difícil de contrariar como uma inclinação perversa.
Algumas subtilezas gramaticais, como o modo conjuntivo, também a iludiam
ainda.

\begin{itemize}
\tightlist
\item
  E se não for aquele o destino deles, Tatyana? E se foram feitos um
  para o outro? Estão a acabar tudo por causa de um punhado de
  ninharias. Ninharias, compreendes?
\item
  Vasco, até parece que não tens olhos na cara. Não vês que a Luz agora
  anda com o Klaus? Estão apaixonadíssimos, não se largam um minuto.
\item
  Achas? Não acredito. O Klaus dá-se bem com toda a gente, é só isso.
\item
  Acredita no que quiseres, mas basta ver a maneira como olham um para o
  outro.
\end{itemize}

VOI CHE SAPETE \textbf{133}

Vasco e Tatyana estavam a descascar batatas para o jantar dessa noite.
Tatyana tinha adquirido reputação como cozi- nheira exímia e era com
prazer que se encarregava, mais amiúde do que o exigiria o espírito
comunitário, da confecção das refeições. Os seus pratos de peixe, em
particular, nunca deixavam de suscitar uma aprovação geral quase sempre
rui- dosa. Vasco, que também se desembaraçava menos mal na cozinha, não
se fazia rogado para a ajudar nas noites em que não tinha planos com
Samanta. Vasco fitou Tatyana disfar- çadamente. Era então verdade que o
tempo operava milagres: pouco sobrava daquela criatura ensimesmada,
vergada ao peso do desgosto, que Berenice apresentara a Vasco. As dores
da ruptura tinham sido ultrapassadas e aparentemente subli- madas em
algo parecido com o cinismo. Vasco recusava-se a acreditar que Luz se
tivesse envolvido com Klaus tão pouco tempo depois do conflito com
Ezequias, mas viu-se forçado a admitir que essa recusa se devia mais a
uma ideia feita sobre a maneira de ser de Luz do que à abundância ou
ausência de provas nesse sentido.

A pouco e pouco, os diálogos que Vasco mantinha com Inácio evoluíram de
trocas de impressões breves e fortuitas para conversas que se
prolongavam, extravasavam das refeições ou eventos onde tinham começado
para noites de deriva por bares e clubes nocturnos. Para alguém, como
Inácio, que surgia com as roupagens e os modos do solitário, ele parecia
singularmente bem relacionado. Era raro o estabelecimento onde entravam
sem que alguém saudasse Inácio com o prazer genuíno do reencontro
desenhado nas feições.

Tudo em Inácio intrigava Vasco, mas talvez a questão para a qual ele se
sentia mais ávido de respostas fosse esta: a obses- são de Inácio por
construções mentais, pelo ideal em detri-

\textbf{134 }ALEXANDRE ANDRADE

mento do empírico, seria uma reacção a um episódio doloroso do seu
passado, ou uma estratégia pensada para atingir algo, uma iluminação
final, uma aproximação à essência, à medula frágil que se escondia por
detrás dos grosseiros revestimentos com que a humanidade se ocupava? O
discurso de Inácio era fluente e preciso; via-se que a troca de
argumentos lhe trans- mitia prazer. A algumas das suas ideias era
impossível deixar de reconhecer originalidade e riqueza, porém Vasco
nunca se deixava convencer totalmente. «Ele persegue dois objectivos
contraditórios: uma epifania sobre a natureza secreta do amor entre
humanos, por um lado, e a sua confirmação em carne e osso, por outro.
Quando o mecanismo incomensurável que ele visa estiver enfim perante os
seus olhos, resplandecente como a verdade, seguir-se-á a frustração
porque o mundo se tornará automaticamente coisa torpe e desadequada. O
seu destino é a paralisia.» E, contudo, Vasco não se conseguia privar de
procurar sentidos e reflexos da sua própria situação nas pala- vras de
Inácio. Vasco olhava Inácio nos olhos até que o adian- tado da hora os
obrigasse a regressar à rua, como que receoso de deixar escapar um eco,
um cambiante, uma sílaba que se aplicasse, com uma exactidão
surpreendente, àquilo que estava a viver com Samanta, com Ezequias e com
Luz, com Tatyana. Gradualmente, Vasco foi-se afastando de Inácio, que
certa- mente nunca se veria à míngua de discípulos e companheiros de
escapadelas nocturnas.

Samanta parecia distante. O espectáculo de Samanta a

empregar toda a sua inteligência e dissimulação para inventar uma
desculpa que a dispensasse de sair com Vasco deixava-o irritado e
perplexo. Estaria Samanta a esconder-lhe algo de mais profundo do que
uma simples flutuação de temperamen- to? Recusando-se a dar ouvidos ao
seu amor-próprio, Vasco

VOI CHE SAPETE \textbf{135}

começou a espiar os momentos de intimidade entre Júlio e Penélope. Cada
gesto abandonado de ternura era um sobres- salto de esperança sentido
por Vasco. Talvez a relação entre os dois estivesse, contra todas as
expectativas, a ganhar consistên- cia, a desabrochar.

\begin{itemize}
\tightlist
\item
  Tenho-te visto muitas vezes com o Inácio --- disse Tatyana. Era
  feriado, uma sexta-feira; a residência estava quase deserta.
\item
  É verdade --- disse Vasco. --- Acho uma certa graça ao seu estilo. No
  bom sentido.
\item
  Nunca simpatizei muito com ele. Ele esquiva-se às pes- soas.
\item
  Um bocadinho, é verdade.
\item
  E a Berenice?
\item
  Acho que foi passar o fim-de-semana a casa dos pais.
\end{itemize}

Porquê?

\begin{itemize}
\tightlist
\item
  Por nada. Pensei que soubesses.
\end{itemize}

Vasco adivinhou segundos sentidos nas palavras de Tatya- na, mas a
fadiga e o mau humor embotavam-lhe a curiosidade. Quando a relação entre
Klaus e Luz saiu da clandestini- dade, tão exuberante sob a claridade
ofuscante do domínio público como fora envergonhada e parcimoniosa
anterior- mente, Ezequias foi-se abaixo, entrou em depressão e quase
deixou de aparecer. Vasco era ainda um dos poucos com quem ele aceitava
falar. O episódio da encomenda subtraída à aten- ção de Luz fazia agora
parte do passado e Vasco já não era o cúmplice relutante, mas sim o
amigo menos preocupado com a história e as responsabilidades individuais
do que com a mis-

são de consolar e mitigar mágoas.

\begin{itemize}
\tightlist
\item
  Fui um imbecil, Vasco. Fui o príncipe dos imbecis, o prémio Nobel dos
  cretinos. Mereci perdê-la dez vezes.
\end{itemize}

\textbf{136 }ALEXANDRE ANDRADE

\begin{itemize}
\tightlist
\item
  Não digas isso. Sabes que isso não é verdade.
\end{itemize}

Mau grado as suas mais nobres disposições, Vasco passou a evitar a
presença de Luz e quando era obrigado a falar com ela as palavras
saíam-lhe breves e cortantes. Um sentimento difuso de injustiça
dominava-o e impedia-o de racionalizar a situação. O despeito era o
mesmo que normalmente se reserva aos traidores.

Vasco bateu, muito docemente e servindo-se apenas dos nós dos dedos, à
porta do quarto de Ester. Ester estava a ler uma revista.
Entreolharam-se, quase cederam ao riso em simultâ- neo e contiveram-se
em simultâneo. Havia muito tempo que não se viam, o sentimento de
estranheza tinha cambiantes de comicidade poderosos e inesperados. «Por
que diabo passei eu tanto tempo sem ver a Ester?» perguntou Vasco para
si.

\begin{itemize}
\tightlist
\item
  Estavas a ler?
\item
  Estava, mas não faz mal. Entra.
\item
  Olha, arranjei bilhetes para o concerto dos Foo Fight- ers. É amanhã.
  Queres vir comigo?
\item
  Estás a brincar? Que sorte, esse concerto está esgota- díssimo há
  semanas.
\item
  Podes vir?
\item
  Claro que posso.
\item
  Bem, na verdade não estou a ser completamente sin- cero contigo.
  Arranjei os bilhetes a pensar na Samanta, mas ela afinal não pode ir.
\item
  Vasco, Vasco...
\item
  Tenho a impressão de que ela vai estar lá. Com outra pessoa. É o meu
  medo, mas preciso de saber. E preferia não estar sozinho.
\item
  Vasco, não digas mais nada, é claro que vou contigo. Nos acessos ao
  Estádio Cidade de Coimbra vivia-se um
\end{itemize}

VOI CHE SAPETE \textbf{137}

ambiente efervescente digno da Queima das Fitas. Apesar do caos
aparente, a mole humana confluía para as entradas do estádio como se
seguisse um declive natural, cavado numa en- costa pela suave acção dos
elementos repetida ao longo dos milénios. Ester tremia de frio, Vasco
tentava abrigá-la com os braços, ele próprio surpreendido pelo vento
cortante e gelado, fora de estação. Gerou-se uma discussão mais viva a
alguns metros de distância. Foram trocados insultos, agressões físicas
em seguida. Ninguém prestou atenção à refrega; em breve a normalidade
era reposta. O estádio estava cheio. Vasco, com um copo de cerveja
esquecido na mão, prestava mais atenção às bancadas do que ao palco, e
nem o impacto poderoso da voz e da guitarra de Dave Grohl o distraiu.

\begin{itemize}
\tightlist
\item
  Acho que estou a ver a Samanta --- berrou Vasco para Ester ---, ali ao
  fundo, no topo sul. No meio daquele grupo, estás a ver?
\item
  Estás a brincar? A esta distância? Concentra-te na música!
\item
  É ela, tenho a certeza!
\item
  Precisavas de um lornhão, como na ópera!
\item
  De um quê?!
\end{itemize}

À saída, encontraram um colega de curso de Ester que lhes deu boleia até
à residência.

Acabaram a noite a beber leite quente, sentados na cozinha.

\begin{itemize}
\tightlist
\item
  Começo a pensar que já passei demasiado tempo aqui.
\end{itemize}

Está na altura de regressar.

\begin{itemize}
\tightlist
\item
  Mesmo que a Samanta se tenha afastado, pode ser que isso tenha sido o
  melhor para vocês os dois. Vasco, não vale a pena estares a procurar
  explicações muito complicadas. A Samanta estava vulnerável porque
  tinha acabado com o Júlio, queria sentir-se acarinhada e tu apareceste
  precisamente nessa
\end{itemize}

\textbf{138 }ALEXANDRE ANDRADE

altura. Vocês não tinham nada a ver um com o outro, qualquer pessoa via
isso.

\begin{itemize}
\tightlist
\item
  Não foi bem isso que se passou. A ideia original era eu facilitar a
  separação entre ela e o Júlio, servir de pretexto. Fiquei com a ideia
  de que seria uma situação em que todos ganhavam. Foi o que a Berenice
  me explicou. Isto faz algum sentido?
\item
  Mais ou menos. Estou a morrer de sono. Berenice...
\end{itemize}

Dias mais tarde, Vasco foi bater à porta do quarto de Tatyana.

\begin{itemize}
\tightlist
\item
  Tatyana, fui ao quarto da Berenice para ver se ela me podia emprestar
  uma caneta azul e ela não está lá. O quarto está vazio. A roupa e os
  livros dela desapareceram. Sabes onde é que ela está?
\end{itemize}

Tatyana pousou o livro que estava a ler e olhou para Vasco com um
sorriso triste nos lábios.

\begin{itemize}
\tightlist
\item
  Tu és mesmo um caso perdido, Vasco.
\item
  Não percebo.
\item
  Será mesmo possível que, ao longo deste tempo todo, não tenhas notado
  aquilo que estava à vista de toda a gente, que a Berenice sofria por
  ti, que ela te desejava, que só queria estar ao pé de ti? Só lhe
  faltou escrever isso numa \emph{t-shirt, }ou pagar um anúncio de
  página inteira no jornal.
\item
  A Berenice? Juro que não. Nunca imaginei. Não pode
\end{itemize}

VOI CHE SAPETE \textbf{139}

da gaveta cheia de cartas que ela te escreveu e nunca enviou. Ou das
noites em que fiquei a consolá-la até quase ao nascer do dia, a tentar
encontrar as palavras que interrompessem aquele fluxo constante de
lágrimas.

\begin{itemize}
\item
  \begin{itemize}
  \tightlist
  \item
    Onde está ela, Tatyana?
  \item
    Foi-se embora, foi para a terra dela, não a procures.
  \end{itemize}
\end{itemize}

Não te preocupes com a Berenice, ela cai sempre de pé.

As despedidas resumiram-se a apertos de mão e palavras fugazes trocadas
com aqueles que Vasco encontrou no corredor e nas escadas que davam para
a rua. Vasco tentou adivinhar a hora do dia, a fase do semestre escolar,
o dia da semana e a es- tação do ano através do movimento humano e
automóvel, da luz e do rumor quotidiano, de todos os minúsculos sinais
que se oferecem à atenção como pequenas erupções de significado. Vasco
cumpriu, a pé e de mãos vazias, o trajecto até ao prédio de cinco
andares que guardava a sua cama e os seus pertences, num compartimento
hostil e irrespirável. A velha fachada, no seu aprumo vertical,
trouxe-lhe um feixe confuso de reminis- cências e inquietações.

Subiu ao segundo andar, não se demorou no seu quarto mais do que o tempo
necessário para olhar em volta e assegu- rar-se de que nenhuma
catástrofe o devassara, não mais do que o tempo de uma expiração e de
uma inspiração.

A porta de Leda, dois pisos mais acima, estava aberta. Na antecâmara, o
sofá e a pilha de bandas desenhadas estavam no mesmo sítio onde os
deixara. Vasco empurrou a porta do quarto sem bater. Leda estava sentada
à secretária, a trabalhar. A sua silhueta serena e escura era como uma
resposta, vaga mas cheia de um bom senso intoxicante. Leda voltou-se,
sem surpresa, como quem coteja tranquilamente o acontecimento com a sua
probabilidade, mas também com prazer.

\textbf{140 }ALEXANDRE ANDRADE

Leda sentia-se cansada porque nesse dia trabalhara duran- te muitas
horas, sem interrupção. Leda e Vasco deitaram-se lado a lado, como
sempre tinham feito, antes e depois da ruptura: de lado, uma das pernas
ligeiramente flectida, os braços de Leda ao longo do corpo, os de Vasco
flectidos e do- brados sobre o peito, a face repousando sobre as mãos
unidas. Partilhavam o leitor de MP3, um auricular no ouvido di- reito de
Vasco, o outro no ouvido esquerdo de Leda. Estavam a escutar a canção
``Music for Evenings'', dos Young Marble

Giants.

\begin{itemize}
\tightlist
\item
  Leda, sentiste a minha falta?
\item
  Tudo continuou como dantes, mas não era a mesma coisa. A forma do teu
  corpo permaneceu no sofá. Não deixei que mais ninguém se sentasse lá.
  Aquele côncavo eras tu.
\item
  Como fazias quando querias consultar uma referência para a tua tese?
  Entraste no meu quarto? Trouxeste livros cá para cima?
\item
  Já sabes que só saio daqui quando uma necessidade absoluta me força a
  isso. O que eu fiz foi simples: de cada vez que tinha uma dúvida,
  anotava-a num quadrado de papel que depois guardava nesta lata de
  rebuçados. Como vês, está quase cheia. Não vais ter mãos a medir nos
  próximos dias.
\item
  Mas conseguiste fazer progressos no manuscrito?
\item
  Digamos que a minha tese derivou para conjecturas e versões
  alternativas, ao sabor das incógnitas que iam surgindo e que a tua
  ausência me impedia de esclarecer. Mas está bem assim. Tem uma certa
  graça; é libertador. Seguir à letra o cur- so da História não é tudo o
  que existe na vida.
\item
  Está muita gente a viver na residência, neste momento?
\item
  Nem queiras saber. Há novos hóspedes que trazem uma corte de amigos e
  conhecidos, há animação de manhã à
\end{itemize}

VOI CHE SAPETE \textbf{141}

noite, descem as escadas com estrondo e tropel, gritam como se o timbre
e o volume das suas vozes fosse à medida do seu apetite pela vida ---
que, neste caso, parece ser dos mais vora- zes e selvagens.

NOVEMBRO 2011

\subsection{O RAMO DOURADO 2012}

\textbf{-- UMA NOVA ESPERANÇA}

\emph{«I am less the jolly one there, and more the silent one with sweat
on my twitching lips»}

Walt Whitman

homem que eu amo fez-me uma surpresa: escreveu, de uma ponta à outra, a
obra-prima de Sir James Frazer,

\emph{The Golden Bough, }e não se esqueceu de ma dedicar, com aquela
caligrafia rica em curvas e adornos opulentos que em certas ocasiões
tanto me exaspera. Encontrei o manuscrito ao regressar a casa, num fim
de tarde que nada parecia ser capaz de resgatar a uma banalidade escura
e chuvosa. O homem que eu amo deixara-o sobre a mesa da cozinha, não a
de imi- tação de pinho ao centro, mas sim a de fórmica branca de abas
rebatíveis, encostada à parede. Incrédula, deixei cair a cor-
respondência que trazia (factura da France Télécom e postal ilustrado de
Zurique), assim como um \emph{Nouvel Observateur }já disforme por causa
do manuseamento e da chuva. A espes- sura do volume, o solene
alinhamento das folhas A4, fizeram-

-me temer algo de funesto, ignoro porquê. Foi a muito custo de tempo e
energia que consegui vencer o meu cepticismo e aceitar que se tratava de
um presente, o mais inacreditável e insensato presente que homem amado
alguma vez ofereceu a mulher amante.

Enquanto esperava que ele chegasse, sentada na sala à

\textbf{144 }ALEXANDRE ANDRADE

mercê do lusco-fusco e com o manuscrito ao colo como se fosse um animal
de companhia, eu pensava nele e na sua no- breza de espírito. Que
atencioso da sua parte: escrever, palavra após palavra, frase sóbria
após frase sóbria, esse grande e in- fluente clássico da literatura
ocidental. Poderia ter escolhido outras obras, obras mais reputadas e
mais canónicas, obras que protagonizaram rupturas, obras que
sobreviveram de ma- neira mais feliz ao desfile das décadas. Mas só
esta, pela sua proporção desmedida e temerária, pelo escrúpulo com que
foi redigida, fruto de um trabalho de quase inverosímil minúcia e
erudição, me traria uma satisfação tão profunda. «Limitei-me a
escrevê-lo, uma palavra a seguir à outra» disse-me ele, mais tarde, já
presente perante mim, voltando para mim o seu ros- to, o rosto
fulgurante a que o excesso de beleza parece empres- tar uma radiância de
desplante e escárnio benigno, esse rosto surgindo da penumbra do
corredor, fingindo surpresa perante a minha surpresa. O nosso
apartamento fica situado na Rue de la Tombe-Issoire, no \emph{14ème
arrondissement. }Da janela vê-se o viaduto do comboio. O pão da padaria
da nossa rua possui um sabor e uma consistência que nenhuma outra
padaria de Paris iguala.

Nunca entro em casa sem recordar a primeira vez em que transpus este
limiar, no final de uma tarde escura e ventosa de Abril. Entrei atrás do
cavalheiro da imobiliária, criatura de humor sombrio e de silhueta tão
ampla que fazia parecer ainda mais pequeno aquele apartamento T2, este
onde vive- mos hoje. Depois de, nesse mesmo dia, ele nos ter mostrado
outras três casas, todas por ele elogiadas com o desdém negli- gente de
quem sabe estar a arremessar pérolas a suínos, aquela última visita
soava a formalidade para cumprir calendário. Talvez fosse por podermos
explorar o espaço sem o fundo

O RAMO DOURADO 2012 - UMA NOVA ESPERANÇA \textbf{145}

sonoro da sua lábia profissional, talvez fosse da fadiga ou de qualquer
imponderável ou detalhe subliminar, o que é certo é que a certeza de que
ia ser aquela a nossa casa tombou sobre nós com o ímpeto das línguas de
fogo dos apóstolos.

O nome da rua intrigou-nos, mas foi só ao fim de duas ou três semanas,
passadas as azáfamas das limpezas, da mudança e das arrumações, que me
lembrei de abrir o meu exemplar, já tão gasto pelas consultas repetidas,
da \emph{Histoire et Mémoire du Nom des Rues de Paris }(Alfred Fierro,
Parigramme). O que pode querer dizer ``Tombe-Issoire''? A única certeza
é a de se tratar de um topónimo muito antigo. A lenda (segundo Fi- erro)
fala de um gigante chamado ``Isoret'' ou ``Isaure'', cujo túmulo estaria
situado nas redondezas. Teorias mais realis- tas apelam a formas
verbais, caídas em desuso, derivadas de ``tombiseur'' (originalmente,
``o falcão que faz cair a ave caça- da''): uma ``tombisoire'' seria
assim, muito simplesmente, um pequeno acidente de terreno susceptível de
provocar a queda, o que é coerente com o costume de baptizar colinas ou
ou- teiros com nomes retirados ao campo semântico da queda ou do acto de
tropeçar. Durante anos, sem nunca me ter dado à canseira de investigar,
eu associara o nome da rua ao cemitério de Montparnasse, que afinal de
contas nem fica tão próximo quanto eu supusera. O meu conhecimento
daqueles bairros, Montparnasse, Denfert-Rochereau, Alésia, fundara-se
até aí em incursões esporádicas, histórias alheias e impressões lite-
rárias. Recordava-me nitidamente, por exemplo, da tarde em que
(sozinha), para matar o tempo que me separava de uma sessão no cinema
``Les 7 Montparnasse'', eu entrara no cemité- rio e fora recompensada
com a descoberta, rigorosamente aci- dental, da sepultura de Samuel
Beckett.

A tarefa de fazer deste espaço o nosso espaço revelou-se

\textbf{146 }ALEXANDRE ANDRADE

doce e dura. Nos primeiros dias, as paredes reflectiam tudo e não
retinham nada. Agarrámo-nos a tudo aquilo que alcan- çávamos: sons,
cheiros, farrapos de informação sobre os inqui- linos anteriores,
texturas, peculiaridades (o gancho de metal no tecto, a saqueta de chá
deixada no armário da cozinha, o inverosímil tom de cor-de-rosa dos
rodapés, o saco de plástico de um hotel termal de luxo de Biarritz
encontrado no roupeiro). E eis que de repente: os \emph{nossos }sons, os
\emph{nossos }cheiros, \emph{nosso }gancho, rodapé, mesa, azulejo,
mancha, suavidade, gorgolejar,

raio de sol.

\emph{The Golden Bough }ocupa doze extensos volumes. Trata-se de um
estudo sobre a evolução das crenças e dos ritos huma- nos que exerceu
considerável influência sobre a literatura inglesa da altura e suscitou
a admiração de D.H. Lawrence,

T.S. Eliot e Ezra Pound, entre outros. Neste livro, é defen- dida a tese
de que todas as crenças religiosas evoluíram a partir de rituais mágicos
inseridos em cultos de fertilidade. A matriz explicativa delineada pelo
autor integra elementos que seriam comuns a todas as proto-religiões
conhecidas, como por exemplo um deus-rei sujeito a um sacrifício
periódico e o matrimónio entre uma divindade solar e uma deusa da Terra
que morreria após as colheitas e ressuscitaria na Primavera. Esta
filiação da religião no mito esbarrou com uma oposição forte e
acrimoniosa na altura da publicação do livro. A versão que o homem por
mim amado escreveu é a versão longa ori- ginal, não expurgada da secção
sobre a crucifixão de Cristo, que seria relegada para a condição de
apêndice numa edição posterior e em seguida suprimida. Ainda que muitas
das ideias nela defendidas estejam hoje obsoletas, ainda que o seu
interesse para as gerações mais recentes de antropólogos seja meramente
histórico, é incontestável que se trata de uma

O RAMO DOURADO 2012 - UMA NOVA ESPERANÇA \textbf{147}

obra cuja dimensão intelectual é assombrosa. Basta folhear as suas
páginas ao acaso para perceber que a pesquisa implicada pela sua
elaboração foi invulgarmente profunda. Assim se ex- plicam aquelas
longas ausências que povoavam os meus dias de dúvidas cruéis. Enquanto
eu cismava, enquanto os meus passos ecoavam nas paredes das nossas
assoalhadas, ele cal- correava a cidade em busca de fontes, consultava
volumes ignorados durante décadas, desfazia com a firmeza impaciente dos
seus dedos camadas de pó de biblioteca. Para tudo aquilo (e era muito)
que não conseguia saber através de livros, con- tava com a sua rede de
correspondentes, na maior parte dos casos missionários colocados nos
confins do mundo. Tal como Frazer, o homem que eu amo só muito raramente
se deslocou para trabalhar no terreno e preferiu confiar em relatos
escritos, por vezes inexactos ou coloridos pela fantasia, é certo, mas
sempre informados por testemunhos em primeira mão. Pas- savam amiúde
pelas minhas mãos sobrescritos com selos de nações exóticas: Bornéu,
Pérsia, Guatemala, África Ocidental Francesa...

Terão algum vestígio de razão aqueles que afirmam que a solidão possui
virtudes, que é moralmente fecunda ou pelo menos que tonifica o
carácter? Durante os meus primeiros anos em Paris, transplantada de uma
remota província alemã, primeiro com os meus pais e depois sozinha, fiz
por acreditar que sim. Agia como se a minha felicidade fosse assunto
exclu- sivo da minha pessoa, da bolha de ar em meu redor e daqueles
poucos que o destino e os dias atraíam para a minha órbita, efémera e
contrafeita. Viver era uma espécie de artesanato que não exigia mais do
que solidez e rotinas. Foi preciso esperar pelo correr dos anos, pela
sucessão de quartos e casas alugadas (apartamento insalubre do
\emph{10ème arrondissement, }rés-do-chão

\textbf{148 }ALEXANDRE ANDRADE

numa ruela sossegada do Pré Saint-Gervais, quarto perto da Porte de
Versailles, estúdio nos Gobelins), pela acumulação de desgostos e
esperanças mudas. Foi preciso esperar por es- tabilidade profissional e
pelo homem que me ama, que eu amo, que escreveu para mim um livro
sumptuoso e datado, influente mas esquecido por todos. A pouco e pouco,
apercebi-

-me de que a única atitude válida era encarar Paris (que outro- ra me
intimidava, complexa e enorme) como uma segregação do engenho humano,
com as suas leis, mistérios e debilidades. Tudo dependia de tudo: o
principal do acessório, a grande história da pequena história, o gesto
quotidiano minúsculo da tendência sociológica. Habituei-me a coleccionar
factos, a interessar-me pelo trivial, a fazer perguntas, a não deixar
pontas soltas. Criei e perdi arquivos pessoais, amizades, livros de
contactos. Descodificar a imensa máquina Paris, ou pelo menos
descrevê-la por extenso, passou a ser a minha missão, condição
necessária para o meu direito a estar aqui. (Essa mis- são ainda dura.)

Todas as relações passam por momentos mais delicados. Agora que me sinto
como se navegasse num rio subitamente largo e tranquilo, recordo todas
as vezes em que ele trazia traba- lho do escritório (julgava eu), fazia
serões intermináveis apesar de saber que teria de comparecer muito cedo
no emprego na manhã seguinte. Na minha mesa, voltada para a parede, eu
quase teria preferido que ele escarnecesse dos meus esforços de
desenhadora autodidacta, da insignificância dos meus rabiscos sem graça,
dos meus bonecos disformes, das folhas de bom papel amarrotadas que
enchiam o cesto. O seu silêncio du- rava horas; eu imaginava-o a
preparar apresentações, a estudar \emph{dossiers, }a polir as frases com
que iria acolher um cliente no dia seguinte, ocupado com as preocupações
sólidas e amplas

O RAMO DOURADO 2012 - UMA NOVA ESPERANÇA \textbf{149}

de um profissional das relações públicas. Só agora percebo que o que o
movia não era senão um empenho absoluto num de- sígnio de proporções
épicas. E tudo isso por mim! Por mim!

Passaram-se já alguns dias desde o dia em que o homem que eu amo me
ofereceu a sua obra revolucionária, acabada de escrever. Como sempre
sucede quando lhe peço para fazer compras para o jantar, o homem que eu
amo passou pelo supermercado Picard e entra agora em casa com um saco
iso- térmico em cada mão. Enquanto eu abro espaço no conge- lador, ele
conta-me que chegou ao supermercado pouco depois da hora de fechar, que
foi obrigado a implorar e bajular para poder fazer as suas compras. Não
o acho capaz de implorar seja a quem for, mas consigo visualizar a cena.
Quando ele assim o quer, sabe ser tão persuasivo que nem a mais coriá-
cea má-vontade resiste por muito tempo. Gotas minúsculas de água da
chuva persistem nos seus cabelos louros em desalinho. Hesito um momento
antes de lhe perguntar se não o apo- quenta a perspectiva de o seu livro
vir a ser considerado mera curiosidade, de valor apenas residual para a
disciplina nascente que é a antropologia, suplantado pelas teorias
estruturalistas de Claude Lévi-Strauss e dos seus seguidores. Ele sorri
e pisca-

-me o olho enquanto diz que não, sacudindo muito a cabeça como um
rapazinho. A humidade dos seus cabelos impregna o ar, até aí neutro e
sem direito a menção. Esqueceu-se de comprar o miolo de castanhas, é
evidente que ele sabe isso, mas prefiro, mesmo sem saber porquê, fingir
que não dei pelo esquecimento. Como seria mesquinho da minha parte
fazer-

-lhe um reparo, dadas as circunstâncias. Tudo mudou; tudo deve mudar.

A dedicatória do livro diz isto: «Com todo o meu afecto, para ti, por
tudo o que passámos juntos». Todo o meu afecto.

\textbf{150 }ALEXANDRE ANDRADE

A totalidade do meu afecto existente. Numa cadeira de cine- ma, uma vez,
murmurei o meu amor ao seu ouvido. Fiz-me sensível e receptiva a
intensos e silenciosos significados do seu beijo firme no meu pescoço. O
tempo depois disso.

O homem que eu amo deixa-me perplexa. O que poderá levar alguém a
consumir os seus melhores anos na redacção de uma obra que é justamente
considerada um dos momentos precursores da antropologia moderna, e que,
pela primeira vez, integrou num corpo teórico coerente toda a linha
histórica que tem origem nas práticas mágicas primitivas e que conduz
até ao desabrochar da mentalidade científica contemporânea? Isto tem
tanto de grandioso como de doentio. Haveria tantas outras maneiras de
ele exprimir o que sente por mim. Julgará ele que me desagrada aquilo
que é corriqueiro ou consagrado pelo uso? Sentirá ele o impulso de
procurar a desmesura para exprimir um sentimento que é, no fundo, tudo o
que há de mais banal? Mas rapidamente desisto de o tentar adivinhar. Não
foi por ele ser transparente e legível como um letreiro no metropolitano
que me apaixonei por ele. Foi num Inverno, menos sombrio do que este,
menos abundante em borrascas. Os edifícios de Paris pareciam-me então
mais nítidos e recortados de encontro ao céu do que agora. Quinto andar
sem elevador. Perdeu-se gato cinzento muito meigo, no dia 22, recompensa
a quem o encontrar. Reunião da assembleia municipal, o vizinho da
flauta, o vizinho da guitarra eléctrica, o vizinho da tosse incessante.

Cheguei até a segui-lo. Cheguei a segui-lo pelas ruas de Paris como num
mau filme policial. Isto durou muito tempo, meses ou anos. Disfarcei-me,
perdi horas em cafés, em jar- dins públicos, em galerias comerciais. Ele
encontrava-se com desconhecidos, eu seguia os desconhecidos. Assim
acontece-

O RAMO DOURADO 2012 - UMA NOVA ESPERANÇA \textbf{151}

ram pequenas aventuras, incursões em bairros e subúrbios cuja existência
eu ignorava e aonde não creio alguma vez vir a regressar. Tantas
dúvidas, tantos receios, afinal para nada. São duas horas da manhã. Não
sei se ele virá dormir a casa esta noite. Folheio o livro no capítulo
20, secção 2. As pontas dos meus dedos tremem de afecto. Não sei se ele
está à espera que eu o leia. É muito extenso. O livro está escrito em
inglês, eu não domino o inglês. O meu tempo livre é abundante, mas
também consumido por milhares de sumidouros. De noite, sentada na cama,
entretenho-me simplesmente com alguns dos inúmeros relatos fascinantes
de que o livro está repleto, por exemplo os tabus associados ao luto.
Decifro a língua inglesa com paciência e afinco. Na Polinésia, é comum
as pessoas que estejam em contacto com os mortos serem interditadas de
tocarem em artigos alimentares. Na Samoa, por exemplo, aqueles que
cuidaram de um cadáver abstêm-se de manusear comida e são alimentados
por outros como se fossem crianças. Em Tonga, tocar num chefe morto
implica um tabu de dez meses lunares. Se quem praticar o acto for, ele
próprio, um chefe, o período de tabu é reduzido para três, quatro ou
cinco meses, de acordo com o grau de superioridade do chefe mor- to.
Durante esse tempo, a pessoa sujeita ao tabu não se pode alimentar pelas
próprias mãos. Se, porventura, sentir necessi- dade de comer e não
existir ninguém nas proximidades para o ajudar, deve deslocar-se como um
quadrúpede e abocanhar os alimentos sem lhes tocar com as mãos. Já entre
os índios Shuswap da Colúmbia Britânica, os próprios viúvos e viúvas são
confinados e proibidos de tocar no próprio corpo durante o luto.

Mas sinto-me longe da essência da obra, do seu significado

supremo, que pressinto, inacessível, diante de mim.

\textbf{152 }ALEXANDRE ANDRADE

A renda da casa é paga no dia um de cada mês, por transferência
automática da nossa conta do Crédit Lyonnais. É mais prático assim.
Preenchemos a nossa declaração de im- postos a meias. Foi mais
complicado da primeira vez. Eu sou trabalhadora independente, ele é
trabalhador dependente. Tentamos distribuir as despesas ao longo do ano
para evitar surpresas no orçamento, mas por vezes há imprevistos. As
canalizações são antigas, já tivemos problemas de fugas duas vezes em
três meses. A instalação eléctrica também não é fiá- vel. A potência
contratada é fraca, não podemos passar a ferro e ter a máquina de lavar
a trabalhar ao mesmo tempo. E há ainda as baratas. Há baratas de dois
tamanhos, não sei se per- tencem a espécies diferentes. O técnico de
desinfestação visitou o prédio há uma semana e meia. Queixou-se de que
muitos habitantes não estavam em casa para lhe abrir a porta e que assim
o tratamento é menos eficaz. A recolha selectiva do lixo já entrou nos
nossos hábitos, mas no início cometemos erros sucessivos que nos valeram
a censura do porteiro.

Quando chega o mês de Março, com a ligeireza de peque- nas aves,
evitamos as poças de água por entre as fachadas da Rue
Notre-Dame-des-Champs.

NOVEMBRO 2006 E FEVEREIRO 2012

\subsection{RUA DA VELHA LANTERNA}

aris como um imenso porto de mar: muitos o imaginaram (porque a imagem é
poderosa, brutal mas complexa), só eu desembarquei nesse cenáriosombrioe
húmidode ruas tortuosas, estivadores e fachadas corroídas pelo ar
marinho, poças de

lodo.

Viajei num dos numerosos voos que ligam Lisboa a Paris. Da janela,
contemplei com um nó na garganta (de antecipação, mas parecia de
saudade) aquela imensidão de água oceânica, subitamente interrompida
pela linha dos cais em formato de meia-lua: Quai du Pont du Jour, Quai
Louis Blériot, Avenue du Président Kennedy, Avenue de New York, Quai des
Tuileries, Quai du Louvre, Quai de la Mégisserie, Quai de Gesvres, Quai
de l'Hôtel de Ville, Quai des Célestins, Quai Henri IV, Quai de la
Rapée, Quai de Bercy. Desembarquei no aeroporto Charles de Gaulle,
cumpri o trajecto de comboio até à cidade com o desinteresse enervado de
quem cumpre uma formalidade.

Desci em Châtelet. Fundir-me na multidão foi a mais simples das tarefas.
Alguns olhares fugidios, que eu interpretei como hostis, não fizeram
senão (estranhamente) confortar-me

\textbf{154 }ALEXANDRE ANDRADE

na sensação de pertencer, de estar no lugar certo. Olhariam aquelas
pessoas de maneira diferente para mim (talvez com mais interesse, outra
demora) se conhecessem a minha história, o que me trouxera a Paris, em
que consistia a minha missão? Certamente que não. A minha seria apenas
mais uma entre as milhares de histórias, nem banais nem bizarras, que
formavam o tecido humano da cidade. E eu não aspirava à notoriedade. O
caudal humano aconchegava-me como uma peça de roupa. O anonimato
convinha-me.

``A minha missão''\ldots{} A minha missão consistia para já em poucas
frases, rabiscadas numa folha de papel dobrada em quatro, com mão
decidida mas impaciente:

\begin{itemize}
\tightlist
\item
  Encontrar um quarto para viver.
\item
  Comprar óculos escuros.
\item
  Comprar bengala de cego.
\item
  Comprar caixa de lápis.
\item
  Comprar bloco de papel.
\item
  Esperar pelo dia 26 de Janeiro.
\end{itemize}

Encontrar um quarto, pois.

Encontrar um quarto foi muito mais fácil do que eu receava.

Tentei a minha sorte no bairro de Saint-Paul, entre a Place des Vosges e
aquilo que seria a ilha de Saint-Louis se existisse ilha,
aproximadamente entre o prolongamento das pontes Marie e Louis-Philippe
(inexistentes). O ar marítimo e gélido cobria as fachadas com uma
finíssima película de água cujo gosto a sal comprovei na minha língua,
infiltrava-se através de várias camadas de cartazes colados nas paredes
e nos painéis apropriados da câmara, diluía em manchas disformes as
letras que dantes anunciavam filmes, provas de ciclismo, candidaturas

RUA DA VELHA LANTERNA \textbf{155}

a eleições municipais. Os corpos largos dos estivadores em trânsito
ocupavam as ruas como num país conquistado, troca- vam comentários
grosseiros à distância, riam um riso pujante e juvenil. Marujos ociosos,
em pequenos grupos, exploravam o bairro, a medo mas como quem antecipa
prazeres inaudi- tos. Mulheres maduras ou jovens, quem sabe se esposas
de pescadores embarcados há semanas, espreitavam por detrás de
cortinados incolores, um só olho fixo, escuro, desafiante.

Entrei em lojas e em cafés, apurei os ouvidos, perguntei. Segui as
direcções rabiscadas numa margem de jornal por um merceeiro de feitio
agreste, bati a uma porta, troquei frases breves com uma porteira, subi
todos os lanços de escada (só mais tarde me dei ao trabalho de os
contar) que me separavam do piso onde, no fundo de um corredor sem luz,
empurrei uma porta e entrei num quarto pequeno mas limpo e lumi- noso.
Dois ou três passos no seu interior chegaram para me convencer de que
estaria tão bem ou tão mal ali como em qualquer outro lugar. O preço era
módico. A porteira quis saber alguma coisa sobre mim; falei-lhe de
estudos, forneci detalhes, mas estava a esforçar escusadamente a minha
fanta- sia. Os três meses de renda que paguei a pronto eram tudo o que
ela precisava de saber sobre mim e sobre os motivos que me traziam à
grande metrópole. O senhorio, pelos vistos confiava cegamente nas
capacidades de fisionomista desta mulher, cujo olhar astuto fora sem
dúvida aguçado por gerações de inqui- linos mais ou menos viciosos.

Perante a modéstia do quarto, compreendia-se que o critério de admissão
fosse frouxo: qualquer indivíduo acima da categoria de destroço humano
poderia sem dúvida ter feito um percurso idêntico ao meu e ver-lhe
confiada a chave que eu agora sopesava na minha mão. A cama pouco mais
era do

\textbf{156 }ALEXANDRE ANDRADE

que um catre; o mobiliário era espartano; havia um pequeno lavatório com
água corrente, mas a retrete era no corredor. Não era o luxo que eu
tinha vindo procurar. Só a missão im- portava. Risquei a frase
``Encontrar um quarto para viver'' da folha de papel e preguei-a à
parede. Deitei-me todo vestido.

Acordei várias vezes durante a noite. De noite, só um ruído violento ou
súbito é capaz de me acordar. De uma das vezes, pareceu-me ouvir passos
pesados algures no prédio, talvez de duas pessoas absorvidas numa
conversa. Já ao romper do dia, escutei música de realejo ao acordar,
sobreposta a um riso inter- minável, um riso incrédulo que continha
fúria e pasmo.

No dia seguinte, o meu primeiro dia completo em Paris, não resisti a
desviar-me do meu programa por uns minutos para saciar a minha vontade
de me aproximar tanto quanto fosse possível dos navios atracados, dos
cargueiros, dos paque- tes, dos cruzeiros altos como palácios. O porto
fervilhava de gente e de movimento. Perguntei-me qual seria a sensação
de avistar Paris no horizonte depois de semanas ou meses de alto-mar. O
que iria na cabeça de um marinheiro de licença, ao mergulhar naquela
colmeia humana demasiado complexa para ser descrita (quanto mais
compreendida), depois de tanto tempo entregue à solidão dos oceanos? Não
me era permi- tido senão especular. Quando dei por mim, já tinha passado
o cais de Bercy. Recriminei-me pela minha falta de cuidado, arrepiei
caminho, escolhi um itinerário mais interior para evi- tar distracções:
Bois de Vincennes, Faubourg Saint-Antoine, Bastilha.

Comecei a entrar em todos os estabelecimentos comerciais que encontrava,
rua a rua, começando pelo lado dos números pares, atravessando para o
lado dos ímpares quando chegava ao fim. Podia dar-me ao luxo de confiar
no acaso. O dia que

RUA DA VELHA LANTERNA \textbf{157}

passava era o dia oito de Janeiro. Faltavam mais de duas sema- nas para
o dia 26 de Janeiro.

Encontrei o bloco de papel sem dificuldades: marca ``Clairefontaine'',
formato A4, capa dura, 100 gramas por metro quadrado, papel liso,
apropriado para desenho.

A caixa de lápis exigiu mais persistência.

Os lápis que eu encontrava à venda eram ou demasiado moles ou demasiado
duros, ou a secção era redonda em vez de ser hexagonal, ou a mina era
demasiado fina e quebradiça, ou então era o meu pulso que estranhava a
fricção com o papel quando os experimentava. Mas a persistência trouxe
os seus frutos. Paguei a caixa de lápis de mina preta com uma única
moeda que depositei na mão do vendedor, sem evitar nem procurar o
contacto visual.

Encontrei os óculos escuros que desejava numa miserável loja de
recordações, cuja sobrevivência no ramo parecia um milagre, situada como
estava numa travessa tão remota e pou- co frequentada. As lentes eram
largas e pareciam opacas a um observador exterior; a armação moldava-se
à minha cara como se feita à medida.

Faltava a bengala de cego.

A escolha mais óbvia teria sido o instituto para cegos da Rue
Gay-Lussac. Seria sem dúvida aí que teria feito a minha primeira
tentativa, se a Rue Gay-Lussac e toda a margem esquerda, toda a história
associada, séculos de paixão e catás- trofes, não estivessem submersos
para sempre.

Valeu-me a sorte, não a dos audazes, mas sim a daqueles que andam à
deriva, diluindo os seus propósitos no caudal do tempo. Numa espécie de
bazar que acumulava no seu inte- rior exíguo uma variedade assombrosa de
bricabraque deparei com uma bengala branca de dobrar, tão parecida com
aquela

\textbf{158 }ALEXANDRE ANDRADE

que eu imaginara que outro, no meu lugar, não deixaria de se comover.
Estava esquecida numa prateleira, talvez há anos, rodeada de
quinquilharia. Rodeei-a firmemente com a mão, senti-a fria e lisa, quase
minha, sólida. Cabia no bolso do meu sobretudo.

Anoitecia. As horas tinham passado sem que eu desse por isso. Sentia
fome e frio. Devorei uma ceia sem sabor num res- taurante económico,
recolhi ao meu quarto. Ao meter a chave na porta, apercebi-me de uma
presença ao fundo do corredor: um jovem, de saída do seu quarto, sem
dúvida tão pequeno e despido como o meu. Trocámos uma saudação
brevíssima. O jovem, provavelmente um estudante, vestia um fato justo em
que, mesmo naquela penumbra e à distância, se adivi- nhavam anos de
remendos e ajustes. Não consegui afastar a impressão de que tinham sido
dele os passos que eu escutara na noite anterior.

Risquei da minha lista:

\begin{itemize}
\tightlist
\item
  \sout{Comprar óculos escuros}.
\item
  \sout{Comprar bengala de cego}.
\item
  \sout{Comprar caixa de lápis}.
\item
  \sout{Comprar bloco de papel}.
\item
  Esperar pelo dia 26 de Janeiro.
\end{itemize}

Faltavam ainda mais de duas semanas, que seriam sem dúvida escuras e
agrestes, para o dia 26 de Janeiro. Deitei-me e dormi, desta vez um sono
profundo e íntegro.

Deixar que o tempo passe é a tarefa menos complicada do mundo.
Espanto-me como tantos se deixam amedrontar por ela.

Basta acordar dia após dia, cultivar pequenas rotinas,

RUA DA VELHA LANTERNA \textbf{159}

fazer-se pequeno e transparente. Não evitar cruzar-me com os outros. Ser
delicado.

Cozinhava as minhas refeições modestas numa placa eléc- trica que
parecia um brinquedo. Abastecia-me nas mercearias e padarias do bairro,
alternava os estabelecimentos consoante os dias, evitava fidelidades,
fugia ao estatuto de ``cliente habi- tual''. Por vezes, rondava a lota
até conseguir apoderar-me de um ou dois peixes caídos no chão. Não eram
poucos os que me imitavam, sem pressa nem pudor. O peixe era
fresquíssimo e transportava o odor potente do oceano.

Informei-me sobre o programa do ftéâtre de la Ville, Place du Châtelet.
Na noite de 25 de Janeiro, véspera do dia que eu aguardava, era exibida
uma coreografia de Marie Chouinard, \emph{Le Nombre d'Or. }Comprei um
bilhete num lugar o mais perto do palco que fosse possível. Na noite do
espectáculo, dirigi-me ao teatro a pé debaixo de uma chuva misturada com
neve, sem nunca acelerar o passo.

Gostei do espectáculo. Não conhecia o trabalho desta coreógrafa. A
competência dos bailarinos era irrepreensível. Algumas opções da
encenação pareceram-me discutíveis. Aqui e ali, movimentos que se teriam
querido ousados e fluidos sur- giam demasiado conscientes de si mesmo,
como se o risco cor- rido fosse fruto de uma ponderação de ganhos e
perdas em vez da genuína vontade de arriscar.

Não poucas vezes, dei por mim a abstrair-me da acção e a deixar o meu
olhar derivar por uma zona indistinta, vaga, sem fronteiras, algures
entre o palco e a primeira fila repleta de (não duvido) figuras gradas
da sociedade e do mundo artístico.

Diz a lenda que foi no local onde, mais tarde, se situou o cubículo do
\emph{souffleur }do ftéatre de la Ville que o escritor Gérard de Nerval
se suicidou, por enforcamento.

\textbf{160 }ALEXANDRE ANDRADE

Foi na madrugada cruelmente fria do dia 26 de Janeiro de 1855, na Rue de
la Vieille Lanterne. Dir-se-ia que a es- colha desta rua, entretanto
desaparecida, fora motivada por um escrúpulo de discrição, pois era uma
das mais escondidas e sombrias do bairro.

No dia seguinte ao da minha noite no teatro, 26 de Ja- neiro, muni-me da
minha caixa de lápis e do bloco e regressei ao local. O dia estava como
eu o desejara: limpo e luminoso. Na Place du Châtelet, de pé, sem nunca
sentir fadiga, desenhei aquilo que via à minha frente. O meu pulso
estava firme, os meus olhos atentos, o ruído do tráfego e das pessoas
não exis- tia. Desenhei sem inventar, sem criar. Desenhei o que existia
ali, naquele momento. Fui consciencioso mas não me dissipei em detalhes
tolos. Isto levou horas. Os dias de Janeiro são curtos em Paris. O sol
deitava-se para as bandas dos Champs-

-Elysées. O meu desenho estava acabado.

Passava da hora de encerramento da biblioteca do Hôtel de Ville, rue
Lobau (segunda a sexta, 9h30-18h). De qualquer dos modos, não trouxera
comigo nem a bengala de cego nem os óculos escuros.

Recolhi ao meu quarto, esgotado mas satisfeito. Nas esca- das, no andar
por debaixo do meu, cruzei-me com uma mu- lher jovem que se desviou para
eu passar com um zelo que se assemelhava a timidez. Voltei-me para trás,
ainda à procura de uma saudação. Pouco mais consegui ver do que uma
cabeleira ruiva, engolida pela escuridão do corredor pobremente ilumi-
nado. Viveria ela no mesmo andar que eu?

Estudei o itinerário entre o meu prédio e a Rue Lobau com o afinco de um
assaltante.

Na manhã seguinte, os meus primeiros passos na rua, no papel de cego,
foram demorados mas firmes. A bengala fazia

RUA DA VELHA LANTERNA \textbf{161}

o seu \emph{toc-toc }reconfortante na calçada, como se fosse uma pul-
sação alheia aos meus movimentos, suficientemente vigorosa para se
sobrepor ao sopro do vento marítimo. Através dos óculos escuros, Paris
reduzia-se a uma colecção de massas descaracterizadas, mas não
ameaçadoras.

Na biblioteca, as atenções pareciam dominadas por um qualquer rumor
sobre um naufrágio, com grande perda de vidas. Murmúrios abafados,
chorosos, ecoavam e irritavam os meus ouvidos subitamente
hipersensíveis. Ninguém para ajudar o pobre invisiual, e eu preferia
assim. Esgueirei-me, mais a apalpar do que a ver, até à primeira estante
que encon- trei; abri um livro ao acaso; introduzi o meu desenho entre
as páginas, fechei o livro, repus o livro no lugar. Estava feito.

Ninguém dera por nada.

Lá fora começara a chover sem que o vento abrandasse.

Portanto, cumprir uma missão era isto: fazer as coisas que havia para
fazer, uma a seguir à outra; fazê-las simplesmente, cada novo gesto como
uma nota musical separada, única e valiosa.

A partir desse, os dias que se seguiram começaram a cum- prir a sua
sagrada missão de se sucederem uns aos outros e de se assemelharem a
ponto de se confundirem. As minhas roti- nas nasceram e instalaram-se
por si só: nada fiz para as trazer à existência nem para as sustentar. A
azáfama barulhenta da cidade, as idas e vindas, partidas e chegadas, as
marés e as borrascas, serviam de cenário à linha estreita e firme da
minha existência insignificante.

O período que me separava do dia 20 de Abril era indi- visível. Eu não
contava os dias: entregava-me, dócil, ao fluxo seguro do tempo, que nada
pode apressar nem suster.

Tinha tempo para cultivar o conhecimento dos meus vizi- nhos. Sondar a
natureza humana não era para mim nem um

\textbf{162 }ALEXANDRE ANDRADE

interesse nem um empecilho. Se me cruzava com uma pessoa no corredor ou
nas escadas, cumprimentava essa pessoa. Se al- guém me dirigia a
palavra, eu respondia. Nunca era parco em detalhes sobre a minha vida,
mesmo que fossem integralmente inventados.

Dito de outra maneira: o sentido de missão não é incom- patível com a
disposição para acolher esse desfile de caras e opiniões alheias que
muitos vêem como um ingrediente obri- gatório da vida.

A jovem ruiva, conforme vim a verificar, ocupava afinal um quarto
minúsculo no mesmo andar em que eu vivia. Era frequente vê-la
acompanhada por cavalheiros sorumbáticos e embaraçados, que preferiam
olhar-me nos olhos com bravura e um simulacro de dignidade em vez de se
esquivarem à mi- nha saudação cortês. E era tão pouco aquilo que eles
tinham a temer de mim e do meu discernimento moral! Eu não esta- va ali
para julgar quem quer que fosse. Eram-me indiferentes estes pequenos
atentados contra o sentido burguês das con- veniências. A minha missão
compunha-se de actos e de perío- dos de espera, nada mais. Winna --
assim se chamava a jovem, um belo e sonoro nome próprio dinamarquês --
dava mostras de um desprendimento idêntico e convidava-me para o seu
quarto sem o menor escrúpulo nem vestígio de inquietação quanto ao que
eu podia pensar sobre o seu emprego do tempo, por detrás daquela porta
quando fechada.

No seu francês aflautado e doce, repleto de expressões inu- sitadas,
Winna falou-me das piruetas do acaso que a tinham conduzido até ali,
longe da sua Dinamarca natal. Pelo que me toca, falei-lhe de mim e do
meu passado, mencionei a minha missão sem fornecer pormenores, respondi
à sua tímida in- sistência com bonomia, inventei menos acerca de mim do
que

RUA DA VELHA LANTERNA \textbf{163}

fizera com qualquer outra das pessoas que conhecera desde a minha
chegada a esta metrópole, ao coração deste enxame de corpos, rotas,
bulícios, afectos.

O lapso de tempo entre Janeiro e Abril chegou para tra- var outros
conhecimentos. O jovem que eu avistara anterior- mente, por exemplo,
veio bater à minha porta por sua própria iniciativa numa tarde de
domingo de tempo miserável. Aque- ci água para o chá na minha placa
eléctrica, conversámos, contámos as nossas histórias. As horas passaram
sem que se desse por isso. A partir de então, passámos a sair de noite
com alguma frequência. Nos \emph{bistrots }locais, bebíamos copo atrás
de copo daquela aguardente que é tão popular nas docas que o seu nome é
sinónimo de ``salário'' no picante calão local. Vim a saber que ele era
estudante de Anatomia, que vinha de Brest, que era pobre, que ansiava
por montar consultório próprio e arrecadar o suficiente para garantir à
mãe um reforma livre de apertos, que fazia versos nas noites de insónia.

As biografias dos outros meus vizinhos eram-me reveladas fragmento a
fragmento, por vezes em versões contraditórias, por vezes pelos
próprios, por vezes por terceiros, quase sempre com uma tonalidade de
escândalo ou de contentamento per- verso. Fiquei assim a conhecer a
filha do dono de um \emph{bistrot }que, apesar de muito jovem,
frequentava um marujo maltês quando este aportava a Paris; ouvi falar
(sem nunca o ver) de um indivíduo que dantes tinha sob a sua tutela quer
Winna quer outras duas moças (estas francesas de gema, do território de
Belfort), mas cuja indolência, assim como uma agorafobia que o
conquistava lentamente, o remetiam agora para o seu apartamento, onde
passava os dias a devorar bandas dese- nhadas, deixando o negócio
entregue às três raparigas; falei e troquei livros com uma criaturinha
intrigante que traba-

\textbf{164 }ALEXANDRE ANDRADE

lhava como secretária numa agência imobiliária do \emph{17ème ar-
rondissement, }séria como uma preceptora, obviamente aterrada com a
perspectiva de permanecer solteira o resto da vida mas sem a mínima
inclinação para fazer algo que a subtraísse a esse destino; captei
rumores sobre um homem que classificavam como um dos contrabandistas de
arte mais hábeis de Paris, cuja aparência insignificante mais depressa
evocaria um \emph{chef d' hôtel }obscuro, habituado a décadas de
eficiência anónima e atenção aos detalhes, preparando-se mentalmente
para uma reforma digna mas solitária.

Estas figuras apareciam-me como num desfile de mascara- dos, ou como uma
lista de personagens de uma peça folheada apressadamente mas nunca
vista. A máscara confundia-se com o carácter, cheio de detalhes e
rugosidades, da pessoa que a envergava, sem que essa confusão me
perturbasse. Conver- sar, fazer chá e discutir literatura eram
actividades agradáveis que ajudavam a fazer passar o tempo.

A Primavera chegava, o sol misturava-se com o granizo e as tardes
alongavam-se.

No dia 20 de Abril, fiz a pé o caminho entre o meu bairro, agora tão
familiar que os seus ruídos mais bruscos me emba- lavam como uma canção
da infância, e a extremidade mais ocidental do longo cais parisiense (ou
mais a jusante, como se diria se existisse rio em vez daquele mar
acobreado, a perder de vista). O itinerário era simples: bastava seguir
a fronteira entre a cidade e a água, insistir para lá da estreitíssima
Ilha dos Cisnes, estar atento ao ponto onde existiria a ponte Mirabeau
(se existisse rio para ser transposto) e onde um tímido molhe é tudo o
que assinala o ponto onde o poeta Paul Celan se terá precipitado, neste
mesmo dia, no ano de 1970.

O contraste com o burburinho a que me habituara era

RUA DA VELHA LANTERNA \textbf{165}

impressionante: a actividade humana era aqui esparsa e alheia à urgência
e frenesim que se sentiam, como um pulsar, em sectores do cais mais
concorridos. As casas sugeriam desleixo, o lixo amontoava-se, gaivotas
sobrevoavam a área com a placi- dez de quem se sabe em maioria.

Sentado numa grade de cerveja voltada ao contrário, desenhei durante
todo o dia. Aquela desolação, contra as minhas expectativas,
inspirou-me. Descobri capacidades que desconhecia para fazer justiça a
todos aqueles melancólicos cambiantes de cinzento.

No dia seguinte, com uma exaltação que me esforçava por disfarçar,
muni-me da minha bengala e dos meus óculos escuros; sondando com
precaução o caminho que cumprira na véspera, dirigi-me para a biblioteca
municipal da Rue de Musset, quase nos confins da cidade. Fizera os meus
cálculos de maneira a chegar pouco depois das 10 horas, hora de aber-
tura, para evitar um excesso de presença humana, mas ainda assim não me
pude escusar aos préstimos de uma senhora ex- tremamente gentil, que me
pegou pela manga do sobretudo e me conduziu até à secção de clássicos
franceses em Braille e caracteres grandes sem que eu lhe tivesse pedido
nada. Foi já ao sair, e quase com gestos de espião, que consegui
introduzir o meu desenho num livro escolhido ao acaso e prontamente
reposto na prateleira. Nada mais me retinha ali. O caminho de regresso
foi cheio de uma alegria miúda mas intensa.

Não sei se um acto repetido duas vezes chega para se poder falar de
rotina. Mas não era a semântica que me ocupava a mente nesta altura. A
passagem à acção trazia-me uma euforia doce; a certeza de que essa acção
formava uma cadeia com as anteriores e com as que se lhe seguiriam
proporcionava-

-me uma sensação de conforto inexprimível, inviolável. Num

\textbf{166 }ALEXANDRE ANDRADE

quadrado de papel, que colei por cima da minha cama, ga- ratujei: ``26
de Abril''.

Ainda nesse dia, Winna veio tamborilar com os dedos na minha porta. O
seu embaraço era evidente, mas algo na minha expressão ou na minha
postura deve ter sido sufi- ciente para vencer as suas hesitações.
Sucedera aquilo que eu, muito abstractamente, receara desde o princípio,
embora sem atribuir demasiada importância à eventualidade: ela avistara-

-me a sair do prédio equipado com os meus apetrechos de cego e
naturalmente tinha ficado perplexa. Sorri-lhe como se se tratasse de
algo banal, como na realidade era. Esforcei-me por lhe transmitir, desde
logo, a certeza de que ela não estava a co- meter qualquer indiscrição.
Sem delongas nem rodeios, porque o caso não o justificava, pus Winna ao
corrente de tudo:

a minha vinda a Paris com uma missão muito bem definida; os cinco
lugares onde encontraram a morte cinco escri-

tores, num passado mais ou menos próximo;

os cinco desenhos a realizar nesses locais, no aniversário de cada uma
das cinco mortes;

o abandono dos cinco desenhos em livros escolhidos ao acaso das estantes
das bibliotecas municipais mais próximas do local do desenho e da morte
do escritor;

um fragmento de endereço postal escrito no verso de cada desenho, de
forma que apenas aquele que possuísse todos os desenhos o pudesse
reconstituir;

a intenção de abandonar Paris imediatamente após levar a missão a cabo.

Não escondi o que quer que fosse. A dissimulação seria, além de
trabalhosa, inútil. Nada havia de secreto na minha missão, bem vistas as
coisas. Nada iria ser alterado pelo facto de uma pessoa, ou mais do que
uma pessoa, estarem a par do

RUA DA VELHA LANTERNA \textbf{167}

que eu tinha feito e do que tinha ainda para fazer.

Winna escutou as minhas explicações com uma ausência de surpresa pela
qual me senti grato, sem saber exactamente porquê. Os meus projectos
pareciam-lhe tão naturais como a mim próprio. Limitou-se a questionar-me
sobre alguns deta- lhes. Por exemplo, para quê o disfarce de invisual?
Expliquei-

-lhe que fora o processo que encontrara (não seria decerto o único) para
garantir que a escolha do livro era totalmente aleatória, por ser feita
às cegas. Será que me dava conta de como a probabilidade de alguém vir a
reconstituir o endereço era ínfima? Ínfima a ponto de ser nula, para
todos os efeitos práticos, respondi-lhe com tranquilidade não fingida.

A pergunta que Winna não fez, e que eu teria esperado da sua parte, era
o porquê destes meus trabalhos, a razão de um plano que, por mais
natural que parecesse explicado a dois, em volta de uma chávena de chá,
num princípio de noite de Primavera fresco e ventoso, pareceria
extravagante aos olhos do mundo. Justificava-se falar de ``decepção''
para descrever o que senti quando se tornou claro que essa pergunta iria
ficar por fazer?

Winna ficou assim a perceber por que razão eu evitava envolver-me nas
peripécias da vida. Só a missão tinha valor aos meus olhos, incluindo os
longos tempos de espera. Com- parou-me com um fantasma que tudo
atravessa por carecer de solidez e que também ocupa o seu tempo próprio
-- mas fê-lo com um sorriso, como se esta minha discrição e recusa em
envolver-me nas vidas alheias fosse, para ela, um traço de carácter raro
e merecedor de louvor.

E, contudo, o punhado de dias que faltavam até à próxima data foram
ricos em fricções inesperadas com aquelas perso- nagens para as quais o
prédio onde eu vivia funcionava como

\textbf{168 }ALEXANDRE ANDRADE

palco de anseios, ocupações e impulsos. O suposto contra- bandista
interpelou-me no patamar (era a primeira vez que me era dado ouvir a sua
voz) e pediu-me, com uma naturalidade demasiado canhestra para ser
estudada, para guardar no meu quarto uma mala que lhe pertencia. A
secretariazinha pediu-

-me aconselhamento sobre assuntos do coração. O estudante tentou
convencer-me a ler a peça em versos decassilábicos que o ocupava há
anos. Quanta convicção naquelas solicitações! Claro que me apressei a
aceder a tudo, podia dar-me a esse luxo. Os desmandos sentimentais da
minha vizinha não requeriam mais do que bom senso, à peça não faltava
algum valor apesar de alguns excessos e ingenuidades. Quanto à mala,
guardei-a debaixo da minha cama sem me entreter em especulações sobre o
seu conteúdo. Os meus sonhos dessa noite foram comedidos, neutros, sem
forma.

26 de Abril. Era a primeira vez que me aventurava tão para norte desde a
minha chegada. Quando se chega aos bair- ros de Pigalle, Montmartre e
Abbesses, a cidade metamorfo- seia-se de metrópole marítima em burgo
fechado, preso aos seus atavismos terrestres e imutáveis. Até o ar que
se respira é diferente, estagnado e intoxicante como ópio. As pessoas
seguem os estranhos com um olhar pesado de preconceitos e ávido de
julgar.

Mário de Sá-Carneiro morreu, pela sua mão, neste dia em 1916, no Hôtel
de Nice, Rue Victor-Massé, com recurso a cinco frascos de arseniato de
estricnina. O Hôtel de Nice chama-se agora Hôtel des Artistes. Há uma
placa na fachada a assinalar o evento funesto. O aspecto exterior do
hotel é neutro, anónimo, opaco às misérias da vida. A reputação do hotel
não é das melhores. Alguns comentários de utilizadores, recolhidos na
Internet:

RUA DA VELHA LANTERNA \textbf{169}

\emph{Sicuramente da evitare. Siamo arrivati stanchi morti dal viaggio e
invece di trovare un albergo ci siamo ritrovati in una stanza di un
motel ad ore.}

\emph{Nous sommes arrivés, nous avons vu la chambre et nous avons fui.
Sale, pas entretenu, bruyant et enfumé (!!), des lits des années 50, les
papiers et peintures (au plomb ?) craquelés, la tuyauterie colmatée au
ruban adhésif. Digne d'un film noir.}

\emph{Non è un hotel...ma un motel a ore...squallido, sporco ten- dente
al disgustoso...! Abbiamo dormito vestiti e la mattina se- guente siamo
scappati via!!!!!}

\emph{Difficile de faire plus ``miteux'' avec une description qui est
totalement fausse et mensongère. Risque alarmant de sécurité avec des
fils électriques dénudés sortant des murs en plusieurs endroits.}

Não entrei. Não precisava de entrar. Fiz o meu desenho, sentado na
esplanada de um café, sem me apressar. Omiti al- guns detalhes por
impaciência pura. Alguma coisa me compe- lia a abandonar aquele local.

``Evento funesto'', ``misérias da vida''\ldots{} Parece que estou a
escrever uma crónica para um jornal de província! O come- dimento verbal
não é uma condição necessária para a minha missão, mas ajuda a manter as
coisas em perspectiva.

No dia seguinte entrei na biblioteca da Rue Chaptal com uma desenvoltura
que teria certamente feito nascer suspeitas num observador mais atento.
O acaso guia os nossos passos com naturalidade quando nos encontramos
num local novo, nunca dantes visitado. Quase esbarrei com uma estante,
aliás

\textbf{170 }ALEXANDRE ANDRADE

tão mal colocada que perturbava a circulação dos cidadãos do- tados de
visão; deixei o desenho da véspera entre as páginas do primeiro volume
que me apareceu nas mãos; saí novamente para a rua, aliviado, quase
alegre.

A temperatura permitia agora deixar a janela do meu quarto aberta
durante a noite.

Passaram-se dias, passaram-se semanas sem que o su- posto contrabandista
com ar de serviçal, o dono da mala, se mostrasse. Mesmo sem sentir
curiosidade a respeito do seu conteúdo, a simples presença debaixo da
minha cama daquele volume inerte parecia interpelar-me e distorcer a
corrente dos meus devaneios. Não era coisa que me sentisse capaz de
ignorar. Por vezes, dava por mim a sair do quarto de forma automática,
sem intenção declarada. Só me apercebia da minha decisão (tratava-se
mesmo de uma decisão?) depois de fechar a porta atrás de mim. Nessas
alturas acabava quase sempre por ir bater à porta do estudante, o que me
expunha às suas perguntas sobre o que eu achava da sua peça. Eu nunca me
esquivava a revelar a minha opinião sincera, onde cabiam tanto elogios
aos méritos inegáveis da obra como apreciações menos positivas a
propósito do ritmo, da evolução psicológi- ca das personagens e da
gramática demasiado artificial. Para minha surpresa, o menor reparo
punha-o fora de si e desen- cadeava nele um acesso de fúria que,
conforme rapidamente compreendi, era dirigido menos a mim do que àquilo
que ele percebia como sendo a sua irrecuperável inadequação enquanto
escritor e também, por arrasto, enquanto pessoa. Era óbvio que havia ali
mais do que mero amor-próprio ou ambição literária. Não foram
necessários dons inquisitórios demasiado apurados da minha parte para
adivinhar a presença latente de uma rapariga que ele desejava, mais do
que tudo

RUA DA VELHA LANTERNA \textbf{171}

na vida, conquistar. Ainda se seduzem mulheres, nos dias que correm, por
meio de dramas rimados? Custava-me a crer, porém ele garantia-me com
calor ser esse o único caminho que conduzia ao coração daquela por quem
suspirava. Convidou-

-me, por mais do que uma vez, a encontrar-me socialmente com ambos.
Alegava que se sentia menos tenso e mais natural a três do que a dois.
Recusei com firmeza. Não queria intro- meter-me numa história
que\ldots{}

Como eu dizia a Winna (nunca a ia procurar por ter medo de ser
inoportuno, era ela quem vinha bater à minha porta, ou antes raspar
muito ao de leve com os nós dos dedos), eu não queria intrometer-me numa
história que não era a minha e onde estaria, irremediavelmente, a mais.

Ninguém está a mais numa história, respondia Winna, acariciando a trança
ruiva pousada sobre o seu ombro es- querdo. As histórias recrutam-nos ou
ignoram-nos segundo a sua lógica muito própria e inflexível. As
peripécias da vida admitem protagonistas ou figurantes, mas nunca
passageiros clandestinos.

O homem que servira de protector a Winna nos seus primeiros tempos em
Paris também era dinamarquês, contou-

-me ela. Eu nunca lhe pusera os olhos em cima, nem esperava alguma vez
fazê-lo se ele se obstinasse em não sair de casa. Dizia-se que era
grande e alto, rotundo como um tonel, louro até às pestanas. Chamava-se
Ole. Eu não conseguia perceber se o sentimento predominante em Winna,
quando me falava dele, era o receio ou a ternura.

Começou a aparecer no prédio um indivíduo que nin- guém conhecia. Era
baixo e ágil, rosto comprido, bigode ralo, um quê de insolência no modo
como encarava as pessoas. Em breve foi possível perceber que existia
intimidade entre ele e

\textbf{172 }ALEXANDRE ANDRADE

a secretária da agência imobiliária. Davam-se ao trabalho de dissimular
quando na proximidade do olhar dos vizinhos, mas não era raro
surpreender em público o seu afecto. A jovem fizera excelente proveito
dos meus conselhos, ou então igno- rara-os por completo -- o que era sem
dúvida mais provável e mais avisado.

O navio do marujo maltês (tinham-me garantido que era maltês, mas
circulava mais do que uma versão) que se envolve- ra com a filha do dono
do \emph{bistrot }estava em Paris. Da primeira vez que me cruzei com
aquele corpo compacto, de gestos tor- nados simples e exactos pela vida
a bordo, notei o volumoso embrulho que ele transportava debaixo do
braço. Era, como me explicou mais uma vez Winna (a companhia que ela me
fazia era agradável e prolongava-se por horas), um presente que ele
trazia para a sua amante demasiado jovem, como sempre fazia quando
regressava de paragens longínquas. Os presentes costumavam ser
grandiosos: ícones preciosos, braceletes anti- gos, tapeçarias raras. A
destinatária destas oferendas nunca as poderia guardar no seu minúsculo
quarto, por isso distribuía-

-as por amigos habituados à discrição.

Que secretos anseios e desgostos sentiria alguém por amar um eterno
viajante que passava mais tempo em latitudes inimagináveis do que em
terra firme, apto a ser abraçado? E isto numa idade tão jovem?

Alguém me disse que o novo parceiro da secretária an- dava a fazer
perguntas sobre mim e sobre os meus hábitos. Arrependi-me por ter
aceitado guardar a mala no meu quarto. Pensei que nem sequer lhe chegara
a sentir o peso, como se o peso contivesse todas as respostas e fosse
tudo aquilo que eu precisava saber para me sentir tranquilo. Continuei
sem tocar na mala.

RUA DA VELHA LANTERNA \textbf{173}

Chegáramos ao mês de Julho. A passagem do tempo sur- preendeu-me como um
intruso.

Jim Morrison, poeta e vocalista da banda fte Doors, mor- reu na noite de
2 para 3 de Julho de 1971 num apartamento da Rue Beautreillis. A
sequência dos acontecimentos dessa noite suscita ainda hoje muita
discussão. Existem versões contra- ditórias. Não falta quem defenda que
a morte ocorreu noutro ponto de Paris e em circunstâncias muito
diferentes daquelas que a história oficial consagrou. Abundam as teorias
da cons- piração: o médico que assinou a certidão de óbito não exis-
tia, etc. A Rue Beautreillis ficava muito perto de minha casa. Nenhuma
excursão a paragens pouco familiares foi exigida pela missão, desta vez.
A rua ficava suficientemente afastada do porto para que os rumores do
cais chegassem abafados, quase irreais, mas suficientemente próxima para
não nos deixar esquecer a proximidade do mar e a promessa de outros
mundos, forasteiros, devir.

Encostei-me à fachada de uma casa para desenhar a fachada da casa do
outro lado da rua estreitíssima. A minha mão movia-se com um abandono
que nem sequer era vontade própria. Guardei desses instantes a impressão
confusa de estar a desenhar arquétipos, ideias, fantasmas sem qualquer
ligação com aquilo que estava perante os meus olhos. Mal olhei para os
traços precipitados deixados na brancura do papel. Dobrei a folha em
quatro e meti-a no bolso. O tempo que gastei com este desenho foi tão
reduzido que ainda pude passar por casa, equipar-me de cego e, com o
passo cauteloso próprio de quem está privado da visão, vencer o curto
trajecto até à biblioteca Forney, Rue du Figuier, antes da hora de
encerramento. O livro que tacteei e onde deixei a folha dobrada tinha o
peso e o cheiro de uma obra antiga, cheia de sabedoria austera.

\textbf{174 }ALEXANDRE ANDRADE

O Verão e o tempo quente trouxeram um afluxo de tu- ristas mais contínuo
e mais intenso. Os forasteiros gostavam de se aventurar pelas ruas
acanhadas e tortuosas limítrofes ao porto, raramente hesitavam em se
misturar com os locais, em beber uma aguardente nos bares frequentados
por estivadores, marujos, rameiras e batoteiros ociosos. O sorriso
deliciado que lhes atravessava o rosto mostrava bem como este convívio
com a fauna nativa os fascinava. Sem dúvida, ensaiavam já as frases que
usariam para contar a experiência a amigos e familiares, nos seus lares
e espaços de socialização, lá longe, nas suas ter- ras de origem.

Havia também famílias inteiras dos arredores de Paris que desciam à
cidade para admirar os paquetes atracados. Os filhos, pequenos e
excitados ou adolescentes e entediados, seguiam atrás dos adultos, estes
vagamente intimidados com a grandiosidade e diversidade de tudo aquilo
-- barcos, casas, pessoas, contentores, o mar. Era o ponto alto de um
dia onde sem dúvida caberia, ou já tinha cabido, o piquenique no parque
Monceau ou Buttes-Chaumont, as \emph{quiches lorraines, }as baguetes com
pasta de atum ou salsichão, o vinho tinto de produção artesanal.

Veio uma vaga de calor. Sufocava-se nas minhas águas-

-furtadas. Comecei a sair com maior frequência, chegava a passar o dia e
o princípio da noite fora de casa. Ia muito ao cinema, deixava-me ficar
sentado em esplanadas com uma bebida à frente escolhida no impulso do
momento. Passeava muito pouco.

Winna falara-me do acaso como ``instigador benevolente de
acontecimentos''. No exterior, as probabilidades de um encontro casual
aumentavam. Na tarde de um dia abafado em que a trovoada ameaçava,
avistei a duas mesas de mim

RUA DA VELHA LANTERNA \textbf{175}

aquela por quem o meu vizinho, o autor da peça, suspirava; ele tinha-ma
apresentado anteriormente, após submeter ao meu julgamento um punhado de
haikus (péssimos). Nicole (era este o nome da rapariga) reconheceu-me
imediatamente e ace- nou-me. Sentei-me ao seu lado de muito bom grado.
Ignorava qual era o estado actual das suas relações com o estudante, mas
estava convicto de que nunca passariam além da amizade

-- os seus temperamentos eram demasiado díspares.

Nicole foi directa ao assunto. Estava a coordenar uma antologia do conto
português e brasileiro do século XIX e procurava alguém para rever as
traduções para a língua france- sa. Disse-lhe que ia pensar, mas na
verdade a minha decisão ficou tomada logo ali. Não percebo mais de
literatura do século XIX do que um qualquer amador esforçado, mas a
tarefa não parecia demasiado exigente e agradava-me ter alguma coisa
entre mãos enquanto não chegava a última das datas que mar- cara no meu
calendário, a partir da qual a minha presença em Paris deixaria de fazer
sentido.

Fiquei por essa altura a saber que a secretariazinha se des- pedira do
emprego. Nunca a julgara capaz de um gesto desses.

«O que é de mais é de mais» ouvi-a dizer à porteira do nosso prédio.
Parecia 10 centímetros mais alta quando se exprimia com aquela
exaltação, toda ela orgulho e raiva. O detestá- vel homem do bigode
aparecia agora mais raramente. O que andaria ele a preparar? Os dias
começavam nitidamente a ficar mais curtos.

Winna revelara-me a sua intenção de entrar em audições para um papel
numa comédia musical. Foi com grande sur- presa que, quando chegou o
dia, ela me convidou para a acompanhar. Disse-me que se sentia melhor
com alguém conhecido ao lado dela. Alguém conhecido, pensei eu sem

\textbf{176 }ALEXANDRE ANDRADE

o dizer, mas não demasiado próximo, para que a recordação de um eventual
falhanço não a acompanhasse para o resto da vida. O convite não fora
feito de maneira leviana: havia embaraço e ponderação nas palavras que
escolhera, na ma- neira como alisava a longa trança ruiva. As audições
tinham lugar nos arredores, em Aubervilliers. Apanhámos o metro,
passámos todo o trajecto a falar de coisas triviais, ninharias sem
consequência, tudo o que ajudasse a dissipar a tensão que ela estaria
seguramente a acumular.

Não sei se a audição correu mal ou bem. Eu até gostei da vozita da
Winna, do seu timbre rouco e pouco vulgar, dos seus movimentos de dança
entre o frágil e o provocador. Mas seria aquele estilo que procuravam
para o papel? Tinha as minhas dúvidas, mas claro está que tive o cuidado
de as guardar para mim e de recompensar o esforço da Winna com elogios e
encorajamentos, sentado à frente dela no café onde lhe ofereci uma água
Perrier, um café que descobríramos por um feliz milagre no meio de um
bairro de armazéns, gara- gens e lojas entaipadas. A Winna, talvez como
consequência do alívio por ver ultrapassada aquela prova, abriu-se
comigo como nunca tinha sucedido até aí. Deixando-se de rodeios, admitiu
que o que a trazia ali era o desejo muito vivo de mu- dar de vida,
abraçar novos desafios, romper com um passado que contivera o seu
quinhão de humilhações e frustrações. A música e os palcos eram uma das
opções, mas porque não a ilustração de livros infantis ou a venda
\emph{online }de joalharia de autor? Os anseios do presente abriram a
porta para as reve- lações sobre o passado, e também aqui Winna me
surpreendeu pela franqueza com que descreveu a sua vida com Ole em
Copenhaga, a mudança para Paris, o deslumbramento inicial que depressa
dera lugar à rotina e a um sentimento de degra-

RUA DA VELHA LANTERNA \textbf{177}

dação, a decadência física do protector, cujo estado actual de
sedentarismo mórbido e apatia parecia impossível de conciliar com o
homem empreendedor e irresistível que ele fora dantes, capaz de arrastar
tudo à sua frente para conseguir os seus objectivos, a amizade e
cumplicidade que criara com as suas associadas de Belfort e que lhe
tinham trazido um conforto que ela já se julgava incapaz de alcançar.

Até este momento, as revelações que eu tinha deixado es- capar sobre a
minha missão nunca tinham surgido por ini- ciativa minha. Em conversas,
em esporádicas ocasiões sociais, eu nunca me esforçava por esconder as
circunstâncias que me tinham trazido a Paris, as minhas movimentações,
os meus afazeres, mas nunca era o primeiro a trazer o assunto para dis-
cussão. A transparência era tão alheia aos meus intentos como a
dissimulação; falar ou calar eram ortogonais ao sucesso. Foi só quando
me vi face a face com Winna, no café de Aubervil- liers, que o esforço
para contar a minha história a alguém me apareceu, singularmente, como
parte dessa mesma história, nela contido e indestrinçável.

Os olhos de Winna declinavam a promessa grave de escutar sem julgar, de
escutar até ao fim acontecesse o que acontecesse. Eu não pedia o ouvinte
perfeito, mas era o ouvinte perfeito que aguardava, com paciência,
sentado à minha frente.

É de costas que um corpo denuncia o seu estado de aten- ção absoluta.
Desejei estar a aproximar-me da mesa onde nos sentávamos, vindo de
longe, e surpreender Winna de costas com o olhar.

Tudo começara em Lisboa. Eu tinha ido ao cinema sozi- nho, para ver o
filme \emph{Elogio do Amor, }realizado por Jean-Luc Godard.

\textbf{178 }ALEXANDRE ANDRADE

Neste filme, a personagem interpretada pelo actor Bruno Putzulu explica
a dificuldade que sente em produzir um filme protagonizado por um
``adulto''. É-se jovem ou velho, catego- rias universalmente
reconhecidas e acompanhadas pelo respec- tivo aparato de convenções; mas
como ser um ``adulto''? Num dado momento, ele descreve ``ser adulto''
(ou seja, PESSOA) como a ambição máxima a que se pode aspirar. Nesse
ins- tante preciso, levantei-me da minha cadeira e saí da sala sem olhar
para trás, saí do cinema para a Avenida Guerra Junquei- ro, banhada por
um sol raso de princípio de Inverno, vagueei à toa por entre casais,
famílias e transeuntes ocupados com as compras de Natal. Dei por mim num
bairro desconhecido, esgotado mas surpreendido com a nitidez rude da
minha resolução. Nesse mesmo dia comprei o bilhete de avião, enchi uma
mala com mudas de roupa e livros e voei para uma Paris imaginada
deixando para trás uma Lisboa tão autêntica que feria retinas incautas
todos os dias, apenas moderadamente transfigurada pela esterilidade,
pela dormência.

Ambicionei partir do zero, deixar para trás uma cidade povoada de
memórias e episódios que viciavam à partida qualquer plano ou
itinerário. Quis cortar amarras, livrar-me da tirania demasiado terna
dos laços, dos motivos e da convi- vência. Acima de tudo, quis expurgar
a minha vida de sentido e cumprir uma missão com o escrúpulo que
normalmente é reservado para as grandes demandas reluzentes de urgência
moral. A ideia da visita aos locais da morte de autores famosos, dos
desenhos e do disfarce surgiu durante a viagem. Os detalhes não
importavam. O que importava era um calen- dário, um quarto meu, a
certeza reconfortante do que havia para fazer.

Ali, face a face com Winna no café de Aubervilliers, a

RUA DA VELHA LANTERNA \textbf{179}

minha convicção parecia esboroar-se. Nenhum ângulo ou es- colha de
palavras parecia próximo de transformar a minha presença em Paris e a
minha missão em algo mais do que um capricho, uma anedota boa para
contar aos amigos se não houver nada de mais suculento para partilhar.
E, contudo, os olhos de Winna mantinham a expressão de gravidade neutra,
exactamente a meio caminho entre o escárnio e a admiração. Se ela me
tivesse repetido que toda a empreitada era inú-

til, que a probabilidade de alguém alguma vez vir a reunir os cinco
desenhos, reconstruir o endereço e fazer-me chegar notícias desse evento
improvável era na prática igual a zero, teria respondido com a alegação
de que me bastava a exis- tência dessa probabilidade, por ínfima que
fosse. Eu não esperava favores do acaso; melhor ainda, trabalhava para
reduzir a hipótese de sucesso a um ponto irrisório. Lucida- mente, sabia
que seria necessário repetir a experiência um número absurdamente alto
de vezes para que acontecesse uma conjugação de factos tão inverosímil.
Estava em paz com a hipótese de os meus esforços serem em vão. Era isto
que teria para dizer a Winna, se os seus olhos exprimissem cepticismo ou
uma curiosidade mais ou menos moderada pela polidez. Mas os seus olhos
permaneciam neutros, puros receptáculos para a imagem em movimento de um
homem que fala.

A minha narrativa soçobrava à medida que eu a debitava. Nada restava a
não ser a vaga recordação do esforço muscular, da minha postura (um
pouco inclinado para a frente, como em flagrante delito de proselitismo)
e do movimento dos órgãos da fala.

Naquela terra de ninguém, no silêncio que se instalou, apercebi-me pela
primeira vez da respiração de Winna. Aquele

\textbf{180 }ALEXANDRE ANDRADE

corpo a respirar era de repente a única coisa real.

Parecia um metrónomo.

O fluxo do tempo provinha daquele peito que subia e descia.

O mesmo tempo onde existiam todas aquelas figuras que até aí se tinham
limitado a povoar esparsamente o meu calendário.

O estudante.

O proxeneta indolente. A secretária.

O pretendente da secretária, com os seus ares de Don Juan. O traficante
de arte desaparecido.

O marinheiro maltês e os seus presentes extravagantes.

Todos os figurantes, dotados de ânimo e propósito, à solta nas ruas de
Paris.

A mala, que permanecia debaixo da minha cama, inviolada. A mala parecia
quase vazia. Sopesei-a, abanei-a, surpreen- dido com a sua leveza. Não
devia conter mais do que alguns

objectos soltos.

Junto ao fecho havia marcas inequívocas de tentativa de arrombamento.
Quem teria penetrado no meu quarto na mi- nha ausência? Fosse quem
fosse, apenas se interessara pela mala; tudo o resto estava intacto.

As minhas suspeitas recaíram imediatamente sobre o homem do bigode,
tanto mais que ele deixara de aparecer, sem dúvida desencorajado pelo
falhanço da sua tentativa de acesso à mala. O ar taciturno da secretária
apenas vinha confirmar esta conjectura. Quando consegui falar com ela, a
maneira como se referiu a ele, num pretérito carregado de amargura e
despeito, dispensou pormenores. Fiquei também a saber que obtivera um
novo emprego, na contabilidade de uma pequena empresa do sector do
calçado. Desejo-lhe

RUA DA VELHA LANTERNA \textbf{181}

todas as felicidades neste mundo, porque ela as merece.

O estudante começou a desafiar-me para jogar xadrez quando descobriu que
eu era um amador. Jogávamos sem relógio; as nossas partidas
prolongavam-se pela madrugada dentro, ambos preferíamos os duelos
posicionais e as aberturas fechadas aos festivais tácticos e aos ataques
arrojados. As con- versas que mantínhamos, intensas a ponto de nos
esquecermos de quem era a vez de jogar, revelaram-me outra faceta dele,
menos dogmática e mais espirituosa. Ele abandonara em de- finitivo o
hábito de me submeter os seus escritos, e eu nunca mais tinha abordado
esse assunto. Foi de sua livre iniciativa que me revelou que acabara
tudo com Nicole, mas garantiu-

-me que continuavam a ser bons amigos e que não se importava que nos
continuássemos a ver. O conto que eu tinha entre mãos para rever estava
à vista, em cima da minha secretária, ao lado do dicionário.

A revisão dos contos corria bem, após um começo lento. Aproveitei para
descobrir vários autores que nem sequer suspeitava que existiam. Nicole
aconselhou-me a regularizar a minha situação laboral, porque isso seria
vantajoso para todos. Passei horas em repartições de Finanças e balcões
da Segurança Social, reenviado de um lado para o outro, mal
informado\ldots{}

Um dia, a porta do quarto do traficante de arte apareceu selada. A
porteira contou que tinham estado lá dois agentes da polícia,
acompanhados por um oficial da justiça. Pareceu relutante em acrescentar
pormenores. Ela estaria ao corrente da mala que eu guardava no meu
quarto?

Eu hesitava em revelar à polícia a existência da mala. O es- tudante
aconselhou-me simplesmente a abri-la, inspeccionar o conteúdo e
livrar-me dela. Ajudou-me a forçar os fechos com

\textbf{182 }ALEXANDRE ANDRADE

uma verruma que foi pedir emprestada, sob um pretexto falso, ao marido
da porteira, biscateiro nas horas vagas.

Dentro da mala encontrámos apenas um exemplar de \emph{Splendeurs et
Misères des Courtisanes, }de Balzac, e um frasco de água-de-colónia meio
vazio.

Eufórica, Winna veio bater à minha porta para anunciar que lhe tinham
dado o papel que ambicionava. Balbuciei os meus parabéns. Ela mal cabia
em si de contente. Garantiu-

-me que eu era a primeira pessoa a quem contava a novidade. Pergunto-me
se terá detectado algo de frouxo nas minhas felicitações. Senti remorsos
por ver com contrariedade a pers- pectiva de ela se mudar para um sítio
menos abominável, mais de acordo com a sua condição de actriz em
ascensão, pronta a surpreender o mundo com o seu talento. Todo o seu
rosto transbordava excitação, os seus olhos pareciam ver apenas um
futuro radioso, alheio à miséria pálida daquelas paredes.

As semanas tinham-se sucedido, discretas e anónimas. Os primeiros
rigores do Inverno anunciavam-se já. Num piscar de olhos, no interior do
prédio, o calor abafado cedera lugar aos primeiros frios e à humidade
que tudo invadia, com equani- midade perfeita.

Quase deixei passar o dia 18 de Novembro sem me aper- ceber disso. Um
jornal abandonado num banco de autocarro alertou-me para a data. Havia
ainda tempo para ir a casa bus- car o bloco e os lápis, para me dirigir
à Rue Hamelin, algures entre o Trocadéro e o Arco do Triunfo, para
reproduzir no papel a sobriedade burguesa daquela rua, algum elemento de
mobiliário urbano, talvez a silhueta fugidia de um ou outro passante.

Marcel Proust morreu a 18 de Novembro de 1922 no número 44 desta rua (e
não, como muitas vezes se pensa,

RUA DA VELHA LANTERNA \textbf{183}

no quarto revestido a cortiça do Boulevard Haussmann).

Neste dia enevoado, de sol intermitente, a minha mão recusou-se a traçar
qualquer semelhança, por vaga que fosse, com o que via à minha frente.
Desenhei figuras geométricas, padrões, pedaços de letras, símbolos ao
acaso. Escrevi algaris- mos. Furei o papel com a mina, dobrei-o, deixei
que a humi- dade do ar o amolecesse.

A minha visita do dia seguinte à biblioteca do Museu Gal- liera, Avenue
Pierre 1er de Serbie, quase provocou um peque- no tumulto. Ao chegar, de
bengala em riste, vi-me o alvo das atenções escandalizadas de utentes e
funcionários. Só ao fim de alguns momentos me apercebi de que esquecera
os óculos escuros em casa. Sem dúvida, os meus olhos varriam o espaço
com a rapidez de quem entra num lugar pela primeira vez; a minha postura
e os meus movimentos não podiam ser os de um invisual e denunciavam a
impostura. Consegui ain- da, com a discrição possível, deixar a informe
folha de papel coberta com os meus rabiscos entre as páginas de um
qualquer livro de referência, antigo e maciço, antes de me esgueirar
pela saída para evitar ser interpelado.

Fiz algumas diligências para doar a minha bengala a uma instituição de
caridade, mas o meu impulso generoso foi to- mado por frivolidade.
Continuei a guardá-la, como sempre o fizera, no fundo do roupeiro.

Comprei um cobertor muito espesso, feito à mão. A re- cordação do frio
que passara durante as noites do último Inverno atormentava-me.

Winna mudou-se para um estúdio perto da Gare de l'Est. Não estava mal
pensado: ficava a meio caminho entre o centro da cidade e o local onde
decorreriam os ensaios. Encontrei a morada rabiscada num papel, preso
entre a minha porta e

\textbf{184 }ALEXANDRE ANDRADE

o lambril. Winna estaria a viver sozinha no seu estúdio? E que direito
me assistia para perguntar isto? Winna sabia que eu sabia que não
precisava de convite para a ir visitar. Bastava apanhar o metropolitano
num dia qualquer, sem avisar. Mas eu hesitava.

Nicole devolveu-me alguns dos contos que eu revera. Disse-me que ainda
precisavam de algum trabalho. Trazia um cachecol de lã cor de mel que
lhe ficava bem.

Comecei a ter mais gosto em confeccionar as minhas próprias refeições na
placa eléctrica que demorava uma eter- nidade a aquecer. Por tentativa e
erro, descobrira as lojas do bairro onde se podia comprar o melhor pão,
os melhores legu- mes, a melhor fruta\ldots{}

A leitura de \emph{Splendeurs et Misères des Courtisanes }revelou-

-me a sagacidade criminosa de Vautrin e o destino trágico da

\emph{demi-mondaine }Esther Van Gobseck.

Pediram-me para assinar uma petição contra a alteração dos horários do
comércio local. Assinei.

As feições devoradas pelo choro da filha do dono do \emph{bistrot
}revelavam tudo: o seu desgosto, o amante embarcado, a von- tade de
partir.

Uma manhã, ao sair à rua, senti algo de diferente no ar e na luz; o mar,
silencioso e paciente, era o mesmo de sempre.

MARÇO-MAIO 2012

\subsection{O MESMO POETA}

assim, amiga, a casa é antiga e muito espaçosa. Fica perto do Campo de
Santa Clara. Nos dias em que a

Feira da Ladra funciona, o burburinho faz lembrar bandos de aves enormes
e impacientes. Por fora, a casa parece um palá- cio. Por dentro, um lar
pequeno-burguês provido de nichos para o amor e para o ócio. Entra-se
directamente para uma sala ampla, cheia de móveis cuja disposição parece
ter sido de- cidida pela sorte e pelo hábito. A sala está decorada com
telas abstractas onde dominam as cores quentes e as estruturas geo-
métricas rígidas, que se repetem como num padrão de quadro para quadro.
Os quadros foram pintados pelo Paulo, já te falei do Paulo, é um dos
três inquilinos, um modelo de simpatia e contador de histórias
incansável. O quarto do Paulo dá para a sala. É um quarto pequeno, limpo
e arejado, que lhe serve também de \emph{atelier. }O quarto do Paulo
partilha uma parede com o do Maciej, que é aquele Erasmus polaco de quem
já te falei, que estuda engenharia e toca sax tenor. O Maciej é um rapaz
cinco estrelas, dou-me lindamente com ele. O quarto do Maciej também tem
acesso directo à sala. Bem pelo contrário, o quarto do O... O quarto do
O...

\textbf{186 }ALEXANDRE ANDRADE

\begin{itemize}
\tightlist
\item
  Amiga, esses prolegómenos estão a deixar-me os nervos em franja. O que
  há com o quarto do O.? Desembucha.
\item
  Para chegar ao quarto do O., torna-se necessário passar pelo quarto do
  Maciej (que está sempre pronto para dar dois dedos de conversa, um
  pouco para socializar e um pouco para aperfeiçoar o seu português, que
  aliás já é muito aceitável, e que tem o cuidado de não tocar saxofone
  depois das 11 da noite), meter por um corredor estreito e sem luz,
  descer dois degraus de alturas diferentes, virar à esquerda e empurrar
  uma porta que se confunde com a parede porque é da mesma cor que a
  parede.
\item
  Cruzes, parece um percurso iniciático de uma seita qualquer. Que
  rituais satânicos têm lugar nesse quarto inaces- sível?
\item
  Antes fossem rituais satânicos, amiga! Belzebu em pes- soa seria um
  alívio comparado com este tormento! Já te falei do O. vezes sem conta,
  verdade?
\item
  Eu deixei de as contar.
\item
  Testei a tua paciência com as minhas queixas repeti- das sobre a sua
  obstinação de mula, sobre as suas hesitações infantis. Durante meses,
  resisti a pronunciar o seu nome \emph{in absentia, }como se o nome
  fosse inseparável do corpo que fi- cou para trás, naquelas serras
  frias e inférteis onde crescemos, juntos e irmanados pelas inclinações
  da alma. Agora que ele se resignou a vir estudar para Lisboa para ser
  alguém na vida, em vez de jurar fidelidade eterna à insípida província
  da sua infância, com um ano de atraso relativamente a mim e relati-
  vamente ao que o bom senso ordenava, eu devia estar feliz, na minha
  cara devia residir em permanência o sorriso de quem tem tudo aquilo a
  que aspirou: saúde, palminho de cara, a vida na espantosa cidade de
  Lisboa, a pessoa que se ama.
\end{itemize}

O MESMO POETA \textbf{187}

\begin{itemize}
\tightlist
\item
  Mas a tua cara é a de quem a vida contrariou, em vez de despejar
  benesses.
\item
  Batia tudo certo, como que a pedido. A casa onde o
\end{itemize}

O. conseguiu arrendar quarto fica perto da minha, o cami- nho faz-se a
pé nas calmas mas a proximidade não é excessiva, não andamos a tropeçar
um no outro dia após dia, dá para respirar, dá para estender os braços e
não encontrar mais do que o espaço vazio. Dividíamos os nossos dias
entre as minhas aulas na Faculdade de Medicina Veterinária, as aulas
dele de gestão no ISEG, uma ou outra saída à noite, a dois ou com
amigos, sessões de \emph{poker }pela noite dentro, maratonas de séries
no AXN. Aos poucos, recuperávamos a intimidade perdida. O meu desprezo
silencioso estendia-se a todos aqueles que afirmam que no amor não
existem segundas oportunidades.

\begin{itemize}
\tightlist
\item
  Em que parte da história entra o poeta russo? Sim, porque tu já me
  contaste que o verdadeiro busílis era um poeta russo. Só faltaram os
  detalhes.
\item
  Os primeiros sinais do desvario foram ambíguos. Por exemplo, estávamos
  os dois sozinhos no quarto dele, ele olhava em redor com ar sonhador e
  convidava-me a dedicar um pen- samento às criaturas ilustres que
  tinham passado pelo quarto outrora.
\item
  Criaturas ilustres.
\item
  Estou a citar. Achas normal? Estas alusões transforma- ram-se numa
  espécie de hábito. Desejei que ele concretizasse, que chegasse aonde
  quer que fosse que queria chegar. Exas- perei-me, discutimos. Foi
  então que o seu interesse pela poesia russa do início do século XX,
  que inicialmente tomei por um capricho, assumiu as proporções de uma
  doença obsessiva. Estamos a falar do O., que não distingue um soneto
  de um haiku e que possui um historial sólido de contribuição para
\end{itemize}

\textbf{188 }ALEXANDRE ANDRADE

baixar as estatísticas de leitura em Portugal.

\begin{itemize}
\tightlist
\item
  Talvez ele queira apenas chamar a atenção. Há quem cometa loucuras
  para que os outros reparem em si.
\item
  Quando tentei ignorá-lo, desviar as conversas, a coisa partiu em
  crescendo. No final de um serão de pizza e gelado, eu ao colo dele,
  Otis Redding no iPod, eis que o O. entra em modo lamechas, fixa os
  olhos na parede e começa a desbobi- nar uma série de palermices sobre
  alguém que teria acabado os seus dias num campo de prisioneiros da
  Sibéria por ter insultado Estaline em verso, e que teria passado
  alguns meses da sua juventude ali mesmo debaixo do tecto do quarto
  onde estávamos agora, colados um ao outro, misturando os nossos
  hálitos e os nossos tédios. Eu não estava a acreditar, amiga!
\item
  Ele estava a falar a sério? Estava a gozar, não estava? Estava a falar
  a sério? A sério que estava? Diz-me que ele estava a gozar. Não estava
  a falar a sério, pois não? Só podia estar a gozar.
\item
  Ele estava a falar a sério. O meu namorado estava a dizer que o poeta
  Ossip Mandelstam (1891-1938), fundador do grupo acmeísta onde
  pontificavam Gumilyov e Akhma- tova, entre outros, viveu no Campo de
  Santa Clara, no mesmo quarto onde ele me recebe, onde trocamos
  carícias fatigadas e as histórias do dia que passou.
\item
  Claro que tentaste chamá-lo à razão.
\item
  Fiz troça, chamei-lhe criança, estalei a língua com des- dém. Tudo em
  vão.
\item
  Se formos a ver, não é morte de homem. Há manias piores. Não há paixão
  sem caprichos. Isso não faz dele um mitómano. E quem te garante que
  não há um fundinho de verdade naquilo que diz?
\item
  Achas? É um delírio puro. Mas sabes o que é mais
\end{itemize}

O MESMO POETA \textbf{189}

curioso? As versões da história variam consoante os altos e bai- xos da
nossa relação. Quando estamos num mar de rosas, ele defende que
Mandelstam fez uma viagem a Portugal algures entre 1908 e 1910 e que foi
nessa altura que se instalou no quarto, tão remoto e camuflado como um
esconderijo, onde o

O. dorme agora quando não dorme em minha casa. Durante esses anos pouco
documentados na sua biografia, o poeta es- tudou em Paris e Heidelberg.
Pensar que fez uma incursão a Lisboa nesses anos pré-implantação da
República tem o seu quê de barroco, mas não é possível provar o
contrário.

\begin{itemize}
\tightlist
\item
  O encanto do inverosímil: quem lhe resiste?
\item
  Quando discutimos, sobretudo quando ele percebe que estou zangada a
  sério, por exemplo quando deixo de responder às mensagens dele, a
  história fica mais elaborada. Mete a queda em desgraça de Ossip
  Mandelstam depois de ter vindo a lume o epigrama, de uma ironia
  devastadora, dirigido a Estaline, mete os anos terríveis que se
  seguiram, a ordem de desterro para os Urais, a intercessão de
  Bukharin, uma das figuras mais poderosas do Kremlin, a comutação da
  pena, a possibilidade de escolher qualquer cidade para viver, com
  excepção das doze maiores da U.R.S.S., a instalação em Voronezh com a
  mulher, Nadezhda. O O. jura a pés juntos que foi nessa altura que
  Mandelstam, por acção de amigos e admiradores da sua obra no Ocidente,
  se refugiou sucessivamente em várias cidades da Europa, incluindo
  Lisboa, o Campo de Santa Clara, o quarto aonde ninguém consegue chegar
  sem ser guiado por alguém que conheça o caminho, e que agora cheira a
  peúgas de rapaz e está coberto com apontamentos e fotocópias de livros
  de eco- nomia e gestão.
\item
  Ena pá.
\item
  Não tem pés nem cabeça, é a coisa mais idiota que
\end{itemize}

\textbf{190 }ALEXANDRE ANDRADE

alguma vez saiu de boca de homem nascido de mulher, mas havias de ver a
convicção com que ele diz isto. Estas coisas não se fingem. Ele
acredita. Também serve para me chamar a aten- ção, sobretudo quando
percebe que estou a perder a paciência com ele, mas não é só para chamar
a atenção.

\begin{itemize}
\tightlist
\item
  Para começar, o que teria levado Mandelstam a regres- sar a Voronezh
  para viver a amargura do desterro, os anos ter- ríveis que se
  seguiram, o destino final num campo de trabalho gélido perto de
  Vladivostok?
\item
  Costuma ser nessa parte da explicação que bato com a porta ou dou dois
  gritos. Mas não duvides de que ele tem uma teoria na ponta da língua
  para explicar isso.
\item
  Amiga, olha, é assim: podia ser muito pior. Não lhe podes conceder
  essa tara? Podia ser tão pior, tão mais sinistro, de tantas maneiras!
\item
  O que é mais estranho, amiga, é que tirando isso tudo corre na
  perfeição: sinto-me bem com o O., penso muitas ve- zes num futuro com
  ele, olho-o nos olhos e vejo bondade, valor, carinho, nobreza. Mas
  como ignorar esta mancha, esta mania bizarra saliente como um
  apêndice? Estou com ele no quarto e de repente parece-me que estou a
  ver Mandelstam sentado ao parapeito a rever com entusiasmo juvenil as
  provas do seu primeiro volume de poemas (``Kamen'', ou seja ``A
  Pedra''), a namoriscar Nadezhda, a escrever uma carta à poe- tisa
  Maria Petrovykh, ou então a andar de um lado para o outro, febril,
  enquanto declama os seus versos em voz alta.
\item
  Tens de ser flexível. O amor também é isso, ser flexível.
\end{itemize}

\subsection{* * *}

O MESMO POETA \textbf{191}

\begin{itemize}
\tightlist
\item
  A casa não é feia e até está bem conservada, para os padrões
  lisboetas, mas transporta-nos imediatamente (é au- tomático) para
  séculos que já passaram. Parece uma casa feita para aristocratas
  italianos em decadência, um cenário para um filme do Visconti, estás a
  ver? E afinal é um dormitório para estudantes. O proprietário
  (contaram-me) vive nos con- fins da Irlanda e comunica com o
  administrador desta e de outras propriedades suas apenas por correio
  normal, longas cartas escritas numa caligrafia apuradíssima e repletas
  de di- vagações sobre o estado do mundo. É num quarto dessa casa, tão
  isolado que nenhum grito de lá escapa, que vive o O., de quem te falei
  tantas vezes. É um rapaz sem histórias. Vai às aulas, estuda, vai
  beber um copo com os amigos, dorme até tarde aos sábados e domingos.
  Uma vez por mês, em média, apanha um comboio e uma camioneta para ir
  visitar os pais reformados, numa terra remota e rodeada de montanhas
  que parecem intransponíveis ao visitante desprevenido. Anda com a N.,
  que tu conheces.
\item
  Assim assim. É amiga de amigos. Se nos cruzássemos na rua, não sei se
  ela me reconheceria.
\item
  Detesto pensar que existe \emph{esta coisa }entre nós. A traição
  mete-me nojo. Só comecei a olhar para o O. de outra maneira quando me
  convenci de que as coisas entre ele e a N. não tinham futuro. Estavam
  num beco sem saída, percebes? Um marasmo emocional, uma coisa sem pés
  nem cabeça. E era a própria N. que o admitia, isto não eram suposições
  minhas.
\item
  Não te estou a julgar. Olha para mim. Conta-me tudo à vontade. Alguma
  vez te julguei? Costumo julgar as pessoas?
\item
  Bem sei que não. Sou eu própria que me acuso e res- pondo às
  acusações, à vez. Até faço vozes diferentes. Ando uma pilha de nervos.
  Obrigado por me ouvires. Se calhar tens
\end{itemize}

\textbf{192 }ALEXANDRE ANDRADE

que fazer, estudar para exames.

\begin{itemize}
\tightlist
\item
  Qual de vocês é que deu o primeiro passo?
\item
  É difícil dizer... Sabes como são estas coisas. Um olhar, uma palavra
  inesperada, um sobressalto, a mão que roça na mão, coisas que levam a
  outras coisas. Lembro-me do dia, da hora, do lugar e daquilo que cada
  um de nós estava a beber. Mas sabes, se aconteceu aquilo que aconteceu
  foi porque es- tava escrito. Não adiantava tentar fugir. Não foi uma
  decisão.
\item
  Viver às escondidas não leva a lado nenhum. Vai ter com a N. e
  conta-lhe tudo. Por aquilo que me contas, a relação deles não tem
  futuro. Se as coisas estão tão mal entre ela e o O., qual é o
  problema? Ela há-de compreender.
\item
  As coisas não são assim tão simples. Eles ainda se vêem, ainda têm
  sentimentos muito fortes um pelo outro.
\item
  O O. é maior e vacinado. Se anda contigo...
\item
  Ele é um querido, é fantástico. Sinto-me tão bem quando estou com ele.
  Sinto-me especial, percebes? Vou visi- tá-lo quando sei que a N. está
  em aulas. Sei de cor o horário dela, escolho a altura das aulas
  práticas porque ela não pode faltar às aulas práticas. É quase sempre
  o Paulo quem me abre a porta. Desconfio que ele nunca sai de casa,
  está sempre lá. Cumprimenta-me sempre como se eu fosse a filha
  pródiga, acabada de chegar dos antípodas. Adoro o Paulo, ele é mesmo
  espectacular. Às vezes o Maciej está na sala a beber chá, outras vezes
  está no quarto a tocar saxofone. Quando não o vejo na sala nem ouço o
  saxofone, é porque ele não está em casa. Não me ofendo por o O. não
  vir ter comigo. Bem sei que nenhum ruído do resto da casa chega ao seu
  quarto. Sigo o corredor tortuoso às apalpadelas, bato à porta dele,
  dá-me gosto surpreendê-lo. Quando estamos sozinhos no quartinho
  minúsculo onde vive, longe do mundo, longe
\end{itemize}

O MESMO POETA \textbf{193}

do ruído e longe de tudo, é como se o tempo parasse.

\begin{itemize}
\tightlist
\item
  Afinal de contas não lhe roubaste o homem, vai ter com ela. Vai ter
  com a N. e conta-lhe tudo, e depois pergunta-
\end{itemize}

-lhe: «Estamos bem?» Ela vai compreender.

\begin{itemize}
\tightlist
\item
  Mesmo que compreendesse...
\item
  O que queres dizer com isso?
\item
  Às vezes pergunto-me... O O. por vezes tem umas ati- tudes que me põem
  a pensar.
\item
  Outra vez a história do poeta russo?
\item
  Também conheces essa história?
\item
  Claro que conheço. Não é uma frase de engate, não é um segredo íntimo.
  O O. vem com essa conversa a toda a hora. A infância de Mandelstam em
  Varsóvia e Sampetersburgo, o grupo dos acmeístas, a rejeição do
  simbolismo, a perseguição política, o exílio, a fidelidade de
  Nadezhda. Às vezes torna-se um pouco maçador. «Era aqui que ele se
  sentava, folheando o seu volume de Pushkin desconjuntado pelo uso, era
  aqui que ele pousava o samovar.» Santa paciência. Se fosse outra
  pessoa, mandava-o passear. Há coisas que só se toleram quando vêm do
  bom e velho O.
\item
  Pois é, mas às vezes ponho-me a pensar. E se houvesse alguma verdade
  naquilo que ele diz?
\item
  Achas? Se um poeta tão conhecido tivesse visitado Lis- boa, isso
  haveria de se saber. Nenhum biógrafo menciona nada parecido.
\item
  Esses anos da vida do poeta estão mal documentados. 1908, 1909, 1910,
  os estudos em França e na Alemanha, o re- gresso a Sampetersburgo...
  Quem pode jurar a pés juntos que ele não fez uma viagem além Pirenéus,
  que não visitou Lisboa? A Europa estava em paz. Mandelstam era jovem,
  espírito de aventura era coisa que não lhe faltava.
\end{itemize}

\textbf{194 }ALEXANDRE ANDRADE

\begin{itemize}
\tightlist
\item
  Admito que bate certo com o seu feitio rebelde, com o espírito que o
  levou a contribuir para revolucionar a poesia russa, para a ajudar a
  sair do beco sem saída a que a condu- zira a revolta contra o
  positivismo do século XIX, a obsessão pelo ideal, pelo incorpóreo,
  pelo espiritual. «O simbolismo russo gritou tanto e tão alto sobre o
  indizível, que o indizível começou a circular como papel-moeda»:
  sempre gostei desta citação dele. Mas, como sabes, o O. cada vez
  insiste mais na outra versão, a da escapadela mais tardia, nos anos 20
  ou 30, durante os anos de perseguição e desterro, entre um porto de
  abrigo provisório e o seguinte. Como se pode levá-lo a sério nessas
  alturas?
\item
  Tens razão, ele por vezes delira. Mas não faz por mal. Achas que faz
  isso para que reparem nele? Será um grito de socorro? Tu conhece-lo há
  muito mais tempo do que eu. Sinto-me insegura, tenho um medo horrível
  de o perder, e ainda por cima há esta história da N. Não sei o que lhe
  diga, não sei o que fazer.
\item
  São dois problemas diferentes. Não os mistures. Ficas como uma galinha
  tonta a correr de um lado para o outro. Uma coisa de cada vez.
\end{itemize}

\subsection{* * *}

\begin{itemize}
\tightlist
\item
  Só de entrar naquela casa, fico deprimido. Estás a ver o género.
  Infiltrações, tinta a lascar, cheiro a mofo e ao peixe frito dos
  vizinhos. Pagam uma fortuna por quartos pequenos e frios. Ah, mas a
  fachada tem azulejos de origem e viveram lá famílias nobres.
\end{itemize}

O MESMO POETA \textbf{195}

\begin{itemize}
\tightlist
\item
  Já lá estive montes de vezes.
\item
  Dantes entrava lá com prazer, com a certeza de ir passar um bom
  bocado. Noites de cartas, futebol, ou sim- plesmente conversas sobre
  isto e sobre aquilo. Foi por causa do Paulo que comecei a visitar a
  casa. A nossa amizade vem dos tempos do liceu. Depois, fiz-me amigo do
  O. e acabámos por ficar bastante próximos. Temos muita coisa em comum.
  Houve uma altura em que estive mesmo em baixo e nessa altura o O.
  estava lá para me apoiar, percebeu que eu estava num farrapo e não me
  largou enquanto não recuperei. Eu nunca me esqueço destas coisas.
  Portanto, vivem lá o Paulo e o O. O Maciej chegou mais tarde. Fala um
  português todo esburacado mas é um porreiro. Toca jazz numa banda.
\item
  Já não toca. Saiu da banda porque não tinha tempo de ir aos ensaios.
\item
  Não sabia isso. De qualquer maneira, era aqui que eu queria chegar:
  não me meto na vida dos outros mas custa-me estar de braços cruzados a
  assistir àquilo que se passa com o O.
\item
  É natural, ele é teu amigo. Os amigos servem para isso mesmo.
\item
  O O. deixa-me perplexo. À primeira vista é um rapaz normalíssimo,
  amigo do amigo, que leva muito a sério os estudos de Gestão sem deixar
  por isso de se divertir, de gozar a vida, de aproveitar ao máximo tudo
  o que esta magnífica cidade tem para oferecer. Mas quando alguém que o
  conhece se dá ao trabalho de penetrar esta carapaça de normalidade,
  começam as surpresas, e não são pequenas. Quem diria que ele seria
  capaz de manter duas namoradas ao mesmo tempo sem um pingo de malícia,
  com a ingenuidade de uma criança, como se tivesse amor que chegasse
  para duas pessoas e não visse mal nenhum em partilhá-lo. Vive uma vida
  dupla como
\end{itemize}

\textbf{196 }ALEXANDRE ANDRADE

quem respira, com um sorriso nos lábios. No fundo, acho que isso é uma
consequência da sua bondade. Recusa-se a que al- guém sofra por sua
causa e depois dá nisto.

\begin{itemize}
\tightlist
\item
  A N. e mais quem?
\item
  A M. esteve a desabafar comigo. Está a viver o seu idílio com o O. mas
  tem receio de melindrar a N., que afinal de contas é amiga do peito.
  Está numa pilha de nervos, não sabe para onde se virar.
\item
  Conheço mal a M., mas parece-me ser o tipo de pessoa que faz
  tempestades num copo de água.
\item
  Há razão para fazer tempestades, neste caso. A pobre coitada está
  dividida entre o dever e o prazer. É a pior situação possível. E não é
  tudo. Há ainda as fantasias do O., que não ajudam a tornar a situação
  mais simples, pelo contrário.
\item
  Estás a falar dos fantasmas de poetas malditos que ele alberga no seu
  quarto minúsculo?
\item
  Então também sabes?
\item
  Todos os amigos dele sabem, e todos os amigos dos amigos, e todos
  aqueles que frequentam a casa do Campo de Santa Clara, e todos os que
  conhecem alguém que frequenta a casa. Não é um segredo de estado. O O.
  nunca se fez rogado para espalhar aos quatro ventos as suas teorias,
  desde o círculo mais próximo ao homem da companhia do gás. Ossip Man-
  delstam e a sua suposta excursão a Lisboa, as suas excentrici- dades e
  infidelidades, a aparente submissão ao poder, seguida de provocações
  pouco subtis, a dedicação de Nadezhda, já são motivo de anedota.
\item
  A M. sente-se inquieta por ele. Não percebe como é que o O. pode
  defender factos insustentáveis. Mas ela própria sente-se tentada a
  acreditar. É natural. O amor deixa-nos capaz de acreditar em tudo. É
  uma maneira de ela se sentir
\end{itemize}

O MESMO POETA \textbf{197}

mais próxima dele.

\begin{itemize}
\tightlist
\item
  O O. consegue ser extremamente convincente. O en- tusiasmo dele
  pega-se como uma gripe. Até eu, que nunca me tinha sentido nem um
  pouco interessada por poesia russa do princípio do século, comprei uma
  antologia de Mandelstam na Feira do Livro, outra de Anna Akhmatova, e
  encomendei na Amazon a biografia escrita por Robert Littell.
\item
  Esse nome...
\item
  É o pai de Jonathan Littell, que ganhou o Goncourt.
\end{itemize}

Foi jornalista e escreveu vários \emph{bestsellers }de espionagem.

\begin{itemize}
\tightlist
\item
  Isso é interessante. Gosto do O. como se fosse meu irmão. Connosco, é
  para a vida e para a morte, e ele sabe isso. É natural que quem gosta
  dele, quem se preocupa com ele, se interrogue. O que será que o leva a
  insistir nestes disparates? Quem gosta dele não vai deixar de gostar
  dele por causa disso, quem o adora, como a M. e a N., não vai deixar
  de o adorar por causa disso, mas o que é certo é que estas
  infantilidades do
\end{itemize}

O. fazem diferença, \emph{estão ali, }tão impossíveis de ignorar como
uma bandeira, um sinal, uma nódoa, uma farpa enterrada na carne. É o
tipo de estranheza que perdura sem se atenuar. Uma pessoa não se
habitua.

\begin{itemize}
\tightlist
\item
  Pode ser uma estratégia para definir quem de facto está do seu lado.
  Uma espécie de teste, percebes?
\item
  O O. nunca foi um calculista. Basta ouvi-lo falar por uns minutos,
  basta ouvi-lo descrever o colchão de palha onde Mandelstam dormia a um
  canto, vestido com o seu único capote para se proteger da traiçoeira
  humidade atlântica, a maneira como o sol de Lisboa, ao incidir rasante
  através da janela baixa, lhe recordava as suas latitudes natais, a
  noite passada febrilmente a anotar o manifesto do poeta Mikhail Kuzmin
  (``Da Bela Claridade'') que consumou a ruptura da
\end{itemize}

\textbf{198 }ALEXANDRE ANDRADE

nova geração com o simbolismo, para termos a certeza de que ele fala com
a maior sinceridade deste mundo.

\begin{itemize}
\tightlist
\item
  Pode ser uma maneira de dizer aos outros: esta é a minha loucura, e
  quem me seguir nesta aventura louca estará comigo agora e para sempre.
  Para sempre junto a mim, entre estas quatro paredes que não são
  minhas, no meu espírito e ao alcance das minhas doces palavras. Não
  achas? Não te parece?
\end{itemize}

\subsection{* * *}

\begin{itemize}
\tightlist
\item
  Levaste tempo.
\item
  Mas estou aqui, não estou?
\item
  Estás aqui. Dá-me o teu casaco. Queres pousar a mochila?
\item
  Deixa estar, obrigada. O Paulo não está em casa?
\item
  O Paulo passa dias sem sair de casa, mas hoje por acaso foi a um
  jantar na Baixa.
\item
  E o Maciej?
\item
  Voou ontem para Varsóvia, para visitar a família.
\item
  É esquisito não ouvir o saxofone dele.
\item
  É, não é? É como se a música já fizesse parte da casa.
\item
  Estavas a estudar para os exames?
\item
  Por acaso não estava. Olha só o que eu descobri: uma lista exaustiva
  do espólio de Mandelstam no \emph{site }da biblio- teca da
  Universidade de Princeton. Correspondência com a Federação dos
  Escritores Soviéticos, uma carta dirigida às autoridades com
  pormenores do seu estado de saúde durante o desterro em Voronezh,
  notas manuscritas para uma peça radiofónica sobre Goethe, um artigo
  sobre François Villon,
\end{itemize}

O MESMO POETA \textbf{199}

um fragmento copiado da \emph{Divina Comédia }e até um extracto
bancário!

\begin{itemize}
\tightlist
\item
  Isso é interessante. Consegues aceder aos documentos
\end{itemize}

\emph{online?}

\begin{itemize}
\tightlist
\item
  Isso era pedir de mais. Só os devem disponibilizar a investigadores.
\item
  Chega aqui, O. Abraça-te a mim.
\item
  Tens as mãos frias. Esta humidade... Às vezes parece-
\end{itemize}

-me que se está melhor lá fora do que aqui dentro. Queres que te faça um
chá?

\begin{itemize}
\tightlist
\item
  Conseguiste consertar o teu jarro eléctrico?
\item
  Comprei um novo, na Rádio Popular. Faço-te um chá num instante.
\item
  Deixa estar. Deixa-te ficar aqui, por favor. Sabes, que- ria
  dizer-te... Estive com o P.
\item
  Esse tem estado muito quietinho no seu canto. Está com muito trabalho?
\item
  Mais ou menos. Ele contou-me tudo sobre a M.
\item
  O que tem a M.?
\item
  Sobre tu e a M., quero dizer.
\item
  Dou-me bem com a M., se é isso que queres dizer.
\item
  Dás-te mesmo muito bem com a M., pelo que me con- tou o P. Unha com
  carne.
\item
  Nunca te escondi nada. Aliás, não há nada para es- conder.
\item
  Olha, O., é assim: tu nunca me deste falsas esperanças e eu
  agradeço-te por isso. És honesto comigo e eu aprecio a honestidade.
  Bem sei que o papel que desempenho na tua vida é insignificante.
  Resignei-me a isso.
\item
  Não digas isso.
\item
  Contento-me em estar ao pé de ti. Não me importo de
\end{itemize}

\textbf{200 }ALEXANDRE ANDRADE

ser minúscula aos teus olhos, não me importo de ser ofuscada pela M.,
tal como não me importava de ser ofuscada pela N. Chegam-me os minutos
que me concedes, basta-me estar ao teu lado, pousada como um
pisa-papéis. Não te levo a mal que andes a namorar à esquerda e à
direita.

\begin{itemize}
\tightlist
\item
  Abraça-me e deixa de dizer tolices. Vem aqui, vá.
\item
  Não estás a perceber. Eu aceito tudo, sinto-me feliz assim. Não
  precisas de me esconder nada.
\item
  Eu não te escondo nada.
\item
  Escondeste-me que andavas a ver a M., tive de ficar a saber por
  terceiros. Mas estou-te a dizer que não faz mal.
\item
  A M. vai e vem, a N. vai e vem, gosto de as receber e de beber chá e
  de recitar poesia com as pessoas que se dão ao trabalho de me vir
  visitar a este quarto remoto neste canto perdido de Lisboa.
\item
  Chá e versos. És um querido, sabias? A sério, não faz mal. Não tens de
  te justificar. Sabes qual é a única coisa que me custa a suportar, no
  meio disto tudo?
\item
  O que é que te custa a suportar?
\item
  O que me custa é ter de partilhar estes poucos me- tros quadrados com
  fantasmas de poetas russos. É isso que me custa. Parece-me que os
  vejo: no parapeito da janela, sentados a um canto, abraçados às
  pernas, deitados ao comprido, cis- mando sobre as profundezas do
  sofrimento humano e sobre os acidentes da alma.
\item
  Não há fantasmas aqui.
\item
  Às vezes pareces capaz de evocar esses espíritos, como um médium.
\item
  Não são espíritos. São as minhas teorias. Pouco me importa que os
  outros as achem ridículas. Eu sei o que sei.
\item
  Acreditas realmente que Ossip Mandelstam em pessoa
\end{itemize}

O MESMO POETA \textbf{201}

foi teu antecessor na qualidade de inquilino deste quarto frio e
adorável? Não estarás, talvez sem querer, a povoar este espaço anónimo
com presenças de figuras do passado que te são que- ridas para
contrariar o efeito destas paredes anónimas, deste espaço que já foi de
tanta gente e que nunca será só teu?

\begin{itemize}
\tightlist
\item
  Queres sair? Podíamos ir tomar um café lá fora. A noite está tão
  bonita.
\item
  Não tens de dizer seja o que for. Não quero que te expliques. Tenho o
  coração ao pé da boca. Digo o que sinto, e é só isso.
\item
  Leva o casaco, pelo sim pelo não. Vai à frente, este corredor é tão
  estreito que não cabem nele duas pessoas lado a lado. O que terá
  passado pela cabeça do arquitecto? Já te contei que viveram aqui
  famílias nobres?
\item
  Lisboa desiludiu-te, O.? É isso que se passa? Tens sau- dades da tua
  terra, da paisagem agreste da montanha, dos es- paços a perder de
  vista? Estás farto dos estudos, estás farto da Gestão de Empresas?
  Gostas de alegar coisas estranhas e surpreendentes para chocar os
  outros, para distinguir aqueles que te amam aconteça o que acontecer
  daqueles que se afas- tam ao primeiro comportamento estranho? É isso?
  Desconfias da intimidade excessiva, e por isso tentas criar um efeito
  de distanciamento?
\item
  Reparaste no friso de azulejos? É tão antigo como a própria casa.
\item
  Quem sou eu para ti? Um divertimento? Uma amiga de ocasião? Mais uma
  confidente, muito ajuizada à espera da sua vez na fila, atrás de todas
  as outras? Seja o que for, sinto-me bem assim. Mas quero que sejas tu
  a dizer-mo, sem máscaras, olhos nos olhos, sem uma legião de espectros
  a fazer coro contigo.
\item
  Vamos lá para fora, vamos apanhar um pouco de ar
\end{itemize}

\textbf{202 }ALEXANDRE ANDRADE

fresco. Aqui dentro uma pessoa até abafa.

\begin{itemize}
\tightlist
\item
  Não quero sair, quero ficar dentro de casa. Aqui mes- mo, nesta sala
  demasiado grande e vazia. É aqui que me vais dizer. Olha para mim.
  Olha para mim. Não te peço nada, mas preciso de saber.
\item
  Nesta sala também está bem. Sinto-me sempre bem nesta sala. Esta sala
  já viu muita coisa, muitas décadas, muita gente. Sinto-me muito bem
  contigo. O que é que precisas de saber?
\item
  A verdade. A tua verdade. Tenho medo de usar a pa- lavra ``alma''.
\end{itemize}

\emph{«Gelo da Primavera, gelo celestial,/Gelo primevo, nuvens,}

\emph{lutadores de encanto --- /Silêncio, que levam uma nuvem pela
rédea.»}

OUTUBRO-NOVEMBRO 2011

Tradução dos versos de Mandelstam: Nina Guerra e Filipe Guerra
(Antologia \emph{Guarda Minha Fala para Sempre, }Assírio \& Alvim, 1996
-- página 221)

Os comentários sobre a salubridade e conforto do hotel onde Mário de
Sá-Carneiro viveu os seus últimos momentos, no conto ``Rua da Velha
Lanterna'', foram retirados do site \textless{}\emph{www.tripadvisor.com}\textgreater{}.

O conto ``In Absentia'' foi anteriormente publicado no n.º 134 da
revista Ler. O conto ``Quarto Escuro'' foi anteriormente publicado no
n.º 1 da revista Forma de Vida.

ALEXANDRE ANDRADE nasceu em 1971 em Lisboa, onde reside. É professor na
Faculdade de Ciências da Universidade de Lisboa. Publi- cou os romances
``Benoni'' (Editorial Notícias, 1997) e ``Aqui Vem o Sol'' (Quasi, 2005)
e as recolhas de contos ``As Não-Metamorfoses'' (Errata, 2004) e ``Cinco
Contos Sobre Fracasso e Sucesso'' (Má Criação, 2005). Mantém o blog
``umblogsobrekleist'' (umblogsobrekleist.blogspot.pt).

\textbf{FÉLIX FÉNÉON LIMA BARRETO}

\textbf{GOTTLIEB STEPHANIE DER JÜNGERE RUBÉN DARÍO}

\textbf{ALEXANDRE ANDRADE ROBERTO ARLT CHARLES CROS JOHANN PETER HEBEL
ALPHONSE ALLAIS LÉON GENONCEAUX}

COORDENAÇÃO: RUI MANUEL AMARAL
